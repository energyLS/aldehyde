% Guidelines: Simple, easy to understand for non-techs, max 1000 char including spaces, 1/2 paragraphs

Green hydrogen is an energy carrier providing many possible applications supporting the sustainable energy transition. 
This versatility drives the global hydrogen demand, putting future hydrogen exporters in the spotlight. 
However, the export of relevant green hydrogen volumes requires large electricity demands in the exporting country's energy system which has major effects on their energy prices, domestic climate mitigation and energy-related emissions, often neglected by hydrogen importers focussing on solely costs and potentials.

We highlight the importance of the exporter's perspective by analyzing numerous scenarios of hydrogen export volumes, domestic mitigation levels and hydrogen regulations. In fact, both hydrogen exports and domestic mitigation show synergies in the exporter's energy system. 
% However, we show that there are also risks with respect to rising domestic electricity prices.
% which hydrogen regulation hedges against and provides a powerful policy instrument to support the domestic energy transition.
Strict hydrogen regulation hedges domestic electricity consumers against rising prices and provides a powerful policy instrument to support the domestic energy transition.
%  by using the large electricity demand necessary for green hydrogen exports. 


