
\subsection{Moroccan energy model}
Here I describe the application of PyPSA-Earth \cite{Parzen2022} to Morocco and the sector extension.
\subsubsection{EmissioSchumm2018n targets}
Describe the emission targets of Morocco. In general: Low per capita emissions
Depicted in Figure \ref{fig:morocco_em}.
Insights from \cite{CAT2021}:
\begin{itemize}
    \item No net-zero target.
    \item 2030: 115 MtCO2e (NDC, excluding LULUCF, Unconditional target)
    \item 2030: 75 MtCO2e (NDC, excluding LULUCF, Conditional target)
    \item 2030: 52\% Renewable energy share in installed electric capacity
\end{itemize}



\begin{figure}[h!]
    \centering
    \includegraphics[width=\linewidth]{morocco_em.pdf}
    \caption{Morocco historical emissions and targets.}
    \label{fig:morocco_em}
\end{figure}




\subsubsection{Green hydrogen}
Emissions vs costs. Which definition will I use? how is it implemented?
Use current papers (\cite{Brauer2022}, \cite{Ruhnau2022}, \cite{Zeyen2022a}) to base my decision on
Regulatory dimensions, adapted from \cite{Brauer2022}:
\begin{itemize}
    \item Temporal correlation (1h, 1d, 1a - loose) 
    \item Geographical correlation (same location - loose)
    \item Origin (100\% new RES - loose)
\end{itemize}
How does my scenario fit in these dimensions?
\subsection{Hydrogen export}
What amount of hydrogen should be used? X TWh?
\subsection{Islanded hydrogen production}
\subsection{Integrated hydrogen production}

This refers to \ref{sec:intro}

Citation is here \cite{Neumann2021}

As seen in the Figure \ref{fig:methods_scenarios}, the cross-reference works.

\begin{figure}[h!]
    \centering
    \includegraphics[width=\linewidth]{miro_methods}
    \caption{Two different scenarios, assuming integrated and islanded hydrogen production.}
    \label{fig:methods_scenarios}
\end{figure}

