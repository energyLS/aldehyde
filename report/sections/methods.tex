\subsection{Moroccan energy model}
\label{subsec:moroccan_model}
The Moroccan energy model is based on the PyPSA-Earth workflow \cite{Parzen2022} and 
adds a sector extension similar to \cite{Brown2018a}.
This capacity extension model optimizes the generation, transmission and storage of Moroccos energy system.


\subsubsection{Energy demand}
The energy demand is obtained from the "LEAP" model \cite{Heaps2022}.
TBD: More infos on MenaFuels: see \cite[p. 35]{Ersoy2022}, better: "Teilbericht 9"

The various sectors are (TBD clarify each bullet point: what is the energy prediction based on):
\begin{itemize}
    \item Electricity demand based on \cite{Parzen2022}. Electricity demand also stated in \cite[primary source 25]{Boulakhbar2020}
    \item Industry demand
    \item Transport demand
    \item Heating demand (cooling?) 
    \item Regionalisation of energy demands
\end{itemize}


\subsubsection{Renewable Energy Sources}
Atlite, using data pipeline from \cite{Parzen2022}. TBD evaluate sources and include in csv file.

\begin{itemize}
    \item Benchmark potentials to other studies (e.g. IRENA, MenaFuels which uses "EnDat", HyPAT provided on online map?, ...)
    \item MenaFuels \cite[p. 25]{Ersoy2022} (better source: Teilbericht 10)
    \item See \cite{Ersoy2022} (better primary source) page 26 on Renewable generation in Morocco: Currently 37\% of the installed capacity from RE
    \item See \cite[p. 5]{Boulakhbar2020} for list of hydro projects (current \& future)
    \item See \cite[primary 10,13,17]{Boulakhbar2020} for MAR potentials
    \item See \cite[primary 13, 38]{Boulakhbar2020} wind power of 6000 MW
    \item In \cite[p. 6]{Boulakhbar2020} they outline why biomass/biogas is not exploited yet in MAR
\end{itemize}



\subsubsection{Conventional electricity generation}
We use: ... (TBD: List our sources)
Other sources: S\&P Global Kraftwerksliste (used by Fraunhofer IEE).
See \cite[p. 26]{Ersoy2022}  on conventional generation in Morocco.


\subsubsection{Electricity (and gas) networks}
The electricity network is based on the PyPSA-Earth workflow using OSM data as main source \cite{Parzen2022}.
3,000 km of 400 kV, 9,000 km of 225 kV, 147 km of 150 kV lines 
11,780 km of 60 kV lines \cite[p. 6, primary 44]{Boulakhbar2020}


\subsubsection{Green hydrogen}
TBD: Describe and use new "green hyrogen" definition by European Comission.
Emissions vs costs. Which definition will I use? how is it implemented?
Use current papers (\cite{Brauer2022}, \cite{Ruhnau2022}, \cite{Zeyen2022}) to base my decision on
Regulatory dimensions, adapted from \cite{Brauer2022}:
\begin{itemize}
    \item Temporal correlation (1h, 1d, 1a - loose) 
    \item Geographical correlation (same location - loose)
    \item Origin (100\% new RES - loose)
\end{itemize}
How does my scenario fit in these dimensions?

\subsection{Water supply}
\label{subsec:water_supply}
Water supply is a crucial input for the electrolysis process. 
Morocco: 88 \% of water is used in agriculture, secondary source \cite{Ersoy2022}.
Transfer of water from North to South is planned, seconary source \cite{Ersoy2022}.
Ideas:
\begin{itemize}
    \item Detailed water analysis in HyPat: Topography -> water pumps, etc. See results in Figure \ref{fig:morocco_water}
    \item Oversize water desalination to provide water for local population/agriculture to increase public acceptance
    \item DAC: Water is a product of DAC process which could be reused for electrolysis
    \item Direct electrolysis of seawater: In development but probably not on big scale in the next 10-15 years
\end{itemize}



\subsection{Hydrogen export}
\label{subsec:hydrogen_export}
What amount of hydrogen should be used? X TWh? -> Source found for that. Talk to Johannes about his assumption
System boundary: Country border. MenaFuels uses closest harbour as system boundary
Nador West Med: Currently under construction and maybe operational in 2022 \cite{Ersoy2022}


\subsection{Cost of hydrogen production}
\label{subsec:cost_of_hydrogen_production}

The cost of hydrogen production is calculated as follows \cite{Zeyen2022}:
% Add formula here
\begin{equation}
    \label{eq:cost_of_hydrogen_production}
    C_\mathrm{H2}= \frac{\sum \limits_{a\in A}^{} C_a + \sum \limits_{a,t} O_{a,t} + \sum \limits_{t} P_t \cdot (im_t - ex_t)   }{d_\mathrm{H2}}
\end{equation}
%where $C_a$ is the capital cost of the asset $a$, $O_{a,t}$ is the operational cost of the asset $a$ in time step $t$, $P_t$ is the production of hydrogen in time step $t$, $im_t$ is the import of hydrogen in time step $t$ and $ex_t$ is the export of hydrogen in time step $t$.

\subsection{Islanded hydrogen production}
In the islanded hydrogen production scenario, the hydrogen production is optimised without considering the local energy system described in \ref{subsec:moroccan_model}.


\subsection{Integrated hydrogen production}
In the integrated scenario, both the hydrogen export (\ref{subsec:hydrogen_export}) and the Moroccan energy system (\ref{subsec:moroccan_model})
are optimised in an integrated approach.
The hydrogen production for export
