\begin{itemize}
    \item Description of the fully sector-coupled capacity expansion and dispatch model using PyPSA-Earth
    \item Explanation of the integrated gas and electricity network planning
    \item Overview of scenario parameters: hydrogen export volumes, climate targets, etc. cf. \ref{subsec:policyandtargets}
    \item Description of the Moroccan energy landscape and its current state
    \item Introduction to the 53 regions considered in the model
    \item Discussion of the simulation process and time resolution (3-hourly)
\end{itemize}

\subsection{Sector-coupled energy model}
\label{subsec:moroccan_model}

The sector-coupled energy model is based on PyPSA-Earth \cite{Parzen2022} and adds a sector extension similar to \cite{Brown2018a}.
This capacity extension model optimizes the generation, transmission and storage of Moroccos energy system by minimising the the total annualised system costs
constrained by e.g. climate targets or green hydrogen policies (s. Sec. \ref{subsubsec:green_hydrogen_constraint}). The geographical scope is Morocco.

\subsubsection{Energy demand}
Hazem
The energy demand is obtained from the "LEAP" model \cite{Heaps2022}.
TBD: More infos on MenaFuels: see \cite[p. 35]{Ersoy2022}, better: "Teilbericht 9"

The various sectors are (TBD clarify each bullet point: what is the energy prediction based on):
\begin{itemize}
    \item Electricity demand based on \cite{Parzen2022}. Electricity demand also stated in \cite[primary source 25]{Boulakhbar2020}
    \item Industry demand
    \item Transport demand
    \item Heating demand (cooling?) 
    \item Regionalisation of energy demands
\end{itemize}



\subsubsection{Renewable Energy Sources}
Atlite, using data pipeline from \cite{Parzen2022}.





\begin{figure}[h!]
    \centering
    \includegraphics[width=\linewidth]{../graphics_general/MAR_brownfield_el.png}
    \caption{Current capacities of electricity generation and distribution, obtained from \cite{Parzen2022} and visualization based on [TBD: add source of "documentation"]}
    \label{fig:MAR_brownfield}
\end{figure}


\subsubsection{Conventional electricity generation}
We use: ... (TBD: List our sources)
Other sources: S\&P Global Kraftwerksliste (used by Fraunhofer IEE).
See \cite[p. 26]{Ersoy2022}  on conventional generation in Morocco.



\subsubsection{Electricity (and gas) networks}
The electricity network is based on the PyPSA-Earth workflow using OSM data as main source \cite{Parzen2022}. Figure \ref{fig:MAR_brownfield} shows the current capacities of electricity generation and distribution based on the PyPSA-Earth workflow. [TBD: Describe more]
3,000 km of 400 kV, 9,000 km of 225 kV, 147 km of 150 kV lines 
11,780 km of 60 kV lines \cite[p. 6, primary 44]{Boulakhbar2020}





\subsubsection{Water supply}
\label{subsec:water_supply}
Water supply is a crucial input for the electrolysis process. 
Morocco: 88 \% of water is used in agriculture, secondary source \cite{Ersoy2022}.
Transfer of water from North to South is planned, seconary source \cite{Ersoy2022}.
Ideas:
\begin{itemize}
    \item Detailed water analysis in HyPat: Topography $\rightarrow$ water pumps, etc. See results in Figure \ref{fig:morocco_water}
    \item Oversize water desalination to provide water for local population/agriculture to increase public acceptance
    \item DAC: Water is a product of DAC process which could be reused for electrolysis
    \item Direct electrolysis of seawater: In development but probably not on big scale in the next 10-15 years
\end{itemize}


\subsection{Green hydrogen policy}
\label{subsubsec:green_hydrogen_constraint}

Note: currently my results run on monthly constraint.

A key model constraint is the requirement of the hydrogen exported from Morocco to be "green". Envisioning hydrogen offtakers from the European Union, the green hydrogen constraint applied in this study aligns with a proposal of the European Commission defining green hydrogen  based on the following criteria:


\begin{enumerate}
    \item Additionality: (from 1.1.2028 onwards)
    \item Temporal correlation: Hourly matching (from 1.1.2030 onwards, until 31.12.2029 monthly) $rightarrow$ currently the models run on monthly matching
\end{enumerate}


PPA with RE-installation


Note, the delegated act also considers the possibility of:
\begin{itemize}
    \item direct connection of RE and electrolyzers,
    \item high share of RE in power mix (>90\%) or the 
    \item avoidance of RE curtailment
\end{itemize}
to qualify for the "green" hydrogen label. These further options are not considered here, since a system-integrated electrolysis in a power system of a (current) share of RE below 90\% is the scope of this study (s. Sec. \ref{subsec:scenarios}).


% https://www.ffe.de/en/publications/how-is-green-hydrogen-defined-according-to-the-eus-delegated-act/

The green hydrogen definition of the European Commission prior to 1.1.2028 is not applied in this study, since relevant volumes of hydrogen export are expected to materialize from 2028 onwards.


additional export volume is equal to additional RE generation

The exported hydrogen 
The export product is green hydrogen:
\begin{itemize}
    \item Which "green" definition will I use? Hourly, in line with the EU Comission from 2027 onwards [TBD: Check details]
    \item 
    \item Describe implementation
    \item clarify: How does my scenario fit in these dimensions?
\end{itemize}

Use current papers (\cite{Brauer2022}, \cite{Ruhnau2022}, \cite{Zeyen2022}) to base my decision on
regulatory dimensions, adapted from \cite{Brauer2022}:
\begin{itemize}
    \item Temporal correlation (1h, 1d, 1a - loose) 
    \item Geographical correlation (same location - loose)
    \item Origin (100\% new RES - loose)
\end{itemize}


\subsection{Hydrogen export}
\label{subsec:hydrogen_export}
The amount of hydrogen to be exported is derived from section \ref{subsec:policyandtargets} and implemented as an exogenous parameter in the range of 0 - 200 TWh (s. Sec. \ref{subsec:scenarios}). Similar to (MenaFuels) \cite{Ersoy2022} the system boundary is the country border of Morocco, hence not considering various transport options (as shipping or pipeline) or further conversions (Methane, Ammonia, etc.) [TBD: justify]
Nonetheless, the demand pattern of ship export (landing, loading, travel time) and the resulting energy system implications (compared to a constant export profile, closer to a pipeline behaviour) to Morocco are considered here and further investigated in Section \ref{subsec:sensitivity}.
The export of hydrogen is allowed via a range of selected ports in Morocco [TBD: provide table in supplementary]. 

The energy system model allows an endogenous spatial export decision, meaning that the total demand (and export profiles) are exogenous, but the model chooses the  cost-optimal export location(s). At each port, the model has the option to build a hydrogen underground or steel tank depending on local geological conditions [TBD: compare open data for underground storage].

Notes on export in PyPSA-Earth-Sec:
\begin{itemize}
    \item Export storage: on, off, costs
    \item Profile: Shipping: on, off, profile
    \item Only hydrogen
    \item Endogenous port decision
\end{itemize}

Points of export:
\begin{itemize}
    \item Only allow export by ship, or is pipeline an option?
    \item Provide map with harbours (\& their capacities?)
    \item Harbour Nador West Med: Currently under construction and maybe operational in 2022 \cite{Ersoy2022}
\end{itemize}


\subsection{Cost of hydrogen production}
\label{subsec:cost_of_hydrogen_production}

The cost of hydrogen production is calculated as follows \cite{Zeyen2022}:
\begin{equation}
    \label{eq:cost_of_hydrogen_production}
    C_\mathrm{H2}= \frac{\sum \limits_{a\in A}^{} C_a + \sum \limits_{a,t} O_{a,t} + \sum \limits_{t} P_t \cdot (im_t - ex_t)   }{d_\mathrm{H2}}
\end{equation}
"where $C_a$ is the capital cost of the asset $a$, $O_{a,t}$ is the operational cost of the asset $a$ in time step $t$, $P_t$ is the production of hydrogen in time step $t$, $im_t$ is the import of hydrogen in time step $t$ and $ex_t$ is the export of hydrogen in time step $t$" \cite{Zeyen2022}.


Alternative method of calculating the LCOH in an integrated system:
\begin{equation}
    \label{eq:cost_of_hydrogen_production_simple}
    C_\mathrm{H2}=\frac{TSC_\mathrm{export}-TSC_\mathrm{noexport}}{d_\mathrm{H2export}}
\end{equation}
Equation \ref{eq:cost_of_hydrogen_production_simple} provides another approach where the total system cost of the integrated system without hydrogen export $TSC_\mathrm{noexport}$ is subtracted from the total system cost of the integrated system with hydrogen export $TSC_\mathrm{export}$ to get the total increase in cost, then divided by the hydrogen export volume $d_\mathrm{H2export}$.

Note, equation \ref{eq:cost_of_hydrogen_production} is not valid for the integrated system (?),
hence the cost of hydrogen of the integrated system is defined as the marginal price of hydrogen (taken from PyPSA).
Questions:
\begin{itemize}
    \item Marginal price of all hydrogen buses or just export?
    \item "Weighted" marginal price makes more sense
    \item Is the equation \ref{eq:cost_of_hydrogen_production} valid?
\end{itemize}


\subsection{Scenarios}
\label{subsec:scenarios}
To examine the conflicts and synergies of hydrogen export and national energy transition, we sweep along two dimensions:
\begin{enumerate}
    \item The hydrogen export volume varies from 0 - 200 TWh, going beyond Moroccos hydrogen export ambitions of 110 TWh [TBD: provide exact number] (cf. sec. \ref{subsec:policyandtargets}) and
    \item the emission reduction between 0 - 100\% based on Morocco's current emissions of 72 MtCO2e (cf. sec. \ref{subsec:policyandtargets}).
\end{enumerate}
This two-dimensional scenario space leads to 110 model runs. 
A 3-hourly resolution is chosen to capture energy system dynamics (e.g. RE generation profiles, energy storage operation) with it's diurnal and seasonal variations in energy supply and demand. [TBD: see Neumann for justification]
A high spatial resolution of 53 nodes represents the geographical heterogeneity of RE resources, demand centers and energy networks. Furthermore, a high spatial resolution provides insights on land use conflicts and competing RE resources as well as a spatial differentiation of export ports. Since the 53 nodes align with the GADM (Global Administrative Areas) regions, policy advisement can be tailored to specific regions considering local needs and constraints.

TBD: Elaborate more on scoping decisions
\begin{itemize}
    \item Update to 2035/2040, or even time dependent?
    \item Line expansion?
    \item Shipping profile?
    \item SMR, CO2 network
    \item Explain Brownfield here (s. Fig. \ref{fig:MAR_brownfield})?
\end{itemize}



\subsection{Miscellaneous}

\subsubsection{Islanded hydrogen production}
In the islanded hydrogen production scenario, the hydrogen production is optimised without considering the local energy system described in \ref{subsec:moroccan_model}.

\subsubsection{Integrated hydrogen production}
In the integrated scenario, both the Moroccan energy system (\ref{subsec:moroccan_model}) and the hydrogen export (\ref{subsec:hydrogen_export})
are optimised in an integrated approach.


\subsubsection{Literature: Potentials}
\begin{itemize}
    \item Benchmark potentials to other studies (e.g. IRENA, MenaFuels which uses "EnDat", HyPAT provided on online map?, ...)
    \item MenaFuels \cite[p. 25]{Ersoy2022} (better source: Teilbericht 10)
    \item See \cite{Ersoy2022} (better primary source) page 26 on Renewable generation in Morocco: Currently 37\% of the installed capacity from RE
    \item See \cite[p. 5]{Boulakhbar2020} for list of hydro projects (current \& future)
    \item See \cite[primary 10,13,17]{Boulakhbar2020} for MAR potentials
    \item See \cite[primary 13, 38]{Boulakhbar2020} wind power of 6000 MW
    \item In \cite[p. 6]{Boulakhbar2020} they outline why biomass/biogas is not exploited yet in MAR
    \item Check the RePP database \cite{Peters2023}
\end{itemize}