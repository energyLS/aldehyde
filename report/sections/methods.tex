\begin{itemize}
    \item Description of the fully sector-coupled capacity expansion and dispatch model using PyPSA-Earth
    \item Explanation of the integrated gas and electricity network planning
    \item Overview of scenario parameters: hydrogen export volumes, climate targets, etc.
    \item Description of the Moroccan energy landscape and its current state
    \item Introduction to the 53 regions considered in the model
    \item Discussion of the simulation process and time resolution (3-hourly)
\end{itemize}

\subsection{Moroccan energy model}
\label{subsec:moroccan_model}
The Moroccan energy model is based on the PyPSA-Earth workflow \cite{Parzen2022} and 
adds a sector extension similar to \cite{Brown2018a}.
This capacity extension model optimizes the generation, transmission and storage of Moroccos energy system.


\subsubsection{Energy demand}
The energy demand is obtained from the "LEAP" model \cite{Heaps2022}.
TBD: investigate OSeMOSYS starter kit.
TBD: More infos on MenaFuels: see \cite[p. 35]{Ersoy2022}, better: "Teilbericht 9"

The various sectors are (TBD clarify each bullet point: what is the energy prediction based on):
\begin{itemize}
    \item Electricity demand based on \cite{Parzen2022}. Electricity demand also stated in \cite[primary source 25]{Boulakhbar2020}
    \item Industry demand
    \item Transport demand
    \item Heating demand (cooling?) 
    \item Regionalisation of energy demands
\end{itemize}



\subsubsection{Renewable Energy Sources}
Atlite, using data pipeline from \cite{Parzen2022}. TBD evaluate sources and include in csv file.

\begin{itemize}
    \item Benchmark potentials to other studies (e.g. IRENA, MenaFuels which uses "EnDat", HyPAT provided on online map?, ...)
    \item MenaFuels \cite[p. 25]{Ersoy2022} (better source: Teilbericht 10)
    \item See \cite{Ersoy2022} (better primary source) page 26 on Renewable generation in Morocco: Currently 37\% of the installed capacity from RE
    \item See \cite[p. 5]{Boulakhbar2020} for list of hydro projects (current \& future)
    \item See \cite[primary 10,13,17]{Boulakhbar2020} for MAR potentials
    \item See \cite[primary 13, 38]{Boulakhbar2020} wind power of 6000 MW
    \item In \cite[p. 6]{Boulakhbar2020} they outline why biomass/biogas is not exploited yet in MAR
    \item Check the RePP database \cite{Peters2023}
\end{itemize}

Figure \ref{fig:potentials} displays the potentials of solar, wind onshore and CSP 
of PyPSA-Earth-Sec compared to other studies. The authors of "OSeMOSYS" 
\cite{Cannone2021} base their potential estimations on \cite[primary: 13,18,19]{Cannone2021}
The project MenaFuels/Ersoy2022 \cite{Ersoy2022} finds potentials of ...
TBD: remove ChatGPT which was added for fun.

\begin{figure}[h!]
    \centering
    \includegraphics[width=\linewidth]{../graphics_general/potentials_comparison/barplot_potentials.pdf}
    \caption{Determined potential of solar, wind onshore and CSP in PyPSA-Earth-Sec compared to other studies.}
    \label{fig:potentials}
\end{figure}


\subsubsection{Conventional electricity generation}
We use: ... (TBD: List our sources)
Other sources: S\&P Global Kraftwerksliste (used by Fraunhofer IEE).
See \cite[p. 26]{Ersoy2022}  on conventional generation in Morocco.


\subsubsection{Electricity (and gas) networks}
The electricity network is based on the PyPSA-Earth workflow using OSM data as main source \cite{Parzen2022}.
3,000 km of 400 kV, 9,000 km of 225 kV, 147 km of 150 kV lines 
11,780 km of 60 kV lines \cite[p. 6, primary 44]{Boulakhbar2020}


\subsubsection{Green hydrogen}
The export product is green hydrogen:
\begin{itemize}
    \item Which "green" definition will I use? Currently: Yearly constraint
    \item Describe implementation
    \item Rely on European Comission
    \item clarify: How does my scenario fit in these dimensions?
\end{itemize}

Use current papers (\cite{Brauer2022}, \cite{Ruhnau2022}, \cite{Zeyen2022}) to base my decision on
regulatory dimensions, adapted from \cite{Brauer2022}:
\begin{itemize}
    \item Temporal correlation (1h, 1d, 1a - loose) 
    \item Geographical correlation (same location - loose)
    \item Origin (100\% new RES - loose)
\end{itemize}


\subsection{Water supply}
\label{subsec:water_supply}
Water supply is a crucial input for the electrolysis process. 
Morocco: 88 \% of water is used in agriculture, secondary source \cite{Ersoy2022}.
Transfer of water from North to South is planned, seconary source \cite{Ersoy2022}.
Ideas:
\begin{itemize}
    \item Detailed water analysis in HyPat: Topography $\rightarrow$ water pumps, etc. See results in Figure \ref{fig:morocco_water}
    \item Oversize water desalination to provide water for local population/agriculture to increase public acceptance
    \item DAC: Water is a product of DAC process which could be reused for electrolysis
    \item Direct electrolysis of seawater: In development but probably not on big scale in the next 10-15 years
\end{itemize}



\subsection{Hydrogen export}
\label{subsec:hydrogen_export}
The amount of hydrogen to be exported is derived from section \ref{subsec:policyandtargets}.
The system boundary is the country border of Morocco, in 
line with the MenaFuels project \cite{Ersoy2022}.

Points of export:
\begin{itemize}
    \item Only allow export by ship, or is pipeline an option?
    \item Provide map with harbours (\& their capacities?)
    \item Harbour Nador West Med: Currently under construction and maybe operational in 2022 \cite{Ersoy2022}
\end{itemize}


\subsection{Cost of hydrogen production}
\label{subsec:cost_of_hydrogen_production}

The cost of hydrogen production is calculated as follows \cite{Zeyen2022}:
\begin{equation}
    \label{eq:cost_of_hydrogen_production}
    C_\mathrm{H2}= \frac{\sum \limits_{a\in A}^{} C_a + \sum \limits_{a,t} O_{a,t} + \sum \limits_{t} P_t \cdot (im_t - ex_t)   }{d_\mathrm{H2}}
\end{equation}
"where $C_a$ is the capital cost of the asset $a$, $O_{a,t}$ is the operational cost of the asset $a$ in time step $t$, $P_t$ is the production of hydrogen in time step $t$, $im_t$ is the import of hydrogen in time step $t$ and $ex_t$ is the export of hydrogen in time step $t$" \cite{Zeyen2022}.


Alternative method of calculating the LCOH in an integrated system:
\begin{equation}
    \label{eq:cost_of_hydrogen_production_simple}
    C_\mathrm{H2}=\frac{TSC_\mathrm{export}-TSC_\mathrm{noexport}}{d_\mathrm{H2export}}
\end{equation}
Equation \ref{eq:cost_of_hydrogen_production_simple} provides another approach where the total system cost of the integrated system without hydrogen export $TSC_\mathrm{noexport}$ is subtracted from the total system cost of the integrated system with hydrogen export $TSC_\mathrm{export}$ to get the total increase in cost, then divided by the hydrogen export volume $d_\mathrm{H2export}$.

Note, equation \ref{eq:cost_of_hydrogen_production} is not valid for the integrated system (?),
hence the cost of hydrogen of the integrated system is defined as the marginal price of hydrogen (taken from PyPSA).
Questions:
\begin{itemize}
    \item Marginal price of all hydrogen buses or just export?
    \item "Weighted" marginal price makes more sense
    \item Is the equation \ref{eq:cost_of_hydrogen_production} valid?
\end{itemize}


\subsection{Islanded hydrogen production}
In the islanded hydrogen production scenario, the hydrogen production is optimised without considering the local energy system described in \ref{subsec:moroccan_model}.


\subsection{Integrated hydrogen production}
In the integrated scenario, both the Moroccan energy system (\ref{subsec:moroccan_model}) and the hydrogen export (\ref{subsec:hydrogen_export})
are optimised in an integrated approach.