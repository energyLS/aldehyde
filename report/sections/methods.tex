\subsection*{Sector-coupled energy model}
\label{subsec:moroccan_model}
The sector-coupled energy model of Morocco is based on the global electricity model  PyPSA-Earth \cite{Parzen2023} and the sector-coupling extension \cite{Abdel-Khalek2024} using linear optimisation and overnight scenarios. This capacity extension model optimises the generation, transmission and storage capacities of Morocco's energy system by minimising the the total annualised system costs constrained by e.g. climate targets or green hydrogen policies.
A more detailed mathematical problem formulation is provided in Section \nameref{subsec:math-formulation}, and a system overview is provided in Section \nameref{sec:system-overview}.
The model is based on the economic year of 2030, optimizing the time span of a full year. We have chosen the year 2030 to account for the relevance of hydrogen exports and domestic climate change in the near-term. However, we do not expect or imply that both substantial hydrogen export volumes and the domestic climate change mitigation fully materialize by the year 2030 given the time span required for the ramp up of the required technologies (Renewable electricity, hydrogen electrolyis, etc.). By fixing the modelling year to 2030 instead of modelling transformation pathways, we can isolate the effects of 0-100\% mitigation (to today) and 0-120 TWh hydrogen export and contextualize our results in a comparable setting.


\subsubsection*{Energy demand}
To obtain the annual energy demand, we use the workflow presented by Abdel-Khalek et al.\cite{Abdel-Khalek2024}. In a first step, the annual energy demand is obtained from the United Nations Statistics Database\cite{unstats2023}. The latest available data is from 2020, since these energy balances are likely to show irregularities due to COVID-19 implications, the base year in this study is 2019. Based on the annual energy demand from 2019, the energy demand of 2035 is projected according to efficiency gains and activity growth rates similar to M{\"u}ller et al.\cite{Muller2023}.
In a second step, the annual and national energy demand is distributed according to the production sites of Morocco.
In case of subsectors where no spatially resolved production sites are available (e.g. paper industry), the demand is distributed according to GDP.
Third, the annual but locationally resolved energy demand is converted to hourly demands. In this study, a constant hourly demand of the industry and agriculture sector is assumed whereas electricity, heating, transport are temporally resolved based on time series derived from Brown et al.\cite{Brown2018a}. The resulting temporal and spatial energy demands of the sectors:
\begin{itemize}
    \item Residential,
    \item Services,
    \item Industry,
    \item Transport (road, rail, aviation and navigation) and
    \item Agriculture
\end{itemize}
are integrated in the sector coupled energy model. The domestic energy services per sector remain constant throughout the scenarios presented in the \nameref{sec:intro}, regardless of endogenous hydrogen export volumes or national carbon emission reductions. However, while energy services (e.g., demand for transportation or heating) are fixed, the primary and final energy consumption are subject to climate change mitigation targets. To reflect the impact of increasing shares of battery electric vehicles (BEV) in the Moroccan energy system, the share of BEVs is linked to emission reduction ambitions (s. Fig. \ref{fig:bev_diffusion}) from 2\% (today's levels) up to a share of 88\% at 100\% domestic climate change mitigation in accordance with Rim et al.\cite{Rim2021}.



\subsubsection*{Renewable Energy Sources}
The sector-coupled energy model follows the data pipeline of Parzen et al.\cite{Parzen2023} by incorporating solar PV, onshore wind and hydro power using the open source package Atlite \cite{Hofmann2021}.
Atlite obtains technology-specific time series of the weather year 2013 based on the ERA5 reanalysis data \cite{Hersbach2020}. In addition to the time series, Atlite calculates land availabilities and linked potentials using the Copernicus Global Land Service \cite{Buchhorn2020}.


\begin{figure*}[t]
    \centering
    \includegraphics[width=0.7\linewidth]{../graphics_general/brownfield_capacities.pdf}
    \caption{Current capacities of electricity generation and distribution, obtained from \cite{Parzen2022} and visualization based on \cite{Horsch2018}. Morocco's electricity generation portfolio is currently dominated by fossil generation (coal and gas), includes some hydropower plants and increasing but still minor capacities of onshore wind and solar PV. Boundaries depicted are based on the Global Administrative Areas and are intended for illustrative purposes only, not implying territorial claims.
    }
    \label{fig:MAR_brownfield}
\end{figure*}


\subsubsection*{Conventional electricity generation}
The conventional electricity generators are obtained from the \textit{powerplantmatching} tool \cite{Powerplantmatching2019}. \textit{Powerplantmatching} uses various input sources as OpenStreetMap2022, Global Energy Monitor, IRENA \cite{IRENA2022, OpenStreetMap2022, GlobalEnergyMonitor} to download, filter and merge the datasets. 
In this study, \textit{powerplantmatching} is applied to Morocco delivering the conventional power plants.

\subsubsection*{Electricity networks and gas pipelines}
The electricity network is based on the PyPSA-Earth workflow using OpenStreetMap2022 data \cite{OpenStreetMap2022} as presented in Parzen et al.\cite{Parzen2023}. First, the data is downloaded, filtered and cleaned. Second, a meshed network dataset including transformers, substations, converters as well as HVAC and HVDC components is build\cite{Parzen2023}. 
The brownfield gas pipeline infrastructure of Morocco is not considered in this study. The Maghreb-Europe pipeline passing through Morocco is the only pipeline in operation but currently subject to political disputes \cite{Rachidi2022}. Further proposed projects are excluded due to  miniscule capacity compared to Morocco's energy system (Tendrara LNG Terminal, 100 million cubic meters per year or uncertain commissioning dates (Nigeria-Morocco Pipeline, start year 2046 \cite{GEM2023b}).


\subsection*{Brownfield model and capacity expansion}
\label{brownfield_model}
Following the workflow of Parzen et al. \cite{Parzen2023}, Figure \ref{fig:MAR_brownfield} shows the current capacities of electricity generation and distribution of Morocco. Based on this brownfield electricity system, the sector-coupled energy model allows the capacity expansion of storage, (electricity networks), hydrogen pipelines, renewable and conventional generators.

\subsection*{Technology and cost assumptions}
\label{subsec:tech_assump}
All technology and cost assumptions are based on the year 2030, taken from the \href{https://github.com/pypsa/technology-data}{technology-data repository}. This study applies version 0.4.0 of the repository, which is available at \href{https://github.com/PyPSA/technology-data/blob/v0.4.0/outputs/costs_2030.csv}{technology-data v0.4.0.}
Based on a comprehensive survey of industry and academia in 2024 by Fernandez et al.\cite{Fernandez2024}, which reports a combined Market Risk Premium and Risk-Free Rate of 12.9\% for Morocco, we follow their findings and apply a discount rate of 13\% in this study. While IRENA\cite{IRENA2023} reports a technology-specific interest rate of 9.1\% for utility-scale PV in Morocco, we apply a more conservative approach following Fernandez et al.\cite{Fernandez2024} taking into account possible higher risks of less-established technologies as water electrolysis and hydrogen infrastructure compared to utility-scale solar PV.



\subsection*{Water supply}
\label{subsec:water_supply}
Hydrogen production through electrolysis requires fresh water, alternatives as the direct use of high saline water sources are still in the experimental stages \cite{Tong2020}. The depletion of freshwater resources and it's competition with other water uses is a concern, especially in regions such as Morocco which is ranked 27th among the world's most water-stressed countries \cite{Maddocks2015}. 
Sea Water Reverse Osmosis (SWRO) emerges as a feasible solution to address this issue. In this model, the additional water cost of 0.80 €/\si{\cubic\metre} through desalination and transport is considered in line with the base scenario for Morocco in 2030 in Caldera et al.\cite{Caldera2020}. These findings are comparable to Kettani et al.\cite{Kettani2020} and Caldera et al.\cite{Caldera2016} stating the water costs of 1\$/$m^3$ resp. 0.60 - 1.50 €/\si{\cubic\metre} for Morocco in 2030. Given a water demand of 9~$m^3/kg_{hydrogen}$
\cite{Hampp2023}, the additional costs for electrolysers result in 0.216~$MWh/kg_{hydrogen}$ (LHV).

However, the challenge of brine disposal presents environmental concerns due to treatment chemicals and high salinity as discussed by Thomann et al. \cite{Thomann2022}, Dresp et al. \cite{Dresp2019} and Tonelli et al. \cite{Tonelli2023}, highlighting the importance of a careful site selection of desalination plants. A minimum distance of four kilometers from marine protected areas is recommended in Thomann et al. \cite{Thomann2022}.



\subsection*{Green hydrogen policy}
\label{subsec:green_hydrogen_constraint}


A key constraint on the model is that the hydrogen exported from Morocco requires to meet sustainability criteria ("green" hydrogen). A variety of studies looks into various dimensions (temporal, geographical, electricity origin) of green hydrogen and their trade-offs \cite{Brauer2022, Ruhnau2022, Zeyen2024}.
Envisioning hydrogen offtakers from the European Union, the green hydrogen constraint applied in this study aligns with the \emph{Delegated regulation on Union methodology for RFNBOs} of the European Commission \cite{Commission2023} defining green hydrogen  based on the following criteria:

\begin{enumerate}
    \item Additionality: The electricity demand of electrolysers must be provided by additional RE power generation (less than 3 years before the installation of electrolysers, from 1.1.2028 onwards),
    \item Temporal correlation: Monthly matching until 31.12.2029, and hourly matching from 1.1.2030 onwards and
    \item Geographical correlation
\end{enumerate}
which are required for Power Purchase Agreements (PPA) with RE-installation. Apart from the PPAs, the delegated act also considers the possibility of:
\begin{itemize}
    \item direct connection of RE and electrolysers,
    \item high share of RE in power mix ($>$ 90\%) or the
    \item avoidance of RE curtailment
\end{itemize}
to qualify as "green" hydrogen. These further options are not considered here, since a system-integrated electrolysis in a power system of a (current) share of RE below 90\% is the scope of this study. The sole focus on RE curtailment and hence low total hydrogen volumes is not applicable due to expected hydrogen exports of up to multiple times of Morocco's domestic electricity demand. The green hydrogen definition of the European Commission prior to 1.1.2028 is not applied in this study, since relevant volumes of hydrogen export are expected to materialize from 2028 onwards and is excluded by the scope of this study presented in the \nameref{sec:intro}).


\subsection*{Hydrogen export}
\label{subsec:hydrogen_export}
The amount of hydrogen to be exported or further synthesized for export is  implemented as an exogenous parameter in the range of 0-120~TWh/a, with sensitivity analysis up to 200 TWh/a presented in the \nameref{sec:si} section \nameref{sec:highexportsens}. The system boundary is the country border of Morocco, hence transport options (as shipping or pipeline) are not considered in detail. The profile for hydrogen (or derivatives) export assumed in this study is constant. This represents the operation of pipeline exports as well as the operation of a further hydrogen synthesis with limited flexibility. The export of hydrogen is allowed via a range of ports in Morocco.


The energy system model allows an endogenous spatial export decision, meaning that the total demand (and export profiles) are exogenous, but the model chooses the  cost-optimal export location(s). At each port, the model has the option to build a hydrogen underground or steel tank depending on domestic geological conditions.

For the electrolysis, we use alkaline electrolysers due to their maturity, durability, and lower costs compared to proton exchange membrane electrolysers \cite{Staffell2019,DEA2019TechnologyData}.
Alkaline electrolysers are well-suited for large-scale use, though they have slower start-up times and a more limited operating range   
than proton exchange membrane electrolysers \cite{Staffell2019, Gotz2016}, which, in contrast, have  shorter cell lifetimes as pointed out in Staffell et al. \cite{Staffell2019}.