
\subsection{Moroccan energy model}
\label{subsec:moroccan_model}
Here I describe the application of PyPSA-Earth \cite{Parzen2022} to Morocco and the sector extension.
Capacity expansion model based on \cite{Brown2018}


\subsubsection{Climate targets}
Describe the emission targets of Morocco. In general: Low per capita emissions
Depicted in Figure \ref{fig:morocco_em}.
Insights from \cite{CAT2021}:
\begin{itemize}
    \item No net-zero target.
    \item 2030: 115 MtCO2e (NDC, excluding LULUCF, Unconditional target)
    \item 2030: 75 MtCO2e (NDC, excluding LULUCF, Conditional target)
    \item 2030: 52\% Renewable energy share in installed electric capacity
\end{itemize}

\begin{figure}[h!]
    \centering
    \includegraphics[width=\linewidth]{morocco_em.pdf}
    \caption{Morocco historical emissions and targets.}
    \label{fig:morocco_em}
\end{figure}


\subsubsection{Energy demand}
HyPat uses "LEAP" to predict energy demand.
Clarify in each subsubsection: what is the energy prediction based on?
MenaFuels: see page 35 in \cite{Ersoy2022}, better: Teilbericht 9
\begin{itemize}
    \item Electricity demand based on \cite{Parzen2022}
    \item Industry demand
    \item Transport demand
    \item Heating demand (cooling?) 
    \item Regionalisation of energy demands
\end{itemize}


\subsubsection{Renewable Energy Sources}
Atlite, using data pipeline from \cite{Parzen2022}.
Benchmark potentials to other studies (e.g. IRENA, MenaFuels -> "EnDat", HyPAT -> online map?, ...)
MenaFuels \cite{Ersoy2022} (better source: Teilbericht 10)
See \cite{Ersoy2022} (better primary source) page 26 on Renewable generation in Morocco: Currently 37\% of the installed capacity from RE

\subsubsection{Conventional electricity generation}
We use: ...
Other sources: S\&P Global Kraftwerksliste (used by Fraunhofer IEE)
See \cite{Ersoy2022} page 26 on conventional generation in Morocco


\subsubsection{Electricity (and gas) networks}


\subsubsection{Green hydrogen}
Emissions vs costs. Which definition will I use? how is it implemented?
Use current papers (\cite{Brauer2022}, \cite{Ruhnau2022}, \cite{Zeyen2022a}) to base my decision on
Regulatory dimensions, adapted from \cite{Brauer2022}:
\begin{itemize}
    \item Temporal correlation (1h, 1d, 1a - loose) 
    \item Geographical correlation (same location - loose)
    \item Origin (100\% new RES - loose)
\end{itemize}
How does my scenario fit in these dimensions?

\subsection{Water supply}
\label{subsec:water_supply}
Water supply is a crucial input for the electrolysis process. 
Morocco: 88 \% of water is used in agriculture, secondary source \cite{Ersoy2022}.
Transfer of water from North to South is planned, seconary source \cite{Ersoy2022}.
Ideas:
\begin{itemize}
    \item Detailed water analysis in HyPat: Topography -> water pumps, etc. See results in Figure \ref{fig:morocco_water}
    \item Oversize water desalination to provide water for local population/agriculture to increase public acceptance
    \item DAC: Water is a product of DAC process which could be reused for electrolysis
    \item Direct electrolysis of seawater: In development but probably not on big scale in the next 10-15 years
\end{itemize}

\begin{figure}[h!]
    \centering
    \includegraphics[width=\linewidth]{water_supply_costs.png}
    \caption{Morocco water supply costs}
    \label{fig:morocco_water}
\end{figure}


\subsection{Hydrogen export}
\label{subsec:hydrogen_export}
What amount of hydrogen should be used? X TWh? -> Source found for that. Talk to Johannes about his assumption
System boundary: Country border. MenaFuels uses closest harbour as system boundary
Nador West Med: Currently under construction and maybe operational in 2022 \cite{Ersoy2022}



\subsection{Islanded hydrogen production}
In the islanded hydrogen production scenario, the hydrogen production is optimised without considering the local energy system described in \ref{subsec:moroccan_model}.


\subsection{Integrated hydrogen production}
In the integrated scenario, both the hydrogen export (\ref{subsec:hydrogen_export}) and the Moroccan energy system (\ref{subsec:moroccan_model})
are optimised in an integrated approach.
The hydrogen production for export


\begin{figure}[h!]
    \centering
    \includegraphics[width=\linewidth]{miro_methods}
    \caption{Two different scenarios, assuming integrated and islanded hydrogen production.}
    \label{fig:methods_scenarios}
\end{figure}

