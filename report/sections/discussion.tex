% This chapter discusses synergies and conflicts (s. Sec. \ref{subsec:synergies-conflicts}) of the dual goal of increasing hydrogen export and emission reductions. 

% In this Chapter, Section \ref{subsec:timepath} derives the policy relevance of temporal matching in dependence of mitigation and export volumes. Implications deriving from hydrogen export on domestic electricity consumers are considered in Section \ref{subsec:balancedinterests}.
% and possible policies increasing fairness are suggested in Section \ref{subsec:fairness}. 
% Limitations of this study are discussed in Section \ref{subsec:limitations}.

% General ideas
% \begin{itemize}
%     \item Identification of conflicts and synergies between hydrogen exports and energy transition
%     \item Implications for renewable energy deployment
%     \item Also: Refer to policy from MAR?
% \end{itemize}


% \subsection{Synergies and conflicts}
% \label{subsec:synergies-conflicts}
% Results from Chapter \ref{sec:results} indicate various synergies and conflicts of hydrogen export and national energy transition. The price of electricity highly depends on the climate ambition and hydrogen export volume, e.g. increasing the costs of electricity for domestic demands at medium climate ambition (60\%) and high export volumes (150 - 200 TWh). On the other hand, the green hydrogen constraint requiring the installation of renewable energies according to the hydrogen exports hedges domestic electricity consumers against higher electricity prices at lower climate ambitions. If such a constraint is not imposed on the hydrogen electrolysis, hydrogen exporters benefit from lower hydrogen prices.

% The electricity price increase flattens out at high climate ambitions, meaning domestic electricity consumers are able to avoid high economic burdens imposed by a hydrogen infrastructure in these scenarios.

% The hydrogen prices increase with hydrogen exports across all climate ambition scenarios. Nonetheless, the increase is less significant when the domestic energy system relies already on substantial amounts of hydrogen and Fischer-Tropsch products. The price of hydrogen for export is lowest when the domestic energy transition is still in the early stages and the export volumes are low -- electrolysers can run at high capacity factors paying of high upfront capital costs. 


\subsection*{Hydrogen regulation and the speed of transformation}
\label{subsec:timepath}
The transition towards climate neutrality of Morocco's energy system and the ramping up of exports up to 120 TWh/a are a matter of decade(s) and undergo a certain pathway. Based on the co-benefits identified in \nameref{subsec:benefits}, steering the transition by minimizing conflicts and amplifying co-benefits potentially reduce both the price of electricity and price of hydrogen. 

The broad scenario design of this study allows conclusions and recommendations beyond the specific Morocco case.
Various potential hydrogen exporting countries with different prerequisite can be found in the integrated analysis of benefits and burdens for domestic electricity consumers and hydrogen exporters among the dimensions of domestic mitigation, hydrogen export and hydrogen regulation.

Based on the main findings of co-benefits and hydrogen regulation, we present three possible pathways of reaching the aims of hydrogen exports and domestic mitigation complemented with tailored hydrogen regulation recommendations:
\begin{enumerate}
    \item \textbf{Quick exports and slow mitigation}: Reaching high export levels before adequately mitigating domestic emissions leads to higher total emissions, if no strict hydrogen regulation is in place. Instead, hourly matching reduces domestic electricity cost effectively and redistributes welfare from exporters to domestic consumers. Hydrogen regulation plays a decisive role in this pathway, even annual or monthly matching unleashes the \textit{price spillover} and \textit{energy spillover}. 
    \item \textbf{Balanced exports and mitigation}: A medium (or balanced) transition towards both exports and mitigation offers benefits for both domestic electricity consumers and hydrogen exporters. Domestic electricity consumers experience a short phase of higher prices (10\% mitigation and 20 TWh/a export), but are then rewarded towards higher export levels given strict (hourly) hydrogen regulation.
    \item \textbf{Slow exports and quick mitigation}: A shift of transition speed towards domestic mitigation imposes increasing cost for hydrogen exporters, while the effect of exports on domestic electricity cost is marginal. The hydrogen regulation plays only a minor role in this pathway, since the electricity system is already highly or fully renewable electricity based and neither \textit{price spillover} nor \textit{energy spillover} are effective.
\end{enumerate}

The three illustrative pathways offer certain room for manoeuvre in policy-making for various countries and their export ambitions as well as speed of domestic mitigation. However, these pathways do not reflect additional constraints as limited renewable energy ramp-up rates, the availability of workforce and bureaucratic hurdles among others. A holistic approach and well-balanced policies is required to steer the energy transition and export ambitions.




\subsection*{Balancing interests of exporters and domestic demand}
\label{subsec:balancedinterests}
The implications of hydrogen export on domestic electricity consumers are reflected in various dimensions. 

First, the price of electricity for domestic businesses and households need to be considered to enable a fair discussion and balance of burdens and opportunities. This study shows that the price of electricity highly depends on hydrogen export volumes and hydrogen regulation. The export of hydrogen in combination with strict hydrogen regulation provides the opportunity of decreasing domestic electricity prices, providing opportunities for domestic businesses and households.

Second, the substantial deployment of solar PV and onshore wind comes along with a substantial land use. Several studies \cite{Terrapon-Pfaff2019, Hanger2016} highlight the requirement of local acceptance of renewable energies in Morocco. From a German/European perspective, hydrogen imports are not necessarily the most economic option \cite{Merten2023} but also provide a way out of having to deal with national acceptance of renewable energies in Germany/Europe.
Shifting such issues of acceptance to Morocco by deploying renewable energy and scaling up their hydrogen exports raises moral questions and should be dealt with carefully. %TBD: Desertec: why did it fail.

Third, this study shows the benefits of integrating solar PV and onshore wind into the main grid for both domestic electricity consumers and hydrogen exporters. By integrating the best sites of solar PV and onshore wind, domestic electricity consumers can access the cost-cutting potentials and benefit from lower electricity prices as shown in our results. On the other hand, off-grid hydrogen generation potentially reserves the promising electricity potentials for hydrogen export only. Nonetheless, \cite{Tries2023b} shows the benefits of hydrogen islanding, offering cost savings for inverters and benefits for power quality. Domestic consumers may profit from these economic opportunities for hydrogen exporters indirectly through corresponding policies.

Hydrogen export ambitions (fostered by e.g. the European Union) do not cause disadvantages for the domestic population \textit{per se}, a proactive transition taking into account the interplay of domestic electricity prices unleashes economic opportunities for both exporters and domestic population.

% Ideas:
% \begin{itemize}
%     \item Danger: Avoid Neo-colonialism (Export industry fostered by EU etc. has a significant impact)
%     \item Increasing electricity prices for domestic consumers? Driven by climate ambition, not by export. Rather synergy at high export and high climate ambition, decreasing electricity prices for domestic consumers
%     \item Land use: The best potentials should not be reserved for export
%     \item Land use competition and its consequences
%     \item Potential risks of neo-colonialism and their significance
% \end{itemize}


% The demand of hydrogen and derivatives to decarbonize the e.g. European energy system imposes the danger of neglecting domestic needs and prerequisites in Morocco. Various studies are lacking [TBD: Add studies] sufficient attention to Moroccos energy system and oversimplify the multi-dimensional task of scaling up hydrogen exports. 
% By integrating and fully considering Moroccos energy system, this study identifies resulting p


% \subsection{Integrated policies for synergy and fairness}
% \label{subsec:fairness}

% \begin{itemize}
%     \item Tax hydrogen electricity price? They get the cheaper electricity prices at least. Ensure equitable economic distribution
%     \item Land use policies?
%     \item Aligning hydrogen export goals with renewable energy targets/climate targets 
%     \item Mitigating risks of neo-colonialism through responsible practices
% \end{itemize}



\subsection*{Limitations}
\label{subsec:limitations}

The sector-coupled energy model as well as the results underly certain limitations. In this study, we leave the option open that the hydrogen for export is further synthesized to hydrogen derivatives as ammonia, Fischer-Tropsch products or sponge iron etc. We do not model the system operation of a further synthesis of export products explicitly, unlike we do for the domestic demands for hydrogen derivatives. Further synthesis of export products and linked transportation modes are investigated in \cite{Hampp2021}, \cite{Galimova2023} and \cite{Verpoort2023} but not in the scope of this study. They will likely have effects on the system operation, since the synthesis in combination with cheaper storage (e.g. oil storage is cheaper than hydrogen storage \cite{DEA2019TechnologyData}) might provide additional flexibility to a certain extent but is limited by the operation flexibility of Fischer-Tropsch or Haber-Bosch processes.

Additionally, the transition of certain sectors is not subject to the optimisation but exogenously defined. Such nodal shifts (e.g. in transport: share of electric vehicles) provide system benefits (e.g. Vehicle-to-Grid) for which an integrated optimisation favours e.g. higher shares of electric vehicles.
%[TBD: How does e.g. \cite{Neumann2022} justify it?].
Recent approaches as \cite{Zeyen2023} improve such caveats by incorporating endogenous learning, but research gaps on the endogenous demand of the transport sector remain.

National climate ambitions and especially the scale-up of hydrogen export require substantial amounts of raw materials (e.g. copper, cement) linked with resource limitations as well as upstream GHG emissions. As pointed out in \cite{Wang2023} in a global analysis, most raw material limitations required for electricity generation do not exceed geological reserves. However, material demands of e.g. electrolysers are not within the scope of \cite{Wang2023}.

The cost of water through desalination and transport has a 
has a single cost for the whole country. However, the transportation costs highly depend on the terrain as pointed out by \cite{Caldera2016}, hence the water costs are higher in more remote areas which is not taken into account in this study. In contrast, the effect of a higher spatial resolution of water costs is assumed to be minor, since the water desalination and transport costs do not contribute substantially to the cost of hydrogen as emphasized in \cite{Hampp2023}. %Also: Direct integration of REs to desalination is not considered

% TBD: comparison with other studies
% \subsection{Comparison with other studies}
% E.g. compare to \cite{Hampp2021} and their cost of electricity. High price of hydrogen due to 15\% WACC!

% TBD: further research
% \subsection{Further research}
% \begin{itemize}
%     \item Transport options
%     \item Include other countries (But focus is Morocco)
%     \item Include other costs
% \end{itemize}