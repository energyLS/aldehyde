\begin{itemize}
    \item Identification of conflicts and synergies between hydrogen exports and energy transition
    \item Implications for renewable energy deployment
    \item Land use competition and its consequences
    \item Potential risks of neo-colonialism and their significance
\end{itemize}


\subsection{Synergies}
Ideas:
\begin{itemize}
    \item High export and high climate ambition leads to synergies (lower electricity prices)
    \item Export profits from high climate ambition: hydrogen export gets cheaper at >70\% CO2 reduction (s. Fig. \ref{fig:export_local_hy_price})
    \item 
\end{itemize}

\subsection{Synergy maximizing time path}
Ideas:
\begin{itemize}
    \item Can we draw a "time path"? Or is this not valid since it is all 2030/2038 scenarios? Should we include a time dimension in the climate ambition dimension?
\end{itemize}

\subsection{Conflicts}
Ideas:
\begin{itemize}
    \item Avoid high electricity prices at 60\% reduction and low export, rather go for higher export there
\end{itemize}

\subsection{Neo-colonial Implications}
Ideas:
\begin{itemize}
    \item Danger: Avoid Neo-colonialism (Export industry fostered by EU etc. has a significant impact)
    \item Increasing electricity prices for local consumers? Driven by climate ambition, not by export. Rather synergy at high export and high climate ambition, decreasing electricity prices for local consumers
    \item Land use: The best potentials should not be reserved for export
\end{itemize}



\subsection{Integrated policies for synergy and fairness}
\begin{itemize}
    \item Tax hydrogen electricity price? They get the cheaper electricity prices at least. Ensure equitable economic distribution
    \item Land use policies?
    \item Aligning hydrogen export goals with renewable energy targets/climate targets 
    \item Mitigating risks of neo-colonialism through responsible practices
\end{itemize}





\subsection{Limitations}
\label{subsec:limitations}

The sector-coupled energy model as well as the results underly certain limitations. In this study, the only export energy carrier investigated is hydrogen, although the export of hydrogen derivatives as ammonia, Fischer-Tropsch products, HVCs or even products as pig iron etc. are under discussion [TBD: Add Verpoort study]. Further synthesis (Haber-Bosch, Fischer-Tropsch) and linked transportation modes as investigated in \cite{Hampp2021} or \cite{Galimova2023} but not the scope of this study. Further synthesis for export products will likely have effects on the system operation, since the synthesis in combination with cheaper storages (e.g. oil storages are cheaper than H2 storages [TBD: Source]) might provide additional flexibility to a certain extend (limited by the operation of Fischer-Tropsch or Haber-Bosch).

Additionally, the transition of certain sectors is not subject to the optimisation but exogenously defined. Such nodal shifts (e.g. in transport: share of electric vehicles) might have system benefits (e.g. V2G) for which an integrated optimisation might enforce certain pathways [TBD: How does e.g. \cite{Neumann2022} justify it?]. Recent approaches incorporating endogenous learning provide further insights [TBD: which ones?] \cite{Zeyen2023} [TBD: Is this source correct?].

National climate ambitions and especially the scale-up of hydrogen export require significant amounts of raw materials (copper, cement, ...) linked with resource limitations as well as upstream GHG emissions. As pointed out in \cite{Wang2023} in a global analysis, most raw material limitations required for electricity generation do not exceed geological reserves. However, material demands of e.g. electrolysers are not within the scope of \cite{Wang2023}.




\subsection{Comparison with other studies}
E.g. compare to \cite{Hampp2021} and their cost of electricity

\subsection{Further research}
\begin{itemize}
    \item Transport options
    \item Include other countries (But focus is Morocco)
    \item Include other costs
\end{itemize}