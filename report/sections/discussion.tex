Chapter \ref{sec:discussion} discusses synergies and conflicts (s. Sec. \ref{subsec:synergies-conflicts}) of the dual goal of increasing hydrogen export and emission reductions. Section \ref{subsec:timepath} puts the results into perspective of a synergy maximizing time frame. Neo-colonial implications deriving from hydrogen export are discussed in Section \ref{subsec:neocolonial}.
% and possible policies increasing fairness are suggested in Section \ref{subsec:fairness}. 
Limitations of this study are discussed in Section \ref{subsec:limitations}.

% General ideas
% \begin{itemize}
%     \item Identification of conflicts and synergies between hydrogen exports and energy transition
%     \item Implications for renewable energy deployment
%     \item Also: Refer to policy from MAR?
% \end{itemize}


\subsection{Synergies and conflicts}
\label{subsec:synergies-conflicts}
Results from Chapter \ref{sec:results} indicate various synergies and conflicts of hydrogen export and national energy transition. The price of electricity highly depends on the climate ambition and hydrogen export volume, e.g. increasing the costs of electricity for local demands at medium climate ambition (60\%) and high export volumes (150 - 200 TWh). On the other hand, the green hydrogen constraint requiring the installation of renewable energies according to the hydrogen exports hedges local electricity consumers against higher electricity prices at lower climate ambitions. If such a constraint is not imposed on the hydrogen exports, the export industry could benefit from cheaper electricity prices for electrolysis.

The electricity price increase flattens out at high climate ambitions, meaning local electricity consumers are able to avoid high economic burdens imposed by a hydrogen infrastructure in these scenarios.

The hydrogen prices increase with hydrogen exports across all climate ambition scenarios. Nonetheless, the increase is less significant when the local energy system relies already on substantial amounts of hydrogen and Fischer-Tropsch products. The price of hydrogen for export is lowest when the local energy transition is still in the early stages and the export volumes are low -- electrolysers can run at high capacity factors paying of high upfront capital costs. 


\subsection{Synergy maximizing time path}
\label{subsec:timepath}
All scenarios investigated in this study are based on the year 2030 as reasoned in Chapter \ref{sec:methods}. The transition towards climate neutrality of Moroccos energy system or the ramping up of exports up to 200 TWh/a are a matter of decades and undergoes a certain pathway. Based on the synergies and conflicts identified in Section \ref{subsec:synergies-conflicts}, steering the transition by minimizing conflicts and maximizing/amplifying synergies reduce both the price of electricity and price of hydrogen. To avoide high electricity prices, it is advisable to avoid high export volumes at medium climate ambitions. It is beneficial, to reach a certain degree of local decarbonization before exporting hydrogen volumes of 100 TWh and above, to use the synergies between the flexibilities provided by a net-zero energy system (electrolysis, hydrogen storage, Fischer-Tropsch) for a large-scale integration of variable renewable energies required for substantial export volumes.


\subsection{Neo-colonial Implications}
\label{subsec:neocolonial}
Neo-colonial implications mentioned in Chapter \ref{sec:intro} are reflected in various dimensions. First, the price of electricity for local businesses and households need to be considered to enable a fair discussion and balance of burdens and opportunities. This study shows that the price of electricity highly depends on hydrogen export volumes and uncareful planning inherits the danger of increasing electricity prices for local demands. Converesly, the export of hydrogen provides the opportunity of decreasing electricity prices through the green hydrogen constraint, providing opportunities for local businesses and households.
Second, the significant deployment of solar PV and onshore wind comes along with a substantial land use. Several studies \cite{Terrapon-Pfaff2019, Hanger2016} highlight the requirement of local acceptance of REs in Morocco. From a German/European perspective, hydrogen imports are not necessarily the most economic option but also provide a way out of having to deal with national acceptance of REs in Germany/Europe. Shifting such problems of acceptance to Morocco by deploying REs and scaling up their hydrogen exports imposes moral scruples and should be delt with carefully. %TBD: Desertec: why did it fail.
Third, the best solar PV and onshore wind potentials should not be reserved for hydrogen exports by setting up islanded hydrogen production in these locations. This leads to higher prices of electricity of the local electricity system which then can not access these cost-cutting potentials.

Hydrogen export ambitions (fostered by e.g. the EU) do not cause disadvantages for the local population \textit{per se}, a proactive transition taking into account the interplay of local electricity prices unleashes economic opportunities for both exporters and local population.

% Ideas:
% \begin{itemize}
%     \item Danger: Avoid Neo-colonialism (Export industry fostered by EU etc. has a significant impact)
%     \item Increasing electricity prices for local consumers? Driven by climate ambition, not by export. Rather synergy at high export and high climate ambition, decreasing electricity prices for local consumers
%     \item Land use: The best potentials should not be reserved for export
%     \item Land use competition and its consequences
%     \item Potential risks of neo-colonialism and their significance
% \end{itemize}


% The demand of hydrogen and derivatives to decarbonize the e.g. European energy system imposes the danger of neglecting local needs and prerequisites in Morocco. Various studies are lacking [TBD: Add studies] sufficient attention to Moroccos energy system and oversimplify the multi-dimensional task of scaling up hydrogen exports. 
% By integrating and fully considering Moroccos energy system, this study identifies resulting p


% \subsection{Integrated policies for synergy and fairness}
% \label{subsec:fairness}

% \begin{itemize}
%     \item Tax hydrogen electricity price? They get the cheaper electricity prices at least. Ensure equitable economic distribution
%     \item Land use policies?
%     \item Aligning hydrogen export goals with renewable energy targets/climate targets 
%     \item Mitigating risks of neo-colonialism through responsible practices
% \end{itemize}





\subsection{Limitations}
\label{subsec:limitations}

The sector-coupled energy model as well as the results underly certain limitations. In this study, the only export energy carrier investigated is hydrogen, although the export of hydrogen derivatives as ammonia, Fischer-Tropsch products, HVCs or even products as sponge iron etc. are under discussion \cite{Verpoort2023}. Further synthesis (Haber-Bosch, Fischer-Tropsch) and linked transportation modes are investigated in \cite{Hampp2021} or \cite{Galimova2023} but not the scope of this study. Further synthesis for export products will likely have effects on the system operation, since the synthesis in combination with cheaper storages (e.g. oil storages are cheaper than H2 storages \cite{DEA2019TechnologyData}) might provide additional flexibility to a certain extend (limited by the operation of Fischer-Tropsch or Haber-Bosch).

Additionally, the transition of certain sectors is not subject to the optimisation but exogenously defined. Such nodal shifts (e.g. in transport: share of electric vehicles) provide system benefits (e.g. Vehicle-to-Grid) for which an integrated optimisation favours e.g. higher shares of electric vehicles.
%[TBD: How does e.g. \cite{Neumann2022} justify it?].
Recent approaches as \cite{Zeyen2023} improve such caveats by incorporating endogenous learning, but research gaps on the endogenous demand of the transport sector remain.

National climate ambitions and especially the scale-up of hydrogen export require significant amounts of raw materials (copper, cement, ...) linked with resource limitations as well as upstream GHG emissions. As pointed out in \cite{Wang2023} in a global analysis, most raw material limitations required for electricity generation do not exceed geological reserves. However, material demands of e.g. electrolysers are not within the scope of \cite{Wang2023}.

The cost of water through desalination and transport has a spatial resolution on country level. However, the transportation costs highly depend on the terrain as pointed out by \cite{Caldera2016}, hence the water costs are higher in more remote areas which is not taken into account in this study. In contrast, the effect of a higher spatial resolution of water costs is assumed to be minor, since the water desalination and transport costs do not contribute significantly to the cost of hydrogen as emphasized in \cite{Hampp2023}. %Also: Direct integration of REs to desalination is not considered

% TBD: comparison with other studies
% \subsection{Comparison with other studies}
% E.g. compare to \cite{Hampp2021} and their cost of electricity. High price of hydrogen due to 15\% WACC!

% TBD: further research
% \subsection{Further research}
% \begin{itemize}
%     \item Transport options
%     \item Include other countries (But focus is Morocco)
%     \item Include other costs
% \end{itemize}