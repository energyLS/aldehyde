
% Purpose: The Introduction section clarifies the motivation for the work presented and prepares readers for the structure of the paper. 

% Structure:

% First, provide some context to orient those readers who are less familiar with your topic and to establish the importance of your work. "Funnel"
% - Anchor: Time ("recently", "since 2011",) and space ("Morocco", "Europe")

% Second, state the need for your work, as an opposition between what the scientific community currently has and what it wants.
% - Emphasize contrast with "but", "however", ..
% Desired vs actual situation

% Third, indicate what you have done in an effort to address the need (this is the task).

% Finally, preview the remainder of the paper to mentally prepare readers for its structure, in the object of the document.

% First: Context
The global energy environment is experiencing fundamental upheaval, driven by the need to reduce greenhouse gas emissions and transition to a low-carbon future. In this context, hydrogen has emerged as a viable clean energy carrier capable of addressing the issues of decarbonizing numerous sectors such as industry, transport, heating and power generation. A growing number of countries are investigating green hydrogen production and use as a crucial component of their strategy to cut greenhouse gas emissions and meet ambitious climate targets.
Simultaneously, several countries are positioning themselves as potential exporters of green hydrogen, discovering an opportunity to leverage their renewable energy resources and technological advances. % [TBD: add examples]
These countries are expected to play a significant role in the global energy market by supplying clean hydrogen and therefore contributing to global decarbonization efforts. However, pursuing both (on-grid) hydrogen exports and national energy transition raises questions on welfare redistribution, prices and co-benefits that require in-depth analysis, additionally shining a light on the role of hydrogen regulation.

% Second: Need
Morocco serves as a blueprint for investigating these dependencies. Morocco, which has vast solar and wind resources \cite{Peters2023, Touili2018, Sterl2022}, has proposed ambitious renewable energy adoption objectives, displaying a commitment to lowering its own greenhouse gas emissions \cite{CAT2021}. At the same time, the country is strategically positioned to deploy its renewable energy potentials to export green hydrogen, opening the door to substantial economic opportunities (and local value chains \cite{Ersoy2022}). Furthermore, the country is a net importer of energy \cite{IEA2022}, and green hydrogen could help reduce its dependency on energy imports. The proximity to Europe, which is expected to be a major hydrogen importer, makes Morocco an attractive potential exporter. %https://www.giz.de/de/downloads/giz2012-en-solar-plan-morocco.pdf


These conditions are the motivation of several studies \cite{vanWijk2021, AbouSeada2022, vanderZwaan2021, Schellekens2010, Cavana2021, Touili2022, Timmerberg2019a, Sens2022} examining the potential of hydrogen in Africa and synergies with European demand. A more detailed study of Morocco has been undertaken by \cite{Boulakhbar2020} examining challenges in integrating Renewable Energies, \cite{Khouya2020} determining the Levelized Cost of Hydrogen based on CSP and wind farms, and \cite{Touili2018} investigating the potential of hydrogen from solar energy. \cite{Hampp2021} investigates various PtX products and their transportation to Europe, and \cite{Eichhammer2019} highlights diverse opportunities and challenges related to exporting hydrogen and PtX from Morocco. While several studies\cite{Hampp2021, AbouSeada2022, vanWijk2021} have examined the potential of hydrogen as a low-carbon energy carrier and others have explored various countries' climate targets and aspirations \cite{Boulakhbar2020}, a significant research gap remains regarding the integrated investigation of both perspectives. 


Apart from hydrogen exports and domestic mitigation, hydrogen regulation plays a decisive role in prices for domestic consumers and hydrogen exporters. Temporal matching in selected European countries is investigated in \cite{Zeyen2022}, \cite{Ruhnau2023a} points out benefits of relaxing simultaneity requirements of RE supply and hydrogen generation. 
The effect of regulatory options on social welfare and carbon emissions is assessed in \cite{Brauer2022}. 
While \cite{Zeyen2022} contextualizes the role of hydrogen regulation with decarbonization scenarios of Germany and Netherlands, none of the mentioned studies fully integrates domestic mitigation and hydrogen regulation scenarios.

Numerous studies focus exclusively on either hydrogen exports, domestic climate mitigation or hydrogen regulation, without discussing the complex relationship between these three dimensions. These studies miss interactions between on-grid hydrogen electrolysis and the domestic electricity system, fail to uncover potential synergies and conflicts for both hydrogen exporters as well as domestic electricity consumers with respect to different hydrogen regulation regimes.

% Third: Scenarios
% \subsection{Scenarios}

Our analysis is based on a scenario vector of domestic climate mitigation, hydrogen exports and hydrogen regulation, sweeping along these three dimensions:
\begin{enumerate}
    \item \textbf{Mitigation}: The (energy and non-energy) emission reduction varies between 0~--~100\% based on Morocco's current emissions of 72 Mt\coe (s. Sec. \ref{subsec:climate_targets}) and
    \item \textbf{Export}: The hydrogen export volume varies from 0~--~200 TWh, going beyond Moroccos hydrogen export ambitions of 114.7~TWh (s. Sec. \ref{subsec:strategy}).
    \item \textbf{Hydrogen regulation}: The temporal matching (of additional renewable electricity and the electrolyser electricity demand) varies between: no regulation, annual, monthly and hourly matching (s. Sec. \ref{subsec:green_hydrogen_constraint}). 
\end{enumerate}
This three-dimensional scenario space results in 484 model runs to grasp the full extend of the interaction between various parameters.
A 3-hourly resolution is chosen to capture energy system dynamics (e.g. RE generation profiles, energy storage operation) with its diurnal and seasonal variations in energy supply and demand. A similar study \cite{Neumann2022} finds minor underestimation of short-term battery storage and onshore wind and minor overestimation of solar photovoltaics and hydrogen storage compared to an hourly resolution, overall justifying the reduction of the model size. 
A spatial resolution of 14 nodes represents the geographical heterogeneity of RE resources, demand centers and energy networks. Furthermore, the spatial resolution provides insights on land use conflicts and competing RE resources as well as a spatial differentiation of export ports. Since the 14 nodes align with the GADM (Global Administrative Areas) level 1 regions, policy advisement can be tailored to specific regions considering local needs and constraints.


% Forth: Task
To investigate interactions between hydrogen exports, domestic mitigation and hydrogen regulation, this research paper presents a fully sector-coupled capacity expansion and dispatch model of Morocco, which includes both gas pipelines and electricity networks. By examining various hydrogen export volumes, climate targets and temporal matchings, we evaluate the potential impact of hydrogen exports on domestic electricity consumers and, vice versa, interactions of domestic mitigation on hydrogen exporters.
Shortcomings of studies on highly renewable energy systems in Africa as low temporal and spatial resolutions as well as the lack of sector-coupled energy models as pointed out in \cite{Oyewo2023} are tackled in this study. The analysis of Morocco's energy system contributes to the expanding research that aims to inform policymaking and decision-making regarding sustainable energy transitions. This study provides valuable insights into the pathways unlocking synergies and reducing conflicts between hydrogen exports and national energy transition, enabling Morocco and other potential exporting regions to follow a harmonious and sustainable trajectory while facing similar energy system planning challenges.

%  Fith: Overview of the study
Chapter \ref{sec:policyandtargets} presents Morocco's hydrogen strategy and climate targets and outlines potential conflicts and synergies between these two goals. Chapter \ref{sec:methods} presents the applied sector-coupled energy model and scenario dimensions required to expose the export-mitigation-regulation nexus. Introducing the results along these three dimensions step-by-step, Chapter \ref{sec:results} evaluates synergies and conflicts based on priced and total expenses for domestic electricity consumers and hydrogen exporters. These results are contextualized along policy recommendations in Chapter \ref{sec:discussion} and concluded in Chapter \ref{sec:conclusion}.


% We explore the intricacies of technology lock-in, local energy-related greenhouse gas emissions, rising local energy prices, land use implications and neo-colonialism \cite{HabeckEnergieimperialismus, Terrapon-Pfaff2019, Hanger2016}, providing a comprehensive overview of the challenges and opportunities that emerge while pursuing these dual objectives.

%\subsection{Miscellaneous}
%
%Current cooperations/business:
%\begin{itemize}
%    \item Indias role: https://www.moroccoworldnews.com/2023/02/353972/indias-renew-energy-global-eyes-moroccos-green-hydrogen-market
%    \item TBD: Investigation of EU/Germany involvement in MAR?
%    \item Cooperation between port authorities in Tanger Med and Hamburg
%\end{itemize}

%\subsubsection{Moroccan RE Strategy}
%The Moroccan RE strategy consists of a "Moroccan Solar Plan" initiated in 2009 \cite[p. 2]{Boulakhbar2020}
%and a wind energy program.
%The Moroccan wind energy program promotes 2000 MW by 2020 and 2600 MW by 2030 \cite[p. 4]{Boulakhbar2020}
%The solar plan ... (TBD: Describe the solar plan)
%(See \cite[p. 13]{Ersoy2022} for a good overview of the Moroccan RE strategy).

%Literature review (specific):
%\begin{itemize}
%    \item Local energy demand considered in: MenaFuels and \cite{Hampp2021}. According to MenaFuels the local demand is usually just some percentages of the potential
%    \item Hydrogen from (North-)Africa (to Europe): \cite{vanWijk2021} \cite{AbouSeada2022} \cite{vanderZwaan2021} and \cite{Schellekens2010} and \cite{Cavana2021}, specific countries from Africa: \cite{Hampp2021}, PV technologies for Hydrogen production in Africa \cite{Touili2022}, Pipeline blend \cite{Timmerberg2019a}, hydrogen cost \cite{Sens2022}
%    \item Morocco: Renewable Integration \cite{Boulakhbar2020}, LCOH hydrogen CSP and wind farms \cite{Khouya2020}, Discussion paper PtX \cite{Eichhammer2019}, hydrogen potential from solar energy \cite{Touili2018}
%    \item Literature review on energy systems in Africa \cite{Oyewo2023} and \cite{Oyewo2022}
%    \item On geopolitics: \cite{Pflugmann2020} and \cite{Graaf2020}
%    \item \cite{vanderZwaan2021} Also looks into North Africa, but on country resolution. they don't look at biomass either
%    \item Overview of the global hydrogen market and potential exporting countries
%    \item Key challenges and opportunities of simultaneous hydrogen export and energy transition
%    \item Previous research on technology lock-in, local emissions, land use, and neo-colonialism
%\end{itemize}


%\subsection{Ideas}
%Narrative: 
%\begin{itemize}
%    \item Security of supply/economic growth/cheap energy for North Africa and Morocco
%    \item Fair cooperation and partnerships by: including local energy demands, investigation of conflicts/synergies
%    \item Best locations with high FLH should not be reserved for export (as done in Global PtX Atlas)
%    \item EU can rely on cost-effective green hydrogen
%    \item Discuss why it is important to understand synergies and conflicts
%    \item Cannibalization: https://www.climatechangenews.com/2023/04/11/green-hydrogen-rush-risks-energy-cannibalisation-in-africa-analysts-say/
%\end{itemize}


% Main research question:
% \begin{itemize}
%     \item Is it beneficial (total system cost, lcoh, lcoe, emissions) to (simultaneously) push the national energy transition and hydrogen export (synergies and conflicts)?   
% \end{itemize}


% Research questions:
% \begin{itemize}
%     \item Are there wealth, cost and emission benefits of an integrated optimisation compared to separated analyses?
%     \item Are there competing RE potential resources and where are they located? Should these resources used for local decarbonization or for export purposes?
%     \item What is the role of green/grey hydrogen?   
%     \item Can an integrated analysis avoid carbon leakage?
%     \item Does ongrid integration avoid installing excess RES and El capacity? (That's a finding of e.g. \cite{Ruhnau2022})
%     \item What are the conflicts and benefits of national transition and hydrogen export?
% \end{itemize}