From big to small
\begin{itemize}
    \item Why is this research important? Climate change, COP27, hydrogen, local decarbonization, combined efforts (assets: especially with scarcity of electrolyzer capacity, we should use them better)
    \item What is the aim?
    \item 
    \item What are the results?
\end{itemize}
Narrative: 
Security of supply/economic growth/cheap energy is highly relevant for North Africa and Morocco. 
Including local energy demands and systems unlocks potentials and enables fair cooperation and partnerships.
Best locations with high FLH should not be reserved for export (as done in Global PtX Atlas). 
EU can rely on cost-effective green hydrogen

Habeck: "gr{\"u}ner Energie-Imperialismus" when visiting Namibia \cite{HabeckEnergieimperialismus}

Research questions:
\begin{itemize}
    \item Are there wealth, cost and emission benefits of an integrated optimisation compared to separated analyses?
    \item Are there competing RE potential resources and where are they located? Should these resources used for local decarbonization or for export purposes?
    \item What is the role of green/grey hydrogen?   
    \item Does ongrid integration avoid installing excess RES and El capacity? (That's a finding of e.g. \cite{Ruhnau2022})
\end{itemize}

Literature research:
\begin{itemize}
    \item Local energy demand considered in: MenaFuels and \cite{Hampp2021}. According to MenaFuels the local demand usually just some percentages.
\end{itemize}