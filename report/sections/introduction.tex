
\subsection{Ideas}
From big to small
\begin{itemize}
    \item Why is this research important? Climate change, COP27, hydrogen, local decarbonization, combined efforts (assets: especially with scarcity of electrolyzer capacity, we should use them better)
    \item What is the aim?
    \item 
    \item What are the results?
\end{itemize}
Narrative: 
Security of supply/economic growth/cheap energy is highly relevant for North Africa and Morocco. 
Including local energy demands and systems unlocks potentials and enables fair cooperation and partnerships.
Best locations with high FLH should not be reserved for export (as done in Global PtX Atlas). 
EU can rely on cost-effective green hydrogen

Morocco is already building projects and is ramping up it's hydrogen strategy, this paper gives guidance on time frames
Storyline for Morocco: Currently importing 90 \% could become a net-exporter. Great chance

Local production of RE plants possible, as stated in \cite{Ersoy2022}

Habeck: "gr{\"u}ner Energie-Imperialismus" when visiting Namibia \cite{HabeckEnergieimperialismus}.

Indias role: https://www.moroccoworldnews.com/2023/02/353972/indias-renew-energy-global-eyes-moroccos-green-hydrogen-market


Research questions:
\begin{itemize}
    \item Are there wealth, cost and emission benefits of an integrated optimisation compared to separated analyses?
    \item Are there competing RE potential resources and where are they located? Should these resources used for local decarbonization or for export purposes?
    \item What is the role of green/grey hydrogen?   
    \item Does ongrid integration avoid installing excess RES and El capacity? (That's a finding of e.g. \cite{Ruhnau2022})
\end{itemize}

Literature research:
\begin{itemize}
    \item Local energy demand considered in: MenaFuels and \cite{Hampp2021}. According to MenaFuels the local demand usually just some percentages.
\end{itemize}


\subsection{Policy and climate targets}
Morocco has implemented various strategies and policies to reach its climate targets and hydrogen exports. 

\subsubsection{Moroccan RE Strategy}
See \cite[p. 13]{Ersoy2022} for a good overview of the Moroccan RE strategy.
"Moroccan Solar Plan" initiated in 2009 according to \cite[p. 2]{Boulakhbar2020}
Moroccan wind energy program: 2000 MW in 2020 and 2600 in 2030 \cite[p. 4]{Boulakhbar2020}

\subsubsection{Moroccan Hydrogen Strategy}
See \cite[p. 14]{Ersoy2022} for a good overview of the Moroccan Hydrogen strategy. Figure \ref{fig:mar_hydrogen_strategy} shows the hydrogen strategy of Morocco.

Details:
\begin{itemize}
    \item capacities of RE: 2030: 5.2 GW, 2040: 23 GW, 2050: 55 GW (only for H2? Include in Figure \ref{fig:mar_hydrogen_strategy}?)
    \item Cooperation between port authorities in Tanger Med and Hamburg
\end{itemize}

\begin{figure}
    \centering
    \includegraphics[width=\linewidth]{policy/mar_hydrogen_strategy_grouped.pdf}
    \caption{Morocco Hydrogen Strategy}
    \label{fig:mar_hydrogen_strategy}
\end{figure}

\subsubsection{Climate targets}
Describe the emission targets of Morocco. In general: Low per capita emissions
Depicted in Figure \ref{fig:morocco_em}.
A GHGe reduction of 32 \% by 2030 is mentioned in \cite[5]{Boulakhbar2020}
NDCs: https://www.iges.or.jp/en/pub/iges-indc-ndc-database/en 
Insights from \cite{CAT2021}:
\begin{itemize}
    \item No net-zero target.
    \item 2030: 115 MtCO2e (NDC, excluding LULUCF, Unconditional target)
    \item 2030: 75 MtCO2e (NDC, excluding LULUCF, Conditional target)
    \item 2030: 52\% Renewable energy share in installed electric capacity
\end{itemize}

\begin{figure}[h!]
    \centering
    \includegraphics[width=\linewidth]{morocco_em.pdf}
    \caption{Morocco historical emissions and targets.}
    \label{fig:morocco_em}
\end{figure}