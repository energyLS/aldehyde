% First: Context
The global energy environment is experiencing fundamental upheaval, driven by the need to reduce greenhouse gas emissions and transition to a low-carbon future. In this context, hydrogen has emerged as a viable clean energy carrier capable of addressing the issues of decarbonizing numerous sectors such as industry, transport, heating and power generation. A growing number of countries are investigating green hydrogen production and use as a crucial component of their strategy to cut greenhouse gas emissions and meet ambitious climate targets.
Simultaneously, several countries are positioning themselves as potential exporters of hydrogen and Power-to-X products, discovering an opportunity to leverage their renewable energy resources and technological advances. % [TBD: add examples]
These countries are expected to play a substantial role in the global energy market by supplying clean hydrogen and therefore contributing to global decarbonization efforts. However, pursuing both (on-grid) hydrogen exports and national energy transition raises questions on welfare redistribution, prices and co-benefits that require in-depth analysis, additionally shining a light on the role of temporal hydrogen regulation.

% Second: Need
Morocco serves as a blueprint for investigating these dependencies. Morocco, which has vast solar and wind resources \cite{Peters2023, Touili2018, Sterl2022}, has proposed ambitious renewable energy adoption objectives, displaying a commitment to lowering its own greenhouse gas emissions \cite{CAT2021}. At the same time, the country is strategically positioned to deploy its renewable energy potentials to export green hydrogen, opening the door to substantial economic opportunities and local value chains \cite{Ersoy2022}. Furthermore, the country is a net importer of energy \cite{IEA2022}, and green hydrogen could help reduce its dependency on energy imports. The proximity to Europe, which is expected to be a major hydrogen importer, makes Morocco an attractive potential exporter. 


These conditions are the motivation of several studies \cite{vanWijk2021, AbouSeada2022, vanderZwaan2021, Schellekens2010, Cavana2021, Touili2022, Timmerberg2019a, Sens2022, Franzmann2023} examining the potential of hydrogen in Africa and synergies with European demand. A more detailed study of Morocco has been undertaken by Boulakhbar et al.\cite{Boulakhbar2020} examining challenges in integrating Renewable Energies, Khouya et al.\cite{Khouya2020} determining the Levelized Cost of Hydrogen based on concentrated solar power and wind farms, and Touili et al.\cite{Touili2018} investigating the potential of hydrogen from solar energy. Hampp et al.\cite{Hampp2023} investigates various PtX products and their transportation to Europe, and Eichhammer et al. \cite{Eichhammer2019} highlights diverse opportunities and challenges related to exporting hydrogen and Power-to-X products from Morocco. Ishmam et al. \cite{Ishmam2024} assesses the impact of hydrogen exports on social factors and domestic jobs, going beyond a simplified energy potential analysis.
While several studies \cite{Hampp2023, AbouSeada2022, vanWijk2021} have examined the potential of hydrogen as a low-carbon energy carrier and others have explored various countries' climate targets and aspirations \cite{Boulakhbar2020}, a substantial research gap remains regarding the integrated investigation of both perspectives. 


Apart from hydrogen exports and domestic climate change mitigation, temporal hydrogen regulation plays a decisive role in prices for domestic consumers and hydrogen exporters. 
Temporal hydrogen regulation defines the rules for green hydrogen production when there is no direct connection between the electrolyser and green electricity production.
Temporal matching in selected European countries is investigated in Zeyen et al.\cite{Zeyen2024}, whereas Ruhnau et al.\cite{Ruhnau2023a} points out benefits of relaxing simultaneity requirements of renewable electricity (RE) supply and hydrogen generation. 
The effect of regulatory options on social welfare and carbon emissions is assessed in Brauer et al.\cite{Brauer2022}. The interplay of additionality criteria and time matching requirements is investigated in Giovanniello et al.\cite{Giovanniello2024}, highlighting how additionality drices the emissions impact of temporal hydrogen regulation.
While Zeyen et al.\cite{Zeyen2024} contextualizes the role of temporal hydrogen regulation with decarbonization scenarios of Germany and Netherlands, none of the mentioned studies fully integrates domestic climate change mitigation and temporal hydrogen regulation scenarios. Furthermore, none of these studies looked into exports in depth.

Numerous studies focus exclusively on either hydrogen exports, domestic climate mitigation or temporal hydrogen regulation, without discussing the complex relationship between these three dimensions. These studies miss interactions between on-grid hydrogen electrolysis and the domestic electricity system, fail to uncover potential synergies and conflicts for both hydrogen exporters as well as domestic electricity consumers with respect to different temporal hydrogen regulation regimes.


% Third: Scenarios
% \subsection{Scenarios}
The novelty of our study is the development of a sector-coupled energy model for the target region, the inclusion of modelling the temporal hydrogen regulation, and the evaluation of the synergies among hydrogen exports, domestic energy transition and hydrogen regulation.
Additionally, we present a broad scenario vector, sweeping along these three dimensions:
\begin{enumerate}
    \item \textbf{Domestic climate change mitigation}: The domestic climate change mitigation varies between 0--100\% based on Morocco's current emissions of 72 Mt\coe,
    \item \textbf{Hydrogen export}: The hydrogen export volume varies from 1--120 TWh, in accordance with Morocco's hydrogen export ambitions of 114.7~TWh/a and
    \item \textbf{Temporal hydrogen regulation}: The temporal matching (of additional renewable electricity and the electrolyser electricity demand) varies between: no regulation, annual, monthly and hourly matching.
\end{enumerate}
This three-dimensional scenario space results in 264 model runs (excluding sensitivity analysis) to grasp the full extent of the interaction between various parameters.
A 3-hourly resolution is chosen to capture energy system dynamics (e.g. RE generation profiles, energy storage operation) with its diurnal and seasonal variations in energy supply and demand. A similar study \cite{Neumann2022} finds minor underestimation of short-term battery storage and onshore wind and minor overestimation of solar photovoltaics (PV) and hydrogen storage compared to an hourly resolution, overall justifying the reduction of the model size. 
A spatial resolution of 14 nodes represents the geographical heterogeneity of RE resources, demand centers and energy networks. Furthermore, the spatial resolution provides insights into land use conflicts and competing RE resources as well as a spatial differentiation of export ports. Since the 14 nodes align with the Global Administrative Areas level 1 regions, policy advisement can be tailored to specific regions considering domestic needs and constraints.


% Forth: Task
To investigate interactions between hydrogen exports, domestic climate change mitigation and temporal hydrogen regulation, this research paper presents a fully sector-coupled capacity expansion and dispatch model of Morocco, which includes both gas pipelines and electricity networks. By examining various hydrogen export volumes, climate targets and temporal matchings, we evaluate the potential impact of hydrogen exports on domestic electricity consumers and, vice versa, interactions of domestic climate change mitigation on hydrogen exporters.
Shortcomings of studies on highly renewable energy systems in Africa such as low temporal and spatial resolutions as well as the lack of sector-coupled energy models as pointed out in Oyewo et al.\cite{Oyewo2023} are tackled in this study. The analysis of Morocco's energy system contributes to the expanding research that aims to inform policymaking and decision-making regarding sustainable energy transitions. This study provides valuable insights into the pathways unlocking synergies and reducing conflicts between hydrogen exports and national energy transition, enabling Morocco and other potential exporting regions to follow a harmonious and sustainable trajectory while facing similar energy system planning challenges.

%  Fith: Overview of the study
Chapter \nameref{sec:intro} presents Morocco's hydrogen strategy and climate targets and outlines potential conflicts and synergies between these two goals. Introducing the results along these three dimensions step-by-step, \nameref{sec:results} evaluates synergies and conflicts based on prices and total cost for domestic electricity consumers and hydrogen exporters, with emphazised consideration of temporal hydrogen regulation. These results are contextualized along policy recommendations in the \nameref{sec:discussion}. \nameref{sec:methods} presents the applied sector-coupled energy model and scenario dimensions required to expose the export-mitigation-regulation nexus.
