

\begin{itemize}
    \item Background and context of the emerging role of hydrogen as a clean energy carrier
    \item Importance of addressing both hydrogen export and national energy transition
    \item Significance of Morocco as a case study
    \item DO: Include sources etc.!
\end{itemize}


The global energy environment is experiencing fundamental upheaval, driven by the need to reduce greenhouse gas emissions and transition to a low-carbon future. In this context, hydrogen has emerged as a viable clean energy carrier capable of addressing the issues of decarbonizing numerous sectors such as industry, transportation, and power generation. A rising number of countries are investigating green hydrogen generation and use as a crucial component of their strategy to cut greenhouse gas emissions and meet ambitious climate targets.

Simultaneously, several countries are positioning themselves as potential exporters of green hydrogen, seeing an opportunity to leverage on their renewable energy resources and technological advances. These countries are expected to play a significant role in the global energy market by supplying clean hydrogen and therefore contributing to global decarbonization efforts. However, pursuing both hydrogen export and national energy transition poses complex problems and trade-offs that must be carefully considered.

Morocco is an excellent case study for managing these complexities. Morocco, which has vast solar and wind resources, has set ambitious renewable energy adoption objectives, displaying a commitment to lowering its own greenhouse gas emissions. At the same time, the country is strategically positioned to use its excess renewable energy to create and export green hydrogen, opening the door to substantial economic opportunities. Furthermore, the country is a net importer of energy, and green hydrogen could help reduce its dependence on energy imports. The proximity to Europe, which is expected to be a major hydrogen importer, makes Morocco an attractive potential exporter.

This research paper aims to thoroughly investigate the synergies and conflicts that arise from Morocco's dual goals of hydrogen export and national energy transition. To simulate a variety of scenarios, we use a fully sector-coupled capacity expansion and dispatch model of Morocco, which includes both gas and electricity networks.

By examining various hydrogen export volumes and climate targets, we evaluate the potential impact of hydrogen exports on the economic welfare of both hydrogen-exporting nations and the local Moroccan population and businesses. (We explore the intricacies of technology lock-in, land use implications, and the potential risks of neo-colonialism, providing a comprehensive overview of the challenges and opportunities that emerge while pursuing these dual objectives.)

Our analysis of Morocco's energy system contributes to the expanding research that aims to inform policymaking and decision-making regarding sustainable energy transitions. Our study provides valuable insights into the pathways that can unlock synergies and reduce conflicts between hydrogen exports and internal decarbonization, enabling Morocco and other potential exporting regions to follow a harmonious and sustainable trajectory while facing similar energy dilemmas.


\subsection{Ideas}
Narrative: 
\begin{itemize}
    \item Security of supply/economic growth/cheap energy for North Africa and Morocco
    \item Fair cooperation and partnerships by: including local energy demands, investigation of conflicts/synergies
    \item Best locations with high FLH should not be reserved for export (as done in Global PtX Atlas)
    \item EU can rely on cost-effective green hydrogen
    \item Discuss why it is important to understand synergies and conflicts
    \item Cannibalization: https://www.climatechangenews.com/2023/04/11/green-hydrogen-rush-risks-energy-cannibalisation-in-africa-analysts-say/
\end{itemize}

Details on benefits for MAR:
\begin{itemize}
    \item MAR is ramping up hydrogen strategy
    \item MAR is a net importer of energy (Currently importing 90 \%), could become exporter
    \item Local production of RE plants possible, as stated in \cite{Ersoy2022}
    \item Avoid "gr{\"u}ner Energie-Imperialismus" (Habeck: when visiting Namibia ) \cite{HabeckEnergieimperialismus}
\end{itemize}



Main research question:
\begin{itemize}
    \item Is it beneficial (total system cost, lcoh, lcoe, emissions) to (simultaneously) push the national energy transition and hydrogen export (synergies and conflicts)?   
\end{itemize}

Research questions:
\begin{itemize}
    \item Are there wealth, cost and emission benefits of an integrated optimisation compared to separated analyses?
    \item Are there competing RE potential resources and where are they located? Should these resources used for local decarbonization or for export purposes?
    \item What is the role of green/grey hydrogen?   
    \item Can an integrated analysis avoid carbon leakage?
    \item Does ongrid integration avoid installing excess RES and El capacity? (That's a finding of e.g. \cite{Ruhnau2022})
    \item What are the conflicts and benefits of national transition and hydrogen export?
\end{itemize}

Literature review (general):
\begin{itemize}
    \item Overview of the global hydrogen market and potential exporting countries
    \item Key challenges and opportunities of simultaneous hydrogen export and energy transition
    \item Previous research on technology lock-in, local emissions, land use, and neo-colonialism
\end{itemize}

Literature review (specific):
\begin{itemize}
    \item Local energy demand considered in: MenaFuels and \cite{Hampp2021}. According to MenaFuels the local demand is usually just some percentages of the potential
    \item TBD: Literature on similar topics 
    \item Identify key concepts, theories, and case studies that are relevant to research question
    \item General: Work out differences to existing literature
    \item \cite{Zwaan2021} Also looks into North Africa, but on country resolution. they don't look at biomass either
\end{itemize}


\subsection{Hydrogen export strategy and climate targets}
\label{subsec:policyandtargets}
Morocco has implemented various strategies and policies to reach its climate targets and promote hydrogen exports. These include programs on the expansion of RE, the development of hydrogen for local demands and exports as well as climate targets.


\subsubsection{Moroccan Hydrogen Strategy}
Figure \ref{fig:mar_hydrogen_strategy} shows the hydrogen strategy of Morocco, including
national (ammonia, hydrogen) and export demands (synfuels, hydrogen).
By 2030, the total hydrogen demand adds up to 13.5 TWh (2040: 58 TWh, 2050: 151 TWh), 
clearly listing higher demands for export than for local use. The hydrogen generation is backed by 5.2 GW of RE in 2030, 23 GW in 2040 and 55 GW in 2050 [Source: TBD].

\subsubsection{Climate targets}
In addition to the hydrogen strategy, Morocco has obliged to climate targets.
Figure \ref{fig:morocco_em} displays the historical emissions and targets of Morocco.
According to the self-defined national climate pledges under the Paris Agreement (NDC's), Morocco aims to limit historically rising GHG emissions to 75 MtCO2e (conditional) respectively 115 MtCO2e (unconditional) excluding LULUCF by 2030 \cite{CAT2021}. 
As stated in \cite{CAT2021}, Morocco has not yet submitted a net-zero target. (Further source on NDCs: https://www.iges.or.jp/en/pub/iges-indc-ndc-database/en )
TBD: A GHGe reduction of 32 \% by 2030 is mentioned in \cite[5]{Boulakhbar2020}, check that.

\subsubsection{Synergies and conflicts}
Both the hydrogen export strategy and climate targets imply a significant expansion of RE capacities. Furthermore, both strategies require a significant energy infrastructure development of e.g. electricity grid, hydrogen pipelines, CO2 network, ports and a significant scale-up of electrolyzers.
Limited resources (e.g. land availability, renewable potentials, workforce, capital) require careful planning to minimize conflicts and maximize possible synergies.



\begin{figure}
    \centering
    \includegraphics[width=\linewidth]{../graphics_general/policy/mar_hydrogen_strategy_semistacked.pdf}
    \caption{Morocco Hydrogen Strategy, cf. \cite[p. 14]{Ersoy2022}. TBD primary source. TBD include RE capacities according to Hydrogen Strategy?}
    \label{fig:mar_hydrogen_strategy}
\end{figure}




\begin{figure}[h!]
    \centering
    \includegraphics[width=\linewidth]{../graphics_general/morocco_em.pdf}
    \caption{Moroccos historical emissions and targets.}
    \label{fig:morocco_em}
\end{figure}


\subsection{Misc}

Current cooperations/business:
\begin{itemize}
    \item Indias role: https://www.moroccoworldnews.com/2023/02/353972/indias-renew-energy-global-eyes-moroccos-green-hydrogen-market
    \item TBD: Investigation of EU/Germany involvement in MAR?
    \item Cooperation between port authorities in Tanger Med and Hamburg
\end{itemize}

\subsubsection{Moroccan RE Strategy}
The Moroccan RE strategy consists of a "Moroccan Solar Plan" initiated in 2009 \cite[p. 2]{Boulakhbar2020}
and a wind energy program.
The Moroccan wind energy program promotes 2000 MW by 2020 and 2600 MW by 2030 \cite[p. 4]{Boulakhbar2020}
The solar plan ... (TBD: Describe the solar plan)
(See \cite[p. 13]{Ersoy2022} for a good overview of the Moroccan RE strategy).