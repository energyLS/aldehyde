The global energy environment is experiencing fundamental upheaval, driven by the need to reduce greenhouse gas emissions and transition to a low-carbon future. In this context, hydrogen has emerged as a viable clean energy carrier capable of addressing the issues of decarbonizing numerous sectors such as industry, transport, heating and power generation. A growing number of countries are investigating green hydrogen production and use as a crucial component of their strategy to cut greenhouse gas emissions and meet ambitious climate targets.


Simultaneously, several countries are positioning themselves as potential exporters of green hydrogen, discovering an opportunity to leverage on their renewable energy resources and technological advances. % [TBD: add examples]
These countries are expected to play a significant role in the global energy market by supplying clean hydrogen and therefore contributing to global decarbonization efforts. However, pursuing both hydrogen export and national energy transition poses complex problems and trade-offs that require careful consideration.

Morocco serves as a blueprint for investigating these dependencies. Morocco, which has vast solar and wind resources \cite{Peters2023, Touili2018, Sterl2022}, has proposed ambitious renewable energy adoption objectives, displaying a commitment to lowering its own greenhouse gas emissions \cite{CAT2021}. At the same time, the country is strategically positioned to deploy its renewable energy potentials to export green hydrogen, opening the door to substantial economic opportunities (and local value chains \cite{Ersoy2022}). Furthermore, the country is a net importer of energy \cite{IEA2022}, and green hydrogen could help reduce its dependency on energy imports. The proximity to Europe, which is expected to be a major hydrogen importer, makes Morocco an attractive potential exporter. %https://www.giz.de/de/downloads/giz2012-en-solar-plan-morocco.pdf

These conditions are the motivation of several studies \cite{Wijk2021, AbouSeada2022, Zwaan2021, Schellekens2010, Cavana2021, Touili2022, Timmerberg2019, Sens2022} examining the potential of hydrogen in Africa and synergies with European demand. A more detailed study of Morocco has been undertaken by \cite{Boulakhbar2020} examining challenges in integrating Renewable Energies, \cite{Khouya2020} determining the Levelized Cost of Hydrogen based on CSP and wind farms, and \cite{Touili2018} investigating the potential of hydrogen from solar energy. \cite{Hampp2021} investigates various PtX products and their transportation to Europe, and \cite{Eichhammer2019} highlights diverse opportunities and challenges related to exporting hydrogen and PtX from Morocco. While several studies\cite{Hampp2021, AbouSeada2022, Wijk2021} have examined the potential of hydrogen as a low-carbon energy carrier and others have explored various countries' climate targets and aspirations \cite{Boulakhbar2020}, a significant research gap remains regarding the complete integration of both perspectives. The literature review indicates that numerous studies focus exclusively on either hydrogen potentials or climate ambitions, without discussing the complex relationship between these two critical aspects.

This research paper aims to investigate the synergies and conflicts that arise from Morocco's dual goals of hydrogen export and national energy transition. To simulate and optimise a variety of scenarios, we use a fully sector-coupled capacity expansion and dispatch model of Morocco, which includes both gas pipelines and electricity networks.
Shortcomings of studies on highly renewable energy systems in Africa as low temporal and spatial resolutions as well as the lack of sector-coupled energy as pointed out in \cite{Oyewo2023} are tackled in this study.

By examining various hydrogen export volumes and climate targets, we evaluate the potential impact of hydrogen exports on hydrogen and electricity prices of both the hydrogen export industry and the local Moroccan population and businesses.
% We explore the intricacies of technology lock-in, local energy-related greenhouse gas emissions, rising local energy prices, land use implications and neo-colonialism \cite{HabeckEnergieimperialismus, TerraponPfaff2019, Hanger2016}, providing a comprehensive overview of the challenges and opportunities that emerge while pursuing these dual objectives.

The analysis of Morocco's energy system contributes to the expanding research that aims to inform policymaking and decision-making regarding sustainable energy transitions. This study provides valuable insights into the pathways unlocking synergies and reducing conflicts between hydrogen exports and national energy transition, enabling Morocco and other potential exporting regions to follow a harmonious and sustainable trajectory while facing similar energy system planning challenges.



%\subsection{Miscellaneous}
%
%Current cooperations/business:
%\begin{itemize}
%    \item Indias role: https://www.moroccoworldnews.com/2023/02/353972/indias-renew-energy-global-eyes-moroccos-green-hydrogen-market
%    \item TBD: Investigation of EU/Germany involvement in MAR?
%    \item Cooperation between port authorities in Tanger Med and Hamburg
%\end{itemize}

%\subsubsection{Moroccan RE Strategy}
%The Moroccan RE strategy consists of a "Moroccan Solar Plan" initiated in 2009 \cite[p. 2]{Boulakhbar2020}
%and a wind energy program.
%The Moroccan wind energy program promotes 2000 MW by 2020 and 2600 MW by 2030 \cite[p. 4]{Boulakhbar2020}
%The solar plan ... (TBD: Describe the solar plan)
%(See \cite[p. 13]{Ersoy2022} for a good overview of the Moroccan RE strategy).

%Literature review (specific):
%\begin{itemize}
%    \item Local energy demand considered in: MenaFuels and \cite{Hampp2021}. According to MenaFuels the local demand is usually just some percentages of the potential
%    \item Hydrogen from (North-)Africa (to Europe): \cite{Wijk2021} \cite{AbouSeada2022} \cite{Zwaan2021} and \cite{Schellekens2010} and \cite{Cavana2021}, specific countries from Africa: \cite{Hampp2021}, PV technologies for Hydrogen production in Africa \cite{Touili2022}, Pipeline blend \cite{Timmerberg2019}, hydrogen cost \cite{Sens2022}
%    \item Morocco: Renewable Integration \cite{Boulakhbar2020}, LCOH hydrogen CSP and wind farms \cite{Khouya2020}, Discussion paper PtX \cite{Eichhammer2019}, hydrogen potential from solar energy \cite{Touili2018}
%    \item Literature review on energy systems in Africa \cite{Oyewo2023} and \cite{Oyewo2022}
%    \item On geopolitics: \cite{Pflugmann2020} and \cite{Graaf2020}
%    \item \cite{Zwaan2021} Also looks into North Africa, but on country resolution. they don't look at biomass either
%    \item Overview of the global hydrogen market and potential exporting countries
%    \item Key challenges and opportunities of simultaneous hydrogen export and energy transition
%    \item Previous research on technology lock-in, local emissions, land use, and neo-colonialism
%\end{itemize}


%\subsection{Ideas}
%Narrative: 
%\begin{itemize}
%    \item Security of supply/economic growth/cheap energy for North Africa and Morocco
%    \item Fair cooperation and partnerships by: including local energy demands, investigation of conflicts/synergies
%    \item Best locations with high FLH should not be reserved for export (as done in Global PtX Atlas)
%    \item EU can rely on cost-effective green hydrogen
%    \item Discuss why it is important to understand synergies and conflicts
%    \item Cannibalization: https://www.climatechangenews.com/2023/04/11/green-hydrogen-rush-risks-energy-cannibalisation-in-africa-analysts-say/
%\end{itemize}


% Main research question:
% \begin{itemize}
%     \item Is it beneficial (total system cost, lcoh, lcoe, emissions) to (simultaneously) push the national energy transition and hydrogen export (synergies and conflicts)?   
% \end{itemize}


% Research questions:
% \begin{itemize}
%     \item Are there wealth, cost and emission benefits of an integrated optimisation compared to separated analyses?
%     \item Are there competing RE potential resources and where are they located? Should these resources used for local decarbonization or for export purposes?
%     \item What is the role of green/grey hydrogen?   
%     \item Can an integrated analysis avoid carbon leakage?
%     \item Does ongrid integration avoid installing excess RES and El capacity? (That's a finding of e.g. \cite{Ruhnau2022})
%     \item What are the conflicts and benefits of national transition and hydrogen export?
% \end{itemize}