
\subsection{Ideas}
From big to small:
\begin{itemize}
    \item Why is this research important? Climate change, COP27, hydrogen, local decarbonization, combined efforts (assets: especially with scarcity of electrolyzer capacity, we should use them better)
    \item What is the aim?
    \item What are the results?
\end{itemize}

Narrative: 
\begin{itemize}
    \item Security of supply/economic growth/cheap energy for North Africa and Morocco
    \item Fair cooperation and partnerships by: including local energy demands, investigation of conflicts/synergies
    \item Best locations with high FLH should not be reserved for export (as done in Global PtX Atlas)
    \item EU can rely on cost-effective green hydrogen
    \item Discuss why it is important to understand synergies and conflicts
\end{itemize}

Details on benefits for MAR:
\begin{itemize}
    \item MAR is ramping up hydrogen strategy
    \item MAR is a net importer of energy (Currently importing 90 \%), could become exporter
    \item Local production of RE plants possible, as stated in \cite{Ersoy2022}
    \item Avoid "gr{\"u}ner Energie-Imperialismus" (Habeck: when visiting Namibia ) \cite{HabeckEnergieimperialismus}
\end{itemize}

Current cooperations/business:
\begin{itemize}
    \item Indias role: https://www.moroccoworldnews.com/2023/02/353972/indias-renew-energy-global-eyes-moroccos-green-hydrogen-market
    \item TBD: Investigation of EU/Germany involvement in MAR?
    \item Cooperation between port authorities in Tanger Med and Hamburg
\end{itemize}

Main research question:
\begin{itemize}
    \item Is it beneficial (total system cost, lcoh, lcoe, emissions) to (simultaneously) push the national energy transition and hydrogen export (synergies and conflicts)?   
\end{itemize}

Research questions:
\begin{itemize}
    \item Are there wealth, cost and emission benefits of an integrated optimisation compared to separated analyses?
    \item Are there competing RE potential resources and where are they located? Should these resources used for local decarbonization or for export purposes?
    \item What is the role of green/grey hydrogen?   
    \item Can an integrated analysis avoid carbon leakage?
    \item Does ongrid integration avoid installing excess RES and El capacity? (That's a finding of e.g. \cite{Ruhnau2022})
    \item What are the conflicts and benefits of national transition and hydrogen export?
\end{itemize}

Literature review:
\begin{itemize}
    \item Local energy demand considered in: MenaFuels and \cite{Hampp2021}. According to MenaFuels the local demand is usually just some percentages of the potential
    \item TBD: Literature on similar topics 
    \item Identify key concepts, theories, and case studies that are relevant to research question
    \item General: Work out differences to existing literature
    \item \cite{Zwaan2021} Also looks into North Africa, but on country resolution. they don't look at biomass either
\end{itemize}


\subsection{Policy and climate targets}
\label{subsec:policyandtargets}
Morocco has implemented various strategies and policies to reach its climate targets and promote hydrogen exports. 
These include programs on the expansion of RE, the development of hydrogen for local demands and exports as well as climate targets.

\subsubsection{Moroccan RE Strategy}
The Moroccan RE strategy consists of a "Moroccan Solar Plan" initiated in 2009 \cite[p. 2]{Boulakhbar2020}
and a wind energy program.
The Moroccan wind energy program promotes 2000 MW by 2020 and 2600 MW by 2030 \cite[p. 4]{Boulakhbar2020}
The solar plan ... (TBD: Describe the solar plan)
(See \cite[p. 13]{Ersoy2022} for a good overview of the Moroccan RE strategy).

\subsubsection{Moroccan Hydrogen Strategy}
Figure \ref{fig:mar_hydrogen_strategy} shows the hydrogen strategy of Morocco, including
national (ammonia, hydrogen) and export demands (synfuels, hydrogen).
By 2030, the total hydrogen demand adds up to 13.5 TWh (2040: 58 TWh, 2050: 151 TWh), 
clearly listing higher demands for export than for local use. The hydrogen generation is backed by 5.2 GW of RE in 2030, 23 GW in 2040 and 55 GW in 2050 [TBD].


\begin{figure}
    \centering
    \includegraphics[width=\linewidth]{../graphics_general/policy/mar_hydrogen_strategy_grouped.pdf}
    \caption{Morocco Hydrogen Strategy, cf. \cite[p. 14]{Ersoy2022}. TBD primary source. TBD include RE capacities according to Hydrogen Strategy?}
    \label{fig:mar_hydrogen_strategy}
\end{figure}

\subsubsection{Climate targets}
Apart from the RE and Hydrogen strategies, Morocco has obliged to climate targets.
Figure \ref{fig:morocco_em} displays the historical emissions and targets of Morocco.
By 2030, Morocco aims to limit historically rising GHG emissions to 75 MtCO2e (conditional) 
respectively 115 MtCO2e (unconditional) (NDC, excluding LULUCF) \cite{CAT2021}. 
According to \cite{CAT2021}, Morocco has not yet submitted a net-zero target.

TBD: A GHGe reduction of 32 \% by 2030 is mentioned in \cite[5]{Boulakhbar2020}, check that.
NDCs: https://www.iges.or.jp/en/pub/iges-indc-ndc-database/en 


\begin{figure}[h!]
    \centering
    \includegraphics[width=\linewidth]{../graphics_general/morocco_em.pdf}
    \caption{Morocco historical emissions and targets.}
    \label{fig:morocco_em}
\end{figure}