\subsection*{Hydrogen export strategy and climate targets}

Morocco has implemented various strategies and policies to reach its climate targets and promote hydrogen exports. These include programs on the expansion of RE, the development of hydrogen for domestic demands and exports as well as climate targets \cite{MarHyStrat2021, CAT2021}.

% \subsection*{Moroccan Hydrogen Strategy}
% \label{subsec:strategy}
Figure \ref{fig:mar_hydrogen_strategy} shows the hydrogen strategy of Morocco \cite{MarHyStrat2021}, including national and export demands.
By 2030, the total hydrogen demand adds up to 13.9 TWh/a and ramps up to 67.9 TWh/a in 2040 and 153.9 TWh/a in 2050, 
clearly listing higher demands for export than for domestic use. The hydrogen generation is backed by 5.2 GW of RE in 2030, 23 GW in 2040, and 57.4 GW in 2050 in the export sector \cite{MarHyStrat2021}. To cover the national demand of hydrogen, a Renewable Energy deployment of 1.6 GW in 2030, 7.0 GW in 2040 and 10.3 GW in 2050 is planned in Morocco's Hydrogen Strategy \cite{MarHyStrat2021}.


% \subsection{Climate targets}
% \label{subsec:climate_targets}
In addition to the hydrogen strategy, Morocco has obliged to climate targets. Figure \ref{fig:morocco_em} displays the historical emissions and targets of Morocco.
According to the self-defined national climate pledges under the Paris Agreement (NDC's), Morocco aims to limit historically rising greenhouse gas emissions to 75 Mt\coe\ (conditional) respectively 115 Mt\coe\ (unconditional) excluding LULUCF by 2030 \cite{CAT2021}. 
As stated by the Climate Action Tracker\cite{CAT2021}, Morocco has not yet submitted a net-zero target. 


% \subsection{Synergies and conflicts}
Both the hydrogen export strategy and climate targets imply a substantial expansion of RE capacities. Furthermore, both strategies require an energy infrastructure development of e.g. electricity grid, hydrogen pipelines, $\mathrm{CO_2}$ network, ports and a considerable scale-up of electrolysers.
Limited resources (e.g. land availability, renewable potentials, workforce, capital) require careful planning to minimize conflicts and maximize possible synergies.

\begin{figure}
    \centering
    \includegraphics[width=\linewidth]{../graphics_general/policy/mar_hydrogen_strategy_semistacked.pdf}
    \caption{Hydrogen Strategy of Morocco \cite{MarHyStrat2021}. The export volume rises up to 115~TWh/a in 2050, the national demand up to 40 TWh/a in 2050.}
    \label{fig:mar_hydrogen_strategy}
\end{figure}


\begin{figure}[h!]
    \centering
    \includegraphics[width=\linewidth]{../graphics_general/morocco_em.pdf}
    \caption{Morocco's historical \co and greenhouse gas emissions (GHG). While emissions are rising, emission targets are in the range between 75 Mt\coe\ (conditional) and 115 Mt\coe\ (unconditional).}
    \label{fig:morocco_em}
\end{figure}
