Morocco has implemented various strategies and policies to reach its climate targets and promote hydrogen exports. These include programs on the expansion of RE, the development of hydrogen for local demands and exports as well as climate targets \cite{MarHyStrat2021, CAT2021}.
%Renewable energy potential and national policy directions for sustainable development in Morocco \cite{Kousksou2015}


\subsection{Moroccan Hydrogen Strategy}
Figure \ref{fig:mar_hydrogen_strategy} shows the hydrogen strategy of Morocco, including national and export demands.
By 2030, the total hydrogen demand adds up to 13.9 TWh and ramps up to 67.9 TWh in 2040 and 153.9 TWh in 2050, 
clearly listing higher demands for export than for local use. The hydrogen generation is backed by 5.2 GW of RE in 2030, 23 GW in 2040, and 57.4 GW in 2050 in the export sector. To cover the national demand of hydrogen, a Renewable Energy deployment of 1.6 GW in 2030, 7.0 GW in 2040 and 10.3 GW in 2050 is planned in \cite{MarHyStrat2021}.


\subsection{Climate targets}
In addition to the hydrogen strategy, Morocco has obliged to climate targets. Figure \ref{fig:morocco_em} displays the historical emissions and targets of Morocco.
According to the self-defined national climate pledges under the Paris Agreement (NDC's), Morocco aims to limit historically rising GHG emissions to 75 MtCO2e (conditional) respectively 115 MtCO2e (unconditional) excluding LULUCF by 2030 \cite{CAT2021}. 
As stated in \cite{CAT2021}, Morocco has not yet submitted a net-zero target. 
%TBD: 
%A GHGe reduction of 32 \% by 2030 is mentioned in \cite[5]{Boulakhbar2020}, check that.
% (Further source on NDCs: https://www.iges.or.jp/en/pub/iges-indc-ndc-database/en )

\subsection{Synergies and conflicts}
Both the hydrogen export strategy and climate targets imply a significant expansion of Renewable Energy (RE) capacities. Furthermore, both strategies require a significant energy infrastructure development of e.g. electricity grid, hydrogen pipelines, $\mathrm{CO_2}$ network, ports and a significant scale-up of electrolysers.
Limited resources (e.g. land availability, renewable potentials, workforce, capital) require careful planning to minimize conflicts and maximize possible synergies.

\begin{figure}
    \centering
    \includegraphics[width=\linewidth]{../graphics_general/policy/mar_hydrogen_strategy_semistacked.pdf}
    \caption{Morocco Hydrogen Strategy, see \cite{MarHyStrat2021}.}
    \label{fig:mar_hydrogen_strategy}
\end{figure}


\begin{figure}[h!]
    \centering
    \includegraphics[width=\linewidth]{../graphics_general/morocco_em.pdf}
    \caption{Moroccos historical emissions and targets.}
    \label{fig:morocco_em}
\end{figure}
