Morocco has implemented various strategies and policies to reach its climate targets and promote hydrogen exports. These include programs on the expansion of RE, the development of hydrogen for local demands and exports as well as climate targets [TBD: add sources].
%Renewable energy potential and national policy directions for sustainable development in Morocco \cite{Kousksou2015}


\subsection{Moroccan Hydrogen Strategy}
Figure \ref{fig:mar_hydrogen_strategy} shows the hydrogen strategy of Morocco, including
national (ammonia, hydrogen) and export demands (synfuels, hydrogen).
By 2030, the total hydrogen demand adds up to 13.5 TWh (2040: 58 TWh, 2050: 151 TWh), 
clearly listing higher demands for export than for local use. The hydrogen generation is backed by 5.2 GW of RE in 2030, 23 GW in 2040 and 55 GW in 2050 [Source: TBD].

\subsection{Climate targets}
In addition to the hydrogen strategy, Morocco has obliged to climate targets.
Figure \ref{fig:morocco_em} displays the historical emissions and targets of Morocco.
According to the self-defined national climate pledges under the Paris Agreement (NDC's), Morocco aims to limit historically rising GHG emissions to 75 MtCO2e (conditional) respectively 115 MtCO2e (unconditional) excluding LULUCF by 2030 \cite{CAT2021}. 
As stated in \cite{CAT2021}, Morocco has not yet submitted a net-zero target. (Further source on NDCs: https://www.iges.or.jp/en/pub/iges-indc-ndc-database/en )
TBD: A GHGe reduction of 32 \% by 2030 is mentioned in \cite[5]{Boulakhbar2020}, check that.

\subsection{Synergies and conflicts}
Both the hydrogen export strategy and climate targets imply a significant expansion of Renewable Energy (RE) capacities. Furthermore, both strategies require a significant energy infrastructure development of e.g. electricity grid, hydrogen pipelines, CO2 network, ports and a significant scale-up of electrolysers.
Limited resources (e.g. land availability, renewable potentials, workforce, capital) require careful planning to minimize conflicts and maximize possible synergies.

\begin{figure}
    \centering
    \includegraphics[width=\linewidth]{../graphics_general/policy/mar_hydrogen_strategy_semistacked.pdf}
    \caption{Morocco Hydrogen Strategy, cf. \cite[p. 14]{Ersoy2022}. TBD primary source. TBD include RE capacities according to Hydrogen Strategy?}
    \label{fig:mar_hydrogen_strategy}
\end{figure}


\begin{figure}[h!]
    \centering
    \includegraphics[width=\linewidth]{../graphics_general/morocco_em.pdf}
    \caption{Moroccos historical emissions and targets.}
    \label{fig:morocco_em}
\end{figure}
