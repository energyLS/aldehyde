Chapter \ref{sec:results} focuses on the energy system fulfilling both goals of hydrogen export and national energy transition under various green hydrogen regulations. First, Section \ref{subsec:increase_limit} investigates the supply and demand of Moroccos electricity system depending on domestic mitigation (dimension 1), Section \ref{subsec:increase_h2} analyses the hydrogen export ramp up (dimension 2). Section \ref{subsec:benefits} reveals economic co-benefits of domestic mitigation and hydrogen exports along both dimensions 1 and 2. The role of hydrogen regulation and temporal matching (dimension 3) is outlined in Section \ref{subsec:benefits_rule}. Section \ref{subsec:rule_all} investigates the effect of hydrogen regulation depending on domestic mitigation and hydrogen exports (dimensions 1, 2 and 3).


% The integration challenges arising from increasing RE penetration are pointed out in Section \ref{subsec:integration_challenges} while emphasizing the importance of system flexibility provided by electricity storage and curtailment. The hydrogen and electricity prices of the integrated optimisation are analyzed in Section \ref{subsec:results_el_prices} and \ref{subsec:results_hy_prices}. 


%Based on these prices, Section \ref{subsec:economic_benefits} derives the total expenses of electricity and hydrogen and outlines economics benefits. 
%The sensitivity analysis in Section \ref{subsec:sensitivity} examines key parameters and their influence on the main results.

\begin{figure*}[h!]
    \centering
    \begin{subfigure}[b]{0.49\linewidth}
        \centering
        \includegraphics[trim={0cm 0cm 0cm 1cm}, clip, width=\linewidth]{../../workflow/subworkflows/pypsa-earth-sec/results/\runstandard/0exp-only/graphs/balances-AC.pdf}
        \caption{Fixed (1 TWh) export and 0--100\% domestic mitigation}
        \label{fig:balances-ac-0exp}
    \end{subfigure}
    \hfill
    \begin{subfigure}[b]{0.49\linewidth}
        \centering
        \includegraphics[trim={0cm 0cm 0cm 1cm}, clip, width=\linewidth]{../../workflow/subworkflows/pypsa-earth-sec/results/\runstandard/co2l20-only/graphs/balances-AC.pdf}
        \caption{Fixed (0\%) domestic mitigation and 0--200 TWh export}
        \label{fig:balances-ac-co2l20}
    \end{subfigure}
    \hfill
    \caption{Electricity supply and demand at fixed export levels and increasing domestic mitigation export (\ref{fig:balances-ac-0exp}) and vice versa (\ref{fig:balances-ac-co2l20}). Increasing domestic mitigation first phases out carbon-intensive coal generation in favor of CCGT, at medium to high mitigation the electricity system is fully renewable supported by flexibility through V2G and sector coupling. Increasing electricity demands cover EVs and hydrogen generation for other sectors.
    At increasing hydrogen exports the additional electricity required for hydrogen electrolysis is covered by onshore wind and solar PV, as imposed by the hydrogen regulation. 
    %See Fig. \ref{fig:balances_AC_nogreen} for electricity balance without hydrogen regulation.
    }
    \label{fig:balances-ac}
\end{figure*}


\subsection{Domestic mitigation}
\label{subsec:increase_limit}

Figure \ref{fig:balances-ac-0exp} depicts the supply and demand of the electricity system at increasing climate mitigation ambitions from 0\% to 100\%. At low climate ambitions, the electricity demand is mainly covered by coal power, existing (brownfield) capacities of onshore wind, hydro, and CCGT play only a minor role. 
At increasing emissions reductions, coal power is phased out in favour of solar PV, furthermore a fuel switch from coal to gas (CCGT) is observable at increasing emission reductions. At medium climate mitigation ambitions, the dipatchable power in the electricity system is provided by OCGT. 
Further increasing the climate ambitions, both solar PV and wind onshore penetrate and dominate the electricity system complemented by dispatchable power from Vehicle-to-Grid (V2G) providing an almost fully renewable electricity sector at 100\% emission reduction. 

At 70\% and above, the electricity demand of eletrolysers increases significantly to supply hydrogen allowing a switch from fossil oil products to Fischer-Tropsch fuels in various sectors (see breakdown in \ref{fig:oil-balance}). 
The Fischer-Tropsch demand in the transport sector increases, even though the increasing BEV diffusion (s. Sec. \ref{subsec:bev_diffusion}) is counterbalancing the demand for oil products. 
Both electrolysers and BEVs drive the electricity demand significantly up to five times of the bulk electricity demand. 
Additionally, the electricity demand for air heat pumps increases, supporting the defossilisation in the heating sector (s. Sec. TODO) and supplying heat to the DAC process (s. Sec. TODO) required for synthetic fuels.




\subsection{Hydrogen exports}
\label{subsec:increase_h2}

The electricity supply at increasing hydrogen exports is displayed in Figure \ref{fig:balances-ac-co2l20}. The deployment of hydrogen exports requires a scale up of onshore wind and solar PV accordingly, as defined by the green hydrogen constraint (s. Sec. \ref{subsec:green_hydrogen_constraint}).
Without the green hydrogen constraint, the additional electricity demand of hydrogen exports would be covered by coal power plants and CCGT (by increasing the capacity factor of existing brownfield coal and CCGT capacities) and an expansion of OCGT installation and supply as pointed out in Section \ref{subsec:gh_constraint_effects}.
In contrast to the expontential increase of electricity demand observable at increasing climate mitigation ambitions (s. Fig.~\ref{fig:balances-ac-0exp}), the electricity demand increases linearly with the hydrogen export ambitions.


\subsection{Co-benefits of local mitigation and hydrogen export}
\label{subsec:benefits}

Both domestic mitigation (s. Sec. \ref{subsec:increase_limit}) and hydrogen exports (s. Sec. \ref{subsec:increase_h2}) profit from each other and show co-benefits. In this Section, we i) show the effects of hydrogen exports on domestic electricity prices, ii) the role of domestic mitigation on hydrogen export expenses, and lastly iii) common co-benefits.

First, Figure \ref{fig:expense_ac} shows the costs for domestic electricity consumers subject to mitigation and hydrogen export volumes. The expenses are normalized to expenses at 1 TWh Hydrogen export at each mitigation level, displaying relative changes for domestic electricity consumers induced by hydrogen exports at a certain mitigation.
Hydrogen exports decrease the domestic electricity expenses by up to 56\%, especially at mitigation below 40\% and exports above 50 TWh. In these ranges, the fossil dominated domestic electricity system profits from excess green electricity originating from additional renewable energy capacities required for hydrogen export (see details on the mechanism in Section \ref{subsec:benefits_rule}). The more the domestic electricity system is decarbonized (domestic mitigation above 50--60\%), the weaker is the decrease of domestic electricity prices.
The only increase of expenses induced by hydrogen exports is observable at a low mitigation of 10\% and exports of 20 TWh. (TODO: explain why).


Second, Figure \ref{fig:expense_h2} shows the expenses of hydrogen exports subject to mitigation and hydrogen export volumes. The expenses are normalized to expenses at 0\% mitigation at each hydrogen export volume, displaying relative changes for hydrogen exports from domestic mitigation at a certain hydrogen export volume.
At hydrogen export volumes of 25--75 TWh, an increase of domestic mitigation up to 40--60\% decreases the hydrogen export prices up to -9\%. (TODO: elaborate reason).
The only price increase observable in the range of our scenarios is observable at low export volumes below 20 TWh. Here, advances in domestic mitigation increase the hydrogen export expenses up to 19\% compared to no domestic mitigation.
This expense increase results from lower electrolysis capacity factors compared to the 0 \% mitigation scenario, where we observe high electrolysis capacity factors of above 80\% (s. Fig. \ref{fig:ely-cf}) (TODO: elabore reason).
The triangle in the top left area shows that hydrogen exporter expenses at certain export levels are independent of domestic mitigation, this is driven by hydrogen regulation further investigated in Section \ref{subsec:benefits_rule}. Here, hydrogen export infrastructure decouples from the domestic electricity system, the expenses are increasingly independent of domestic mitigation. The hydrogen exports are 2-3.5 times higher than the domestic electricity demand (s. Fig. \ref{fig:expense_h2}), dwarfing the importance of the domestic electricity system.


Third, we derive co-benefits for domestic mitigation and hydrogen exports. 
Within the area of 40-60\% mitigation and 50-100 TWh hydrogen exports, we see i) domestic electricity consumers profit from hydrogen exports compared to very low (1 TWh) exports and ii) hydrogen exporters decrease their expenses compared to no (0 \%) domestic mitigation. Hence, both domestic mitigation and hydrogen exports show clear co-benefits at medium mitigation (40--60\%) and moderate to high hydrogen exports (25--150 TWh). 
% This area is realistic, refer to third punch?



\begin{figure*}[h!]
    \centering
    \begin{subfigure}[b]{0.49\linewidth}
        \centering
        \includegraphics[width=\linewidth]{graphics/integrated_comp/contour_exp_AC_exclu_H2 El_all_20_filterTrue_nTrue.pdf}
        \caption{Relative expense of local electricity customers (normalized to 1 TWh hydrogen export)}
        \label{fig:expense_ac}
    \end{subfigure}
    \hfill
    \begin{subfigure}[b]{0.49\linewidth}
        \centering
        \includegraphics[width=\linewidth]{graphics/integrated_comp/contour_exp_H2_False_False_exportonly_20_filterTrue_nTrue.pdf}
        \caption{Relative expense of hydrogen exporters (normalized to 0\% \co reduction)}
        \label{fig:expense_h2}
    \end{subfigure}
    \hfill
    \caption{  
    Expenses of local electricity consumers (\ref{fig:expense_ac}) and hydrogen exporters (\ref{fig:expense_h2}),
    normalized to expenses at 1 TWh hydrogen export (\ref{fig:expense_ac}) and
    to 0\% \co reduction (\ref{fig:expense_h2})
    at each mitigation level. Local electricity consumers profit from increasing hydrogen exports, especially at low domestic mitigation and high exports. Hydrogen exporters profit from domestic mitigation at medium mitigation efforts.}
    \label{fig:expenses_default}
\end{figure*}



\subsection{Hydrogen regulation reduces domestic electricity prices and emissions}
% Old: PUNCH 2: Effects of hydrogen regulation on electricity and hydrogen expenses (in fossil dominated electricity systems
\label{subsec:benefits_rule}

Hydrogen regulation has strong effects on electricity system dispatch (and emissions), increasing the prices and expenses for hydrogen exporters whilst decreasing for domestic electricity consumers. This Section shows the mechanisms of temporal matching in the 200 TWh export and 0\% domestic mitigation scenario.

As the temporal matching becomes stricter, additional renewable capacities and complementing hydrogen storage are required to meet the electricity demand of electrolysis on an annual, monthly and hourly basis, as shown in Figure \ref{fig:tsc-200-0}. These renewable capacities i.) push out high marginal cost fossil generation (\textit{price spillover}) and ii.) provide excess electricity to the domestic electricity system (\textit{energy spillover}).
%Merit order: the price spillover is a consequence of the energy spillover

The \textit{price spillover} is observable in Figure \ref{fig:pdc-200-0} depicting the price duration curve at 200 TWh export and 0\% domestic mitigation. The price duration curve describes the sorted price of electricity (spatial mean) at every hour of the year.
Stricter hydrogen regulation pushes the price duration curve towards the left, since additional renewable electricity capacities phase out fossil generation with higher marginal costs than renewable electricity. (Elaborate: which generation is price setting?)


The \textit{energy spillover} is shown in Figure \ref{fig:dispatch_rule}. In the \textit{no rule} scenario, which does not constrain the electricity input of hydrogen electrolysis, Coal and OCGT are running close to their (brownfield) capacity limits to supply electricity (see capacity factors in Figure \ref{fig:cf-noghc}). If annual matching is applied, the renewable electricity supply to the electricity system equals the annual demand for electrolysis, increasing the supply of solar PV by 36 TWh while decreasing the supply of fossil electricity generation. 
Monthly matching increases this trend, while hourly matching completely phases out OCGT and CCGT. Figure \ref{fig:dispatch_rule} highlights the \textit{energy spillover} effect: going beyond the electricity demand of hydrogen electrolysis, renewable capacities installations necessary to meet the temporal matching decarbonize the domestic electricity demand and grid emissions. This effect is steered by the strictness of temporal matching.
In addition to the synergies between hydrogen export and domestic mitigation pointed out in Section \ref{subsec:benefits}, the \textit{energy spillover} outlines a clear synergy between domestic mitigation and hydrogen regulation.



The \textit{price spillover} effects results in lower electricity prices for domestic consumers. Taking the electricity demand into account, the decrease of prices results in a decrease of electricity expenses shown in 
Figure \ref{fig:expense_h2ac}. Stricter hydrogen regulation decreases the domestic electricity spending from 2.8 B€ to 1.0 B€ (decrease by -65\%). In contrast, the expenses for hydrogen exporters increase from 16.9 B€ to 17.9 B€ (increase by 6\%), carrying the cost of the temporal matching constraint and hence financing additional solar PV and hydrogen storage required to meet the electrolysis electricity demand on a certain temporal basis (TODO: prove in appendix). (TODO: Combined expenses decrease: elaborate). (TODO: explain why there is a slight increase in annual/monthly generation)


To sum up, hydrogen regulation is a policy instrument acting in a twofold way. It steers the redistribution between hydrogen exports and domestic electricity consumers (based on the \textit{price spillover} effect), decreasing the expenses for domestic electricity consumers and increasing the expenses for hydrogen exporters. The \textit{energy spillover} effect not only decarbonizes hydrogen exports but decreases grid emissions for  domestic electricity demand as well. Both effects scale with stricter temporal matching.


\begin{figure}[h!]
    \centering
    \includegraphics[trim={0cm 0cm 0cm 0.65cm}, clip, width=\linewidth]{../../results/gh/price_sorted_AC_200_2.0_Hours.pdf}
    \caption{Price duration curve at 200 TWh export and 0\% domestic mitigation. Stricter hydrogen regulation pushes the price duration curve towards the left, since additional renewable electricity capacities phase out fossil generation with higher marginal costs than renewables.}
    \label{fig:pdc-200-0}
\end{figure}



\begin{figure*}[h!]
    \centering
    \begin{subfigure}[b]{0.49\linewidth}
        \centering
        \includegraphics[trim={0cm 0cm 0cm 0.65cm}, clip, width=\linewidth]{../../results/gh/dispatch_200_Co2L2.0.pdf}
        \caption{Electricity dispatch}
        \label{fig:dispatch_rule}
    \end{subfigure}
    \hfill
    \begin{subfigure}[b]{0.49\linewidth}
        \centering
        \includegraphics[trim={0cm 0cm 0cm 0.65cm}, clip, width=\linewidth]{../../results/gh/abs_sin_expenses_h2ac_200_Co2L2.0.pdf}
        \caption{Expenses of domestic electricity consumers and hydrogen exporters}
        \label{fig:expense_h2ac}
    \end{subfigure}
    \hfill
    \caption{Electricity dispatch (\ref{fig:dispatch_rule}) and expenses of consumers (\ref{fig:expense_h2ac}) for various (hydrogen) temporal matching regimes in the 200 TWh export and 0\% mitigation scenario. Stricter temporal matching decreases carbon-intensive electricity generation (coal \& gas) for hydrogen generation and even domestic electricity consumers (s. Fig. \ref{fig:dispatch_rule}). Expenses for export hydrogen generation increase to fulfill the temporal matching constraint, whereas domestic electricity consumers profit from stricter hydrogen regulation.}
    \label{fig:expenses_rule}
\end{figure*}


\subsection{Hydrogen regulation across all mitigation and export scenarios}
% Old PUNCH 3: The role of hydrogen regulation across mitigation and export scenarios
\label{subsec:rule_all}

Based on the effects of hydrogen regulation using the scenario of 200 TWh export and 0\% mitigation (s. Sec. \ref{subsec:benefits_rule}), this Section explores hydrogen regulation across all mitigation and export scenarios to derive the policy relevance of temporal matching in dependence of mitigation and export volumes. This includes certain co-benefit scenarios identified in Section \ref{subsec:benefits}.


Figure \ref{fig:expenses_all} shows the relative change of electricity and hydrogen expenses of all domestic mitigation (0--100\%) and hydrogen export (0--200 TWh) scenarios in dependence of temporal matching. 
Across all scenarios, hydrogen regulation hedges local electricity consumers against rising prices. The decrease in is most remarkable in scenarios for low mitigation and high export volumes. Here, the introduction of annual matching has already a striking effect, because the \textit{price spillover} observed in Figure \ref{fig:pdc-200-0} decreases (domestic) the electricity prices. The \textit{price spillover} unfolds in fossil-dominated (hence low domestic mitigation) systems, combined with high hydrogen exports providing a strong \textit{energy spillover} effect.
In contrast, in medium to high mitigation scenarios only the hourly matching has observable effects, since the electricity system is already  dominated by renewable electricity and only hourly matching phases out the remaining fossil generation. The differentiation among the domestic mitigation leads to these staged effects of hydrogen regulation on domestic electricity expenses.

In only three out of 121 scenarios, hourly matching regulation increases the hydrogen expenses above 7\% compared to no hydrogen regulation. These scenarios are defined by low export volumes (TODO: number) where the relative change is above 7\% but the absolute increase is comparably low.


Figure \ref{fig:expenses_all} highlights the realistic scenarios for the Morocco case. We have derived a set of mitigation-export combinations following the pathway \textit{balanced exports and mitigation} displayed in Figure \ref{fig:expenses_default_path}. These "realistic" combinations exclude the extremes of either i) quick exports and slow mitigation or ii)
slow exports and quick mitigation. The realistic scenarios show that in only three mitigation-export combinations the annual or monthly matching decreases the domestic electricity expenses, whereas all other realistic combinations are only sensitive to the strictest -- hourly matching -- hydrogen regulation. 


However, for other potential hydrogen exporters than Morocco, different mitigation-export scenarios are applicable. We outline three clusters:
\begin{enumerate}
    \item Balanced exports and mitigation, 
    \item High exports and no/low mitigation and
    \item Low-High exports and high mitigation.
\end{enumerate} 
Cluster 1 represents export regions as Morocco, pursing both domestic mitigation and hydrogen exports. Strict temporal matching has remarkable effects. In cluster 2, the hydrogen regulation is most effective, even annual or monthly matching unleashes the \textit{price spillover} and \textit{energy spillover}. In cluster 3, hydrogen regulation is not as important in these electricity system, since they are already highly or fully renewable electricity based and neither \textit{price spillover} or \textit{energy spillover} are effective.


To sum up, Section \ref{subsec:rule_all} outlines an integrated analysis of benefits and burdens for domestic electricity consumers and hydrogen exports among the dimensions of domestic mitigation, hydrogen export and hydrogen regulation. The effects of hydrogen regulation are strongest at low mitigation and high export scenarios, hydrogen regulation can effectively hedge domestic electricity consumers against rising prices across all mitigation and export scenarios. The hydrogen regulation is most effective in fossil-dominated electricity systems paired with high export volumes, providing tailored policy recommendation for export regions with different prequesites.


\begin{figure}[h!]
    \centering
    \includegraphics[trim={0cm 0cm 0cm 0.65cm}, clip, width=\linewidth]{../../results/gh/rel_multiple.pdf}
    \caption{Relative change of electricity and hydrogen expenses and total system cost depending on the green hydrogen regulation. Domestic electricity consumers profit across all export and mitigation scenarios but most at high export and low mitigation. Hydrogen exporters experience higher expenses with stricter hydrogen regulation. The hydrogen regulation regulates the welfare distribution between both groups.}
    \label{fig:expenses_all}
\end{figure}




% \subsection{Integration challenges and flexibility requirements}
% \label{subsec:integration_challenges}

% The ramp-up of variable renewable energies required for meeting emission limits (s. Sec. \ref{subsec:increase_limit}) and hydrogen export targets (s. Sec. \ref{subsec:increase_h2}) poses variable renewable integration challenges and flexibility requirements.
% The intermittent profile of solar PV (and onshore wind) is reflected in Figure \ref{fig:diurnal}, showing the diurnal residual load (electricity load minus solar PV and wind power generation) at 0 export and 0 - 100\% emission reduction. At high emission reductions, a high penetration of RE generation (especially solar PV) leads to negative residual loads at noon imposing great system integration challenges. 
% Increasing hydrogen exports and variable renewable energies accordingly leads to similar integration challenges (s. Fig. \ref{fig:diurnal_exportdim}).

% On the electricity system level, these flexibility requirements are provided by (EV) batteries and curtailment of onshore wind and solar PV (s. Fig. \ref{fig:integration_options}). %(TODO show EV batteries in appendix)
% The main drivers of increasing capacities of variable renewable energies (hydrogen export and climate ambition) are also the requiring the expansion of a hydrogen infrastructure providing additional flexibility to the electricity system. Hydrogen electrolysis is the link and enabler of additional flexibility by coupling the energy carriers electricity and hydrogen (unidirectional), displayed in Figure \ref{fig:electrolysis_p_nom_opt}. The electrolysis capacity shows a clear correlation with either the climate ambition or the export dimension. The capacity increases only slightly up to 60\% of $\mathrm{CO_2}$ reduction because the carbon emission abatement is mainly achieved in the electricity sector by phasing out carbon-intensive energy carriers as pointed out in Section \ref{subsec:increase_limit}. A further increase of the carbon limit is achieved by DAC and Fischer-Tropsch fuels, for which hydrogen (and hence electrolysis capacity) is necessary. At increasing hydrogen exports, the electrolyser capacity increases accordingly (s. Fig. \ref{fig:electrolysis_p_nom_opt}).

% The export-driven and climate-driven hydrogen infrastructure provides the ability to integrate RE at large scale, mainly due to the deployment of hydrogen storage (s. Fig. \ref{fig:integration_options}) buffering the daily generation patterns of Solar PV (for daily patterns see Figure \ref{fig:operation-ely-ft}).


% SOLUTIONS
% - Battery 
% - Battery BEV (no graph yet)
% - Hydrogen storage
% - Curtailment
% These challenges are (partly) tackled by the expansion of battery storage and increasing curtailment rates of onshore wind and solar PV (s. Fig. \ref{fig:integration_options}) at emission reductions up to $70$\%. 

% Figure \ref{fig:integration_options} reveals that at increasing hydrogen export and high emission reduction (above $70$\%) both battery storage and curtailment are reduced, due to the synergy with export driven hydrogen infrastructure and it's ability to integrate RE at large scale.

% Flexibility requirements
% Figure \ref{fig:integration_options} shows the installed capacities of battery and hydrogen storage. Both installations show a clear correlation with either the climate ambition or the export dimension. With higher climate ambitions, battery storage are deployment in larger scale at 60\% $\mathrm{CO_2}$ reduction and above, balancing the integration of variable REs (esp. solar PV). The hydrogen (underground) storage are deployed along the export dimension to buffer the hydrogen generation to meet the constant hydrogen export demand. At high climate ambition, the installed battery capacity reduces with higher export since the hydrogen storage additionally buffers daily generation patterns of Solar PV.


% \subsection{Local electricity prices}
% \label{subsec:results_el_prices}

% % General price explaination
% First, both the electricity and hydrogen prices are derived from the nodal power balance constraint, representing the prices in a "perfect" market. %\cite{Zeyen2022}
% In a second step, these temporally and spatially resolved prices are weighted based on different consumptions in 3-hourly resolution to obtain a uniform, annual and weighted marginal price of hydrogen and electricity of Morocco in each scenario. 

% The weighted marginal prices are displayed in Figure \ref{fig:marginal_prices}. Both subfigures are based on the same scenarios which varies the hydrogen export volume from 0 - 200 TWh, and the emission reduction between 0 - 100\% resulting in 121 optimisation results per subfigure (s. Chap. \ref{sec:intro}).

% The average price of electricity for local consumers (s. Fig. \ref{fig:local_el_price}) is weighted based on the local electricity demand, hence the demand of all consumers excluding eletrolyzers. 
% % Increasing the climate ambition
% It shows low prices at emissions reductions below 30\%, then increasing prices along the $\mathrm{CO_2}$ reduction dimension. The low prices result from existing (brownfield) coal capacities, where the investment costs have already been written off and cheap operational expenditures of coal power plants (without carbon pricing) offer low electricity prices. Further increasing the climate ambition leads to a higher penetration of variable renewable energies, the flexibility requirements pointed out in Section \ref{subsec:integration_challenges} increase. Due to a shift in the transport sector from ICEs to BEVs with increasing climate ambitions, V2G provides a flexibility option for the integration of variable renewable energies. Curtailment rates for onshore wind and solar PV decrease when going from 0\% to 40\% emission reduction (s. Fig. \ref{fig:integration_options}). This flexibility option is counterbalanced by the increased installation of PV and onshore wind, leading to an increase in curtailment rates from 40\% to 70\% before decreasing again at emission reductions of 70\% and above due to additional flexibility provided by electrolysis and hydrogen storage (s. Fig. \ref{fig:hystorage_cap}).




% Going up the hydrogen export
% Summary, why and how things change when increasing hydrogen exports.
% Low emission: I get RE for free and can balance with fossils
% Medium emission: Already 100% renewable electricity system has challenges integrating further REs
% High emission: Already high RE capacity and many flexibilities through electrolysis (and FT)

% At low emission reductions, additional hydrogen exports do not increase the electricity price for local consumers significantly. This is due to the fact, that additional hydrogen exports scale renewable energies accordingly as required by the green hydrogen constraint (s. Sec. \ref{subsec:green_hydrogen_constraint}). This means, that green hydrogen exports provide additional renewable energy capacity participating the electricity markets and keeping the electricity price constant at low emission reductions. Without the green hydrogen constraint, the marginal price of electricity increases with hydrogen exports as pointed out in Section \ref{subsec:gh_constraint_effects}.

% The increase of hydrogen exports at medium emission ambitions (60\%) shows a different picture, increasing the electricity price up to 58 €/MWh. At 60\% emission reduction and no hydrogen export, the electricity sector is already almost fully renewable (s. Fig. \ref{fig:balances-ac-0exp}), the flexibility provided by V2G and BEV batteries is sufficient to integrate the variable renewable energies. In this environment, an increase of hydrogen exports (and REs accordingly) pushes the electricity system to it's integration limits, requiring more storage and increasing RE curtailment as observable in Figure \ref{fig:solar_curt}, hence increasing the price of electricity.

% At high levels of emission reduction, this effect is less pronounced since the electricity system is already integrating a large amount of REs to run electrolysers and Fischer-Tropsch. Additional RE capacities induced by hydrogen exports do not significantly change the flexibility requirements, which are already highly deployed through electrolysers, hydrogen storage and (partly) Fischer-Tropsch processes.

% \begin{figure*}[h!] % Use 'figure*' to span both columns
%     \centering
%     \begin{subfigure}[b]{0.49\linewidth}
%         \centering
%         \includegraphics[width=\linewidth]{graphics/integrated_comp/contour_mg_AC_exclu_H2 El_all_20_filterTrue.pdf}
%         \caption{Average price of electricity for local consumers}
%         \label{fig:local_el_price}
%     \end{subfigure}
%     \hfill
%     \begin{subfigure}[b]{0.49\linewidth}
%         \centering
%         \includegraphics[width=\linewidth]{graphics/integrated_comp/contour_mg_H2_False_False_exportonly_20_filterTrue.pdf}
%         \caption{Average price of hydrogen for export}
%         \label{fig:export_hy_price}
%     \end{subfigure}
%     \hfill
%     \caption{Marginal prices of electricity and hydrogen subject to export volumes and emission limits depending on various weightings. Black lines indicate the lowest price at each emission limit. Note: 0 export cut-off}
%     \label{fig:marginal_prices}
% \end{figure*}

% \subsection{Hydrogen export prices}
% \label{subsec:results_hy_prices}

% The average price of hydrogen for export (s. Fig. \ref{fig:export_hy_price}) is weighted based on the hydrogen demand at the export nodes. 
% The price of hydrogen highly depends on the operation of electrolysers (CAPEX share) and the cost of electricity required for the process (OPEX share). The CAPEX share of hydrogen costs (\ref{fig:capex-rel}) is indirectly proportional to the capacity factor of electrolysis (\ref{fig:ely-cf}). A high capacity factor leads to a low share of CAPEX in the total hydrogen costs. The OPEX share of hydrogen costs (\ref{fig:opex-rel}) is directly proportional to the price of electricity (\ref{fig:electricity-price}), the cost of water or operation-dependent maintenance is neglectable. % Sources, e.g.  Hampp2023

% Combining the CAPEX and OPEX share leads to the total hydrogen costs displayed in Figure (\ref{fig:hydrogen-cost}). The cost of hydrogen differs to the price of hydrogen determined as marginal prices (\ref{fig:hydrogen-price}). At emission reductions $<40$\%, the hydrogen cost (in particular the OPEX) is very low due to the access to cheap electricity across the whole hydrogen export dimension. In contrast, the hydrogen (marginal) price, defined as additional cost per extra unit of hydrogen, depends on the hydrogen export volume also at low emission reductions. The reason for this is the green hydrogen constraint (s. Sec. \ref{subsec:green_hydrogen_constraint}), which requires the additional installation of renewable energies per extra unit of hydrogen accordingly as shown in Section \ref{subsec:increase_h2}. This leads to in general higher hydrogen prices (including additional REs) than hydrogen costs (not including additional REs). 

% The difference between hydrogen costs and hydrogen prices is more pronounced at low climate ambitions, since coal power supplies cheap electricity (s. Fig. \ref{fig:coal-supply}) resulting in low OPEX for electrolysis, whereas the hydrogen price includes the financing of additional renewable energies. At high climate ambitions, the trends of costs and prices are very similar: in both cases, the absence of carbon-intensive electricity generation units requires additional renewable energy installations, regardless of taking the green hydrogen into account (hydrogen price) or not (hydrogen cost).

% % Increasing the climate ambition
% At low export volumes of 20 TWh and carbon reductions $<40$\%, low prices are observable. In these scenarios, the hydrogen demand is comparably low (low hydrogen export, no Fischer-Tropsch demand), hence the electrolysers can run at a high capacity factor above 90\% and utilize their capital-intensive investments resulting in a low contribition of CAPEX costs to the price of hydrogen as shown in Figure \ref{fig:capex-rel}. Increasing the climate ambition (still at low export volumes) goes along with increasing the renewable energy installations, decreasing the capacity factor of electrolysis and ulimatively increasing the hydrogen price as pointed out in \cite{Ruggles2021}.

% % Going up the hydrogen export
% The same reason (additional RE installations, decreasing electrolyser capacity factors, increasing marginal cost of hydrogen) is also valid for increasing the hydrogen export volume. Nonetheless, Figure \ref{fig:export_hy_price} shows different nuances of price increases when increasing hydrogen export volumes. In the range of 0 - 50\% emission reduction, the gradient of price increase is rather high, symbolised by crossing narrow contour lines. At higher emission reductions (80 - 100\%), the contour lines are more distant, meaning the effect of hydrogen prices increasing with the hydrogen export volume is
% weakened. This is due to the capacity factor of the electrolysers: At high climate ambitions they already run at low capacity factors, an additional increase of REs (required for hydrogen export) does not lower the CF significantly (from 50\% to 40\%). Instead, increasing the hydrogen export at lower climate ambitions is linked to a decrease of capacity factor from around 90\% to less then 60\%  (s. Fig. \ref{fig:ely-cf}) resulting in a steeper increase of hydrogen prices at lower export volumes.





% % Flattening out
% At low emission reductions and high hydrogen exports, the effect of increasing hydrogen prices flattens out. The energy system is dominated by the hydrogen export infrastructure. A further increase of hydrogen exports 
% would push the system towards it's wind onshore and solar PV potential limits, forced to use less favourable renewable energy sites (lower capacity factors). % TODO get proof for this by simulating high export scenarios

% \begin{itemize}
%     \item Increasing electricity prices up to $60$--$70$\%, then decreasing electricity prices 
%     \item When hydrogen export is above 1.5 fold of local electricity demand, the price decrease flattens out
%     \item Synergy in top right corner at high reduction and high export: lower prices here 
%     \item Above 70\%: local hydrogen production ramps up as well. Until then: no benefit/synergy. Theory: Battery kicking in and smoothing el prices?
% \end{itemize}



% Weighting explaination:
% \begin{enumerate}
%     \item electricity consumption on each node ($\rightarrow$ electricity prices everyone sees: "Export-Local-Average"),
%     \item hydrogen consumption on each node ($\rightarrow$ hydrogen prices everyone sees: "Export-Local-Average"),
%     \item electricity demand of electrolysers on each node ($\rightarrow$ electricity prices which electrolysers see: "Export-orientated") and
%     \item electricity demand of all consumers except electrolysers ($\rightarrow$ electricity prices which local consumers see: "Local-orientated").
% \end{enumerate}


% \subsection{Electricity prices for local consumers and eletrolysers}
% \label{subsec:economic_benefits}
% Derived from the Figure \ref{fig:marginal_prices}, Figure \ref{fig:pricecut} disregards the export dimension and shows the weighted electricity prices for local consumers, electrolysers and average prices at an export volume of 100~TWh. It shows that at low emission reductions ($0$--$10$\%), both the prices for local consumers and electrolysers rise from $40$~\euro/MWh to approx. $50$~\euro/MWh. The higher the emission reduction from $10$\% onwards, the more both prices diverge. At high climate ambitions ($100$\%), the electricity price for local consumers is at $78$~\euro/MWh; $47$\% higher than the electricity price for electrolysers ($53$~\euro/MWh).

% The observed difference in weighted prices results from the operation of the energy system: The (non-weighted) price of electricity is cheaper when the electrolysers are running and more expensive when local industry and households are consuming electricity.
% % \begin{itemize}
% %     \item TBD: Why does this effect amplify at higher climate ambition?
% %     \item TBD: Reason could be: at 100 TWh hydrogen export, the export demand dominates hence from a total system cost it makes sense to favour electrolysers. BUT: at lower export volumes, this effect is even more pronounced.
% %     \item Visualize: AC prices heatmap, electrolyser heatmap, local demand heatmap. $\rightarrow$ the higher the climate ambition, the higher the correlation between AC prices and electrolysers should be
% % \end{itemize}

% \begin{figure}[h!]
%     \centering
%     \includegraphics[width=\linewidth]{graphics/pricecuts/electricity100.pdf}
%     \caption{Weighted electricity prices for local consumers, electrolysers and average at 100 TWh export volume}
%     \label{fig:pricecut}
% \end{figure}

% TBD: sensitivity analysis
% \subsection{Sensitivity analysis}
% \label{subsec:sensitivity}
% Figure XX shows the sensitivity to exogenous parameters (WACC, technology cost, temporal resolution) (TBD).

% No battery storage:
% \begin{itemize}
%     \item Slightly more hydrogen storage
%     \item Lower capacity factor of electrolysers at low $\mathrm{CO_2}$ reductions
%     \item More curtailment at low $\mathrm{CO_2}$ reductions and high export (massive wind curtailment $\rightarrow$ Due to green hydrogen constraint? Why is there so much excess wind?)
% \end{itemize}

% Sensitivity analysis on:
% \begin{itemize}
%     \item Weighted Average Cost of Capital (WACC)
%     \item Technology cost
%     \item Hourly resolution
%     \item Battery storage vs no battery storage
% \end{itemize}