\subsection{Hydrogen export cost in 2030 and 2050}

\begin{figure}[h!]
    \centering
    \includegraphics[width=\linewidth]{miro_cost}
    \caption{Two different scenarios, assuming integrated and islanded hydrogen production.}
    \label{fig:results_costs}
\end{figure}


\subsection{Capacity factors and utilization of assets}

\begin{figure}[h!]
    \centering
    \includegraphics[width=\linewidth]{miro_capacity}
    \caption{Two different scenarios, assuming integrated and islanded hydrogen production.}
    \label{fig:results_capacity}
\end{figure}

\subsection{Spatial analysis of (competing) RE resources}
Both the islanded systems and the Moroccan energy system have the same land resources available. 
The optimal solutions for both optimisations rely on the same land resources, in combination exceeding the 
available resources in some regions (s. Fig. \ref{fig:land_conflicts}). 


\begin{figure}[h!]
    \centering
    \includegraphics[width=\linewidth]{land_conflicts}
    \caption{Land conflicts arising from separated optimisations}
    \label{fig:land_conflicts}
\end{figure}


\subsection{Carbon intensity}
Bullet points
\begin{itemize}
    \item What is the carbon intensity for export and local consumption of the islanded scenario? 
    \item What is the carbon intensity of the integrated scenario?
    \item Compare with blue hydrogen (1 kgCO2/kgH2) and EU threshold for low-carbon gas (3 kgCO2/kgH2)
    \item Differnce between 2030 and 2050?
\end{itemize}



\subsection{Sensitivity analysis}
\subsection{Further ideas}
List of possible results and aspects worth looking at
\begin{itemize}
    \item 
    \item 
\end{itemize}