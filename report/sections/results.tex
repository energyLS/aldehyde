In this section we focus on the synergies and conflicts
of hydrogen export and national energy transition.
First, the results of the separated/islanded and integrated optimizations
are compared in \ref{subsec: sep_int_opti} in regard to the total system cost and
hydrogen export costs and competing RE/land resources.
Second, Section \ref{subsec: syn_conf_integrated} focuses on the integrated optimisation 
depending on the ambitions in national energy transition and hydrogen export

\subsection{Separated and integrated optimizations}
\label{subsec: sep_int_opti}

\subsubsection{Total system cost}
Figure \ref{fig:int_sep_total_system_cost} displays the total system cost
of the integrated vs. islanded optimizations. The total system cost of the
islanded optimizations are higher.

\begin{figure}[h!]
    \centering
    \includegraphics[width=\linewidth]{graphics/system_comp/total_system_cost_140.pdf}
    \caption{Total cost of the three different systems}
    \label{fig:int_sep_total_system_cost}
\end{figure}

\subsubsection{Hydrogen export cost (in 2030 and 2050)}
The cost of hydrogen is lower in the islanded scenario due to best potentials and no transport included?

\subsubsection{Spatial analysis of (competing) RE resources}
Both the islanded systems and the Moroccan energy system have the same land resources available. 
The optimal solutions for both optimisations rely on the same land resources, in combination exceeding the 
available resources in some regions (s. Fig. \ref{fig:land_conflicts}). 
Similar to Europe/Germany, public acceptance of Renewable Energies have to be taken in account. 
They are not unlimited, as discussed in \cite{Hanger2016} and \cite{TerraponPfaff2019}.
Paper on competing land resources can be found here \cite{Patankar2022}


\subsubsection{Capacity factors and utilization of assets}
(Maybe not interesting/necessary)



\subsection{Synergies and conflicts of the integrated optimization}
\label{subsec: syn_conf_integrated}
Figure \ref{fig:contour_plot} shows the synergy maximizing time path.

\begin{figure}[h!]
    \centering
    \includegraphics[width=\linewidth]{graphics/integrated_comp/contour_plot_lcoh_levels10.pdf}
    \caption{Synergies and conflicts between the two scenarios looking at the levelized cost of hydrogen}
    \label{fig:contour_plot}
\end{figure}

\begin{figure}[h!]
    \centering
    \includegraphics[width=\linewidth]{graphics/integrated_comp/contour_plot_totalsystemcost_levels100.pdf}
    \caption{Synergies and conflicts between the two scenarios looking at the total system cost}
    \label{fig:contour_plot_totalsystem}
\end{figure}



\subsection{Sensitivity analysis}
Sensitivity analysis on:
\begin{itemize}
    \item WACC
    \item Hydrogen Export in TWh
    \item Emission reduction level
\end{itemize}


\subsection{Further ideas}
List of possible results and aspects worth looking at
\begin{itemize}
    \item Carbon intensity
        \item What is the carbon intensity for export and local consumption of the islanded scenario? 
        \item What is the carbon intensity of the integrated scenario?
        \item Compare with blue hydrogen (1 kgCO2/kgH2) and EU threshold for low-carbon gas (3 kgCO2/kgH2)
        \item Differnce between 2030 and 2050?
\end{itemize}


\subsection{First results and insights}

First results and insights are:
\begin{itemize}
    \item Cost of hydrogen with SMR: 40 €/MWh, Cost of hydrogen without SMR 75 €/MWh
    \item The export focuses on port in "MAR.2.1{\_}1"
\end{itemize}


