\subsection{Hydrogen export cost (in 2030 and 2050)}

\begin{figure}[h!]
    \centering
    \includegraphics[width=\linewidth]{../graphics_general/miro_cost}
    \caption{Two different scenarios, assuming integrated and islanded hydrogen production.}
    \label{fig:results_costs}
\end{figure}

\begin{figure}[h!]
    \centering
    \includegraphics[width=\linewidth]{../graphics_general/first_results/LCOH.pdf}
    \caption{Levelised cost of hydrogen}
    \label{fig:LCOH}
\end{figure}

As depicted in Fig. \ref{fig:LCOH}, the cost of hydrogen is lower in the islanded scenario due to best potentials and no transport included (export 10 TWh)?


\subsection{Total system cost}

\begin{figure}[h!]
    \centering
    \includegraphics[width=\linewidth]{../graphics_general/first_results/total_system_cost.pdf}
    \caption{Total cost of the three different systems}
    \label{fig:total_system_cost}
\end{figure}

\subsection{Capacity factors and utilization of assets}

\begin{figure}[h!]
    \centering
    \includegraphics[width=\linewidth]{../graphics_general/miro_capacity}
    \caption{Two different scenarios, assuming integrated and islanded hydrogen production.}
    \label{fig:results_capacity}
\end{figure}

\subsection{Spatial analysis of (competing) RE resources}
Both the islanded systems and the Moroccan energy system have the same land resources available. 
The optimal solutions for both optimisations rely on the same land resources, in combination exceeding the 
available resources in some regions (s. Fig. \ref{fig:land_conflicts}). 
Similar to Europe/Germany, public acceptance of Renewable Energies have to be taken in account. 
They are not unlimited, as discussed in \cite{Hanger2016} and \cite{TerraponPfaff2019}.
Paper on competing land resources can be found here \cite{Patankar2022}

\begin{figure}[h!]
    \centering
    \includegraphics[width=\linewidth]{../graphics_general/land_conflicts}
    \caption{Land conflicts arising from separated optimisations}
    \label{fig:land_conflicts}
\end{figure}




\subsection{Synergies and conflicts}
Figure \ref{fig:contour_plot} shows the synergy maximizing time path.

\begin{figure}[h!]
    \centering
    \includegraphics[width=\linewidth]{../graphics_general/first_results/contour_plot.pdf}
    \caption{Synergies and conflicts between the two scenarios}
    \label{fig:contour_plot}
\end{figure}

\subsection{Carbon intensity}
Bullet points
\begin{itemize}
    \item What is the carbon intensity for export and local consumption of the islanded scenario? 
    \item What is the carbon intensity of the integrated scenario?
    \item Compare with blue hydrogen (1 kgCO2/kgH2) and EU threshold for low-carbon gas (3 kgCO2/kgH2)
    \item Differnce between 2030 and 2050?
\end{itemize}



\subsection{Sensitivity analysis}
Sensitivity analysis on:
\begin{itemize}
    \item WACC
    \item Hydrogen Export in TWh
    \item Emission reduction level
\end{itemize}


\subsection{Further ideas}
List of possible results and aspects worth looking at
\begin{itemize}
    \item 
    \item 
\end{itemize}


\subsection{First results and insights}

First results and insights are:
\begin{itemize}
    \item Cost of hydrogen with SMR: 40 €/MWh, Cost of hydrogen without SMR 75 €/MWh
    \item The export focuses on port in "MAR.2.1{\_}1"
\end{itemize}


