
\begin{figure*}[h!]
    \centering
    \begin{subfigure}[b]{0.49\linewidth}
        \centering
        \includegraphics[trim={0cm 0cm 0cm 1cm}, clip, width=\linewidth]{../../workflow/subworkflows/pypsa-earth-sec/results/\runstandard/0exp-only/120/graphs/balances-AC.pdf}
        \caption{Fixed (1 TWh) export and 0--100\% domestic climate change mitigation}
        \label{fig:balances-ac-0exp-120}
    \end{subfigure}
    \hfill
    \begin{subfigure}[b]{0.49\linewidth}
        \centering
        \includegraphics[trim={0cm 0cm 0cm 1cm}, clip, width=\linewidth]{../../workflow/subworkflows/pypsa-earth-sec/results/\runstandard/co2l20-only/120/graphs/balances-AC.pdf}
        \caption{Fixed (0\%) domestic climate change mitigation and 1--120 TWh/a export}
        \label{fig:balances-ac-co2l20-120}
    \end{subfigure}
    \hfill
    \caption{Electricity supply and demand at fixed export levels and increasing domestic climate change mitigation (\ref{fig:balances-ac-0exp-120}) and vice versa (\ref{fig:balances-ac-co2l20-120}). Increasing domestic climate change mitigation first phases out carbon-intensive coal generation in favor of CCGT, at medium to high domestic climate change mitigation the electricity system is fully renewable supported by flexibility through Vehicle-to-Grid (V2G) and sector coupling. Increasing electricity demands include Battery Electric Vehicles (BEV) and hydrogen generation for other sectors.
    At increasing hydrogen exports the additional electricity required for hydrogen electrolysis is covered by onshore wind and solar PV, as imposed by the temporal hydrogen regulation. 
    }
    \label{fig:balances-ac}
\end{figure*}


\subsection*{Domestic climate change mitigation increases the domestic electricity demand}
\label{subsec:increase_limit}
First of all, we investigate the supply and demand of Morocco's electricity system depending on domestic climate change mitigation (dimension 1). Figure \ref{fig:balances-ac-0exp-120} depicts the supply and demand of the electricity system at increasing climate mitigation ambitions from 0\% to 100\%. At low climate ambitions, the electricity demand is mainly covered by coal power, existing (brownfield) capacities of onshore wind, hydro, and combined-cycle gas turbines (CCGT) play only a minor role. 
At increasing domestic climate change mitigation, coal power is phased out in favour of solar PV, furthermore a fuel switch from coal to gas (CCGT) is observable at increasing emission reductions. At medium climate mitigation ambitions, the dipatchable power in the electricity system is provided by CCGT. 
Further increasing the climate ambitions, wind onshore and especially solar PV penetrate and dominate the electricity system complemented by dispatchable power from Vehicle-to-Grid providing an almost fully renewable electricity sector at 100\% emission reduction. 
While mitigation targets can (in part) be met with technologies like fossil gas with CCS or Direct Air Capture, the model finds that increasing renewable capacities, particularly solar PV and wind, is the most cost-effective solution. As mitigation efforts intensify, renewables become the cost-optimal choice for decarbonizing the energy system due to their cost-competitiveness and large-scale availability.

At 70\% and above, the electricity demand of eletrolysers increases substantially to supply hydrogen allowing a switch from fossil oil products to Fischer-Tropsch fuels in various sectors (s. Fig. \ref{fig:oil-balance}). 
The Fischer-Tropsch demand in the transport sector increases, even though the increasing Battery Electric Vehicle (BEV) diffusion is counterbalancing the demand for oil products (see \nameref{sec:si} section \nameref{subsec:bev_diffusion}). 
Both electrolysers and BEVs drive the electricity demand substantially, up to two times of the domestic end-use electricity demand. 
Additionally, the electricity demand for air heat pumps increases, supporting the defossilisation in the heating sector and supplying heat for Direct Air Capture required for synthetic fuels.


\subsection*{Hydrogen exports require a multiple of the domestic electricity demand}
\label{subsec:increase_h2}

Here, we analyse the hydrogen export ramp up (dimension 2). The electricity supply at increasing hydrogen exports is displayed in Figure \ref{fig:balances-ac-co2l20-120}. The deployment of hydrogen exports requires a scale up of onshore wind and solar PV accordingly, as defined by the green hydrogen constraint outlined in \nameref{subsec:green_hydrogen_constraint}.
Without the temporal hydrogen regulation, the additional electricity demand of hydrogen exports is covered by coal power plants and CCGT (by increasing the capacity factor of existing brownfield coal and CCGT capacities) and an expansion of open-cycle gas turbines (OCGT) installation and supply as pointed out in Figure \ref{fig:barplotscons}.
The cost-optimal solution of the model shows, that increasing hydrogen exports do not necessarily require additional renewable capacities. Instead, the electricity demand for hydrogen electrolysis is covered by additional electricity supply from fossil generation. However, if the cost-optimal solution is constrained by the green hydrogen policy, an increase of hydrogen exports goes along with additional renewable electricity capacites as defined in the EU's delegated Act.
In contrast to the exponential increase of electricity demand observable at increasing climate mitigation ambitions displayed in Figure \ref{fig:balances-ac-0exp-120}, the electricity demand increases linearly with the hydrogen export ambitions.


\begin{figure*}[h!]
    \centering
    \begin{subfigure}[b]{0.49\linewidth}
        \centering
        \includegraphics[width=\linewidth]{graphics/integrated_comp/contour_exp_AC_exclu_H2 El_all_20_filterTrue_nTrue_exp120-PATH.pdf}
        \caption{Relative cost of electricity for domestic customers (normalized to 1 TWh/a hydrogen export)}
        \label{fig:expense_ac_120}
    \end{subfigure}
    \hfill
    \begin{subfigure}[b]{0.49\linewidth}
        \centering
        \includegraphics[width=\linewidth]{graphics/integrated_comp/contour_exp_H2_False_False_exportonly_20_filterTrue_nTrue_exp120-PATH.pdf}
        \caption{Relative cost of hydrogen for exporters (normalized to 0\% domestic climate change mitigation)}
        \label{fig:expense_h2_120}
    \end{subfigure}
    \hfill
    \caption{  
    Cost for domestic electricity consumers (\ref{fig:expense_ac_120}) and hydrogen exporters (\ref{fig:expense_h2_120}),
    normalized to costs at 1 TWh/a hydrogen export (\ref{fig:expense_ac_120}) and
    to 0\% \co reduction (\ref{fig:expense_h2_120})
    at each domestic climate change mitigation level. Domestic electricity consumers profit from increasing hydrogen exports, especially at low domestic climate change mitigation and high exports. Hydrogen exporters profit from domestic climate change mitigation at medium mitigation efforts. Both (\ref{fig:expense_ac_120}) and (\ref{fig:expense_h2_120}) include possible pathways of i) quick exports and slow climate change mitigation, ii) balanced exports and mitigation and iii) slow exports and quick climate change mitigation. Years are illustrative.}
    \label{fig:expenses_default_120}
\end{figure*}


\subsection*{Domestic climate change mitigation and hydrogen export show co-benefits}
\label{subsec:benefits}


In an integrated analysis of domestic climate change mitigation (dimension 1) and hydrogen exports (dimension 2) this study outlines that both domestic climate change mitigation and hydrogen exports profit from each other and show co-benefits. In this Section, we i) show the effects of hydrogen exports on domestic electricity prices, ii) the role of domestic climate change mitigation on hydrogen export cost, and lastly iii) common co-benefits.

First, Figure \ref{fig:expense_ac_120} shows the cost for domestic electricity consumers subject to mitigation and hydrogen export volumes. The costs are normalized to costs at 1 TWh/a Hydrogen export at each domestic climate change mitigation level, displaying relative changes for domestic electricity consumers induced by hydrogen exports at a certain domestic climate change mitigation.
Hydrogen exports decrease the relative domestic electricity cost by up to 45\%, especially at mitigation below 40\% and exports above 50 TWh. 
In these ranges, the domestic electricity consumers profit from excess green electricity originating from additional renewable energy capacities required for hydrogen export.
The more the domestic electricity system is decarbonized (domestic climate change mitigation above 50--60\%), the weaker is the decrease of domestic electricity prices with hydrogen exports.
The only increase of cost induced by hydrogen exports is observable at a low climate change mitigation of 10\% and exports of 20 TWh. 

Second, Figure \ref{fig:expense_h2_120} shows the cost of hydrogen exports subject to mitigation and hydrogen export volumes. The costs are normalized to costs at 0\% climate change mitigation at each hydrogen export volume, displaying relative changes for hydrogen exports from domestic climate change mitigation at a certain hydrogen export volume.
At hydrogen export volumes of 25--75 TWh, an increase of domestic climate change mitigation up to 40--60\% decreases the hydrogen export prices up to -9\%. 
The only price increase observable in the range of our scenarios is observable at low export volumes below 20 TWh. Here, advances in domestic climate change mitigation increase the hydrogen export cost by up to 19\% compared to no domestic climate change mitigation.
This cost increase results from lower electrolysis capacity factors compared to the 0\% climate change mitigation scenario, where we observe high electrolysis capacity factors of above 80\% (s. Fig. \ref{fig:ely-cf-hourly}).
The triangle in the top left area shows that cost for hydrogen exporters at certain export levels are independent of domestic climate change mitigation, this is driven by temporal hydrogen regulation further investigated in section \nameref{subsec:benefits_rule}. Here, hydrogen export infrastructure decouples from the domestic electricity system, the cost is increasingly independent of domestic climate change mitigation. The hydrogen exports are 2 times higher than the domestic electricity demand (s. Fig. \ref{fig:expense_h2_120}), dwarfing the importance of the domestic electricity system.


Third, we derive co-benefits for domestic climate change mitigation and hydrogen exports. 
Within the area of 40-60\% climate change mitigation and 50-100 TWh/a hydrogen exports, we see i) domestic electricity consumers profit from hydrogen exports compared to very low (1 TWh) exports and ii) hydrogen exporters decrease their cost compared to no (0\%) domestic climate change mitigation. Hence, both domestic climate change mitigation and hydrogen exports show clear co-benefits at medium mitigation (40--60\%) and moderate to high hydrogen exports (25--120 TWh). 


Apart from the cost of electricity for domestic customers and cost of hydrogen for exporters, Figure \ref{fig:expenses_default_120} presents possible pathways of climate change mitigation and export. We have derived three illustrative mitigation-export pathways presenting various speeds of transformation:
\begin{enumerate}
    \item \textbf{Quick exports and slow climate change mitigation},
    \item \textbf{Balanced exports and climate change mitigation} and
    \item \textbf{Slow exports and quick climate change mitigation}.
\end{enumerate}

In the long run, domestic electricity consumers profit in all pathways from hydrogen exports. A slight and short increase of domestic electricity prices in the early transformation phase (2030) is compensated by later (2050) strong (pathway 1) and medium (pathway 2) benefits for domestic electricity customers, profiting from hydrogen exports.
In pathway 3, the short-term price increases are only marginal, but the long-term benefit from decreased is not as strong as in the scenarios with quicker export scale-ups. Figure \ref{fig:expense_h2_120} shows that hydrogen exporters have clear benefits in pathway 2, the cost of hydrogen for exports decrease compared to no domestic climate change mitigation. By ramping up hydrogen exports in an early stage of domestic climate change mitigation, high relative cost of hydrogen can be avoided.


\begin{figure}[h!]
    \centering
    \includegraphics[trim={0cm 0cm 0cm 0.65cm}, clip, width=\linewidth]{../../results/gh/price_sorted_AC_120_2.0_Hours.pdf}
    \caption{Price duration curve at 120 TWh/a export and 0\% domestic climate change mitigation. Stricter temporal hydrogen regulation pushes the price duration curve towards the left, as additional renewable electricity capacities phase out fossil generation with higher short-term marginal costs than renewables. Negative prices below -50 €/MWh are cut off.}
    \label{fig:pdc-120-0}
\end{figure}



\begin{figure*}[h!]
    \centering
    \begin{subfigure}[b]{0.49\linewidth}
        \centering
        \includegraphics[trim={0cm 0cm 0cm 0.65cm}, clip, width=\linewidth]{../../results/gh/dispatch_120_Co2L2.0.pdf}
        \caption{Electricity dispatch and demand}
        \label{fig:dispatch_rule}
    \end{subfigure}
    \hfill
    \begin{subfigure}[b]{0.49\linewidth}
        \centering
        \includegraphics[trim={0cm 0cm 0cm 0.65cm}, clip, width=\linewidth]{../../results/gh/abs_sin_expenses_h2ac_120_Co2L2.0.pdf}
        \caption{Cost for domestic electricity consumers and hydrogen exporters}
        \label{fig:expense_h2ac}
    \end{subfigure}
    \hfill
    \caption{Electricity dispatch and demand (\ref{fig:dispatch_rule}) and cost for consumers (\ref{fig:expense_h2ac}) for various (hydrogen) temporal matching regimes in the 120 TWh/a export and 0\% climate change mitigation scenario. Stricter temporal matching decreases carbon-intensive electricity generation (coal \& gas) for hydrogen generation and even domestic electricity consumers (s. Fig. \ref{fig:dispatch_rule}). Cost for export hydrogen generation increase to fulfill the temporal matching constraint, whereas domestic electricity consumers profit from stricter hydrogen regulation.}
    \label{fig:expenses_rule}
\end{figure*}


\subsection*{Hydrogen regulation reduces domestic electricity prices and emissions}
\label{subsec:benefits_rule}

Hydrogen regulation has strong effects on electricity system dispatch (and emissions), increasing the prices and cost for hydrogen exporters whilst decreasing for domestic electricity consumers. This section shows the mechanisms of temporal matching in the 120 TWh/a export and 0\% domestic climate change mitigation scenario.

As the temporal matching becomes stricter, additional renewable capacities and complementing hydrogen storage are required to meet the electricity demand of electrolysis
on a time base defined by the temporal matching constraint, as shown in Figure \ref{fig:tsc-120-0}. These renewable capacities i.) push out fossil generation with high short-run marginal costs (\textit{price spillover}) and ii.) provide excess electricity to the domestic electricity system (\textit{energy spillover}). Both spillover effects are linked, the energy spillover causes a merit order effect affecting the domestic electricity prices.

The \textit{price spillover} is observable in Figure \ref{fig:pdc-120-0} depicting the price duration curve at 120 TWh/a export and 0\% domestic climate change mitigation. The price duration curve describes the sorted price of electricity (spatial mean) at every hour of the year.
Stricter hydrogen regulation pushes the price duration curve towards the left, as additional renewable electricity capacities phase out fossil generation, which has higher short-run marginal costs due to fuel expenses, compared to renewable electricity.
As renewables generate power when available, they reduce the dispatch of fossil generators, whose higher marginal costs make them less competitive in the short term. However, fossil fuels still play a complementary role, stepping in to fill gaps during periods of low renewable generation. This mechanism is fully captured in our capacity expansion and operation model, which accounts for both types of generation in response to price changes and regulatory shifts.


The \textit{energy spillover} is shown in Figure \ref{fig:dispatch_rule}. In the \textit{no rule} scenario, which does not constrain the electricity input of hydrogen electrolysis, the electricity supply of coal and OCGT is substantially higher than in the annual, monthly, or hourly case. If annual matching is applied, the renewable electricity supply to the electricity system equals the annual demand for electrolysis, increasing the supply of solar PV by 20 TWh/a while decreasing the supply of fossil electricity generation. 
Monthly matching increases this trend, while hourly matching completely phases out OCGT and CCGT. Figure \ref{fig:dispatch_rule} highlights the \textit{energy spillover} effect: going beyond the electricity demand of hydrogen electrolysis, renewable capacities installations necessary to meet the temporal matching decarbonize the domestic electricity demand and decrease the grid emissions. This effect is steered by the strictness of temporal matching, hence stricter temporal matching rules enable a further decrease in emissions.
The slight increase in total dispatch for annual and monthly matching is balanced by resistive heaters in the heating sector.
In addition to the synergies between hydrogen export and domestic climate change mitigation, the \textit{energy spillover} outlines a clear synergy between domestic climate change mitigation and temporal hydrogen regulation.



The \textit{price spillover} effects results in lower electricity prices for domestic consumers. Taking the electricity demand into account, the decrease of prices results in a decrease of electricity cost shown in 
Figure \ref{fig:expense_h2ac}. Stricter temporal hydrogen regulation decreases the domestic electricity cost from 2.7 B€ to 1.2 B€. In contrast, the hydrogen cost for exporters increases from 10.1 B€ to 10.6 B€, carrying the cost of the temporal matching constraint and hence financing additional solar PV and hydrogen storage required to meet the electrolysis electricity demand on a certain temporal basis.

The combined cost of domestic electricity and export hydrogen decreases with stricter temporal hydrogen regulation, whereas the total system cost increase as displayed in Figure \ref{fig:tsc-120-0}. This seeming inconsistency is based on the fact, that electricity consumers (domestic and hydrogen electrolyser input) pay less contribution margin to the cost of capital of coal power plants, since they get phased out with stricter temporal hydrogen regulation. On the other hand, the cost of capital of existing coal power plants is not part of the optimization-based total system cost, since they are a brownfield capacity and not extended. In short, the reduction of utilisation of the brownfield coal power assets decreases the cost for electricity consumers, whereas the additional renewable capacities required to meet the temporal hydrogen regulation drive the total system cost as shown in Figure \ref{fig:tsc-120-0}. 


To sum up, temporal hydrogen regulation is a policy instrument acting in a twofold way. It steers the redistribution of costs between hydrogen exports and domestic electricity consumers (based on the \textit{price spillover} effect), decreasing the cost for domestic electricity consumers and increasing the cost for hydrogen exporters. 
Second, the temporal hydrogen regulation 
not only decarbonizes hydrogen exports, but additionally decreases grid emissions of the domestic electricity demand via the \textit{energy spillover} effect. Both effects scale with stricter temporal matching.



\subsection*{The effects of temporal hydrogen regulation are strongest in high export and low climate change mitigation systems}
\label{subsec:rule_all}

Based on the effects of temporal hydrogen regulation in the scenario of 120 TWh/a export and 0\% climate change mitigation presented in the previous section, this section explores the temporal hydrogen regulation across the pathway ii.) \textit{Balanced exports and mitigation} by combining all three dimensions of domestic climate change mitigation, hydrogen export and temporal hydrogen regulation. Figure \ref{fig:expenses_real_120} shows the relative change of electricity and hydrogen cost of the \textit{balanced exports and mitigation} scenarios in dependence of temporal matching. 
Across all displayed scenarios, temporal hydrogen regulation hedges domestic electricity consumers against rising prices. Depending on the domestic climate change mitigation and hydrogen export, hourly matching decreases the cost of electricity for domestic consumers by up to 31\%.

The scenarios of the \textit{balanced exports and mitigation} pathway in Figure \ref{fig:expenses_real_120} show that in only three mitigation-export combinations the annual or monthly matching decreases the cost for domestic electricity, whereas all other combinations of the \textit{Balanced exports and mitigation} pathway are only sensitive to the strictest -- hourly matching -- hydrogen regulation. 

In only two scenarios, hourly matching regulation increases the hydrogen cost above 7\% compared to no temporal hydrogen regulation. This is the case for 0\% climate change mitigation and 1 and 20 TWh/a export scenarios.

The effects of temporal hydrogen regulation on domestic electricity prices are most dominant in high export and low climate change mitigation scenarios. 
In these cases, the introduction of annual matching has already a striking effect, because the \textit{price spillover} observed in Figure \ref{fig:pdc-120-0} decreases (domestic) electricity prices. The \textit{price spillover} unfolds in fossil-dominated (hence low domestic climate change mitigation) systems, combined with high hydrogen exports providing a strong \textit{energy spillover} effect.
See also Figure \ref{fig:expenses_all_200} for  scenarios following the \textit{quick exports and slow climate change mitigation} pathway in which the effect of temporal hydrogen regulation is even stronger.

In contrast, in the \textit{slow exports and quick climate change mitigation} scenarios only the hourly matching has observable effects, since the electricity system is already  dominated by renewable electricity and only hourly matching phases out the remaining fossil generation (s. Fig. \ref{fig:expenses_all_200}). The differentiation among the domestic climate change mitigation leads to these staged effects of temporal hydrogen regulation on domestic electricity cost.


\begin{figure}[h!]
    \centering
    \includegraphics[trim={0cm 0cm 0cm 0.65cm}, clip, width=\linewidth]{../../results/gh/rel_multiple_120exp_Co2L2.0dec_onlyreal.pdf}
    \caption{Relative change of electricity and hydrogen cost and total system cost depending on the temporal hydrogen regulation. Domestic electricity consumers profit across all export and mitigation scenarios but most at high export and low climate change mitigation. Hydrogen exporters experience higher cost with stricter temporal hydrogen regulation. The temporal hydrogen regulation regulates the welfare distribution between both groups.}
    \label{fig:expenses_real_120}
\end{figure}
