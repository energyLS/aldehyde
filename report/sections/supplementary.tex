\subsection{Flexibility options}
Battery and hydrogen storage as well as curtailment.

\begin{figure*}[t] % Use 'figure*' to span both columns
    \centering
    \begin{subfigure}[b]{0.45\linewidth}
        \centering
        \includegraphics[width=\linewidth]{graphics/integrated_comp/contour_curtailmentrate_solar_20_filterFalse.pdf}
        \caption{Solar PV curtailment}
        \label{fig:solar_curt}
    \end{subfigure}
    \hfill
    \begin{subfigure}[b]{0.45\linewidth}
        \centering
        \includegraphics[width=\linewidth]{graphics/integrated_comp/contour_curtailmentrate_wind_20_filterFalse.pdf}
        \caption{Wind curtailment}
        \label{fig:wind_curt}
    \end{subfigure}
    \hfill
    \begin{subfigure}[b]{0.45\linewidth}
        \centering
        \includegraphics[width=\linewidth]{graphics/integrated_comp/contour_Battery_GWh_20_filterFalse.pdf}
        \caption{Battery storage capacity}
        \label{fig:battery_cap}
    \end{subfigure}
    \hfill
    \begin{subfigure}[b]{0.45\linewidth}
        \centering
        \includegraphics[width=\linewidth]{graphics/integrated_comp/contour_H2_GWh_20_filterFalse.pdf}
        \caption{Hydrogen storage capacity}
        \label{fig:hystorage_cap}
    \end{subfigure}
    \hfill

    \caption{Marginal prices of electricity and hydrogen subject to export volumes and emission limits depending on various weightings. Black lines indicate the lowest price at each emission limit.}
    \label{fig:integration_options}
\end{figure*}

Figure \ref{fig:barplots} displays the energy balances of modelled scenarios.


\begin{figure}[h!]
    \centering
    \begin{subfigure}[b]{\linewidth}
        \centering
        \includegraphics[trim={0 6.5cm 0 0}, clip, width=\linewidth]{../../workflow/subworkflows/pypsa-earth-sec/results/nresults_full_3H_ws/graphs/balances-AC.pdf}
        \caption{Balances AC}
        \label{fig:balances_AC}
    \end{subfigure}
    
    \vspace{0.5cm} % Add vertical space between the subfigures
    
    \begin{subfigure}[b]{\linewidth}
        \centering
        \includegraphics[trim={0 6.5cm 0 0}, clip, width=\linewidth]{../../workflow/subworkflows/pypsa-earth-sec/results/nresults_full_3H_ws/graphs/balances-energy.pdf}
        \caption{Balances energy}
        \label{fig:balances_energy}
    \end{subfigure}

    \vspace{0.5cm} % Add vertical space between the subfigures
    
    \begin{subfigure}[b]{\linewidth}
        \centering
        \includegraphics[trim={0 6.5cm 0 0}, clip, width=\linewidth]{../../workflow/subworkflows/pypsa-earth-sec/results/nresults_full_3H_ws/graphs/costs.pdf}
        \caption{costs}
        \label{fig:balances_costs}
    \end{subfigure}
    
    \caption{Barplots}
    \label{fig:barplots}
\end{figure}



Figure \ref{fig:land_conflicts} displays the land conflicts arising from separated optimisations.

\begin{figure}[h!]
    \centering
    \includegraphics[width=\linewidth]{../graphics_general/land_conflicts}
    \caption{Land conflicts arising from separated optimisations}
    \label{fig:land_conflicts}
\end{figure}

% Add figure first_results/Export-only-cf-onwind.pdf here

Figure \ref{fig:cf-onwind} displays the capacity factor of onshore wind.

\begin{figure}[h!]
    \centering
    \includegraphics[width=\linewidth]{../graphics_general/first_results/Export-only-cf-onwind.pdf}
    \caption{Capacity factor of onshore wind}
    \label{fig:cf-onwind}
\end{figure}

% Add figure first_results/Export-only-cf-solar.pdf here

Figure \ref{fig:cf-solar} displays the capacity factor of solar.

\begin{figure}[h!]
    \centering
    \includegraphics[width=\linewidth]{../graphics_general/first_results/Export-only-cf-solar.pdf}
    \caption{capacity factor of solar PV}
    \label{fig:cf-solar}
\end{figure}

% Add figure first_results/Export-only-high-p-nom-max-onwind.pdf

Figure \ref{fig:p-nom-max-onwind} displays the potential of onshore wind.

\begin{figure}[h!]
    \centering
    \includegraphics[width=\linewidth]{../graphics_general/first_results/Export-only-p-nom-max-onwind.pdf}
    \caption{Potential of onshore wind}
    \label{fig:p-nom-max-onwind}
\end{figure}

% Add figure first_results/Export-only-high-p-nom-max-solar.pdf here

Figure \ref{fig:p-nom-max-solar} displays the potential of solar PV.

\begin{figure}[h!]
    \centering
    \includegraphics[width=\linewidth]{../graphics_general/first_results/Export-only-p-nom-max-solar.pdf}
    \caption{Potential of solar PV}
    \label{fig:p-nom-max-solar}
\end{figure}


Figure \ref{fig:land_use_solar} displays the land use of solar PV in three different scenarios.

\begin{figure}[h]
    \centering
    \begin{subfigure}[b]{0.3\textwidth}
        \includegraphics[width=\textwidth]{../fix_co2/graphics/spatial/export_only_land-use_solar_20export.pdf}
        \caption{Low export}
        \label{fig:graphic1}
    \end{subfigure}
    \begin{subfigure}[b]{0.3\textwidth}
        \includegraphics[width=\textwidth]{../fix_co2/graphics/spatial/export_only_land-use_solar_140export.pdf}
        \caption{High export}
        \label{fig:graphic2}
    \end{subfigure}
    \begin{subfigure}[b]{0.3\textwidth}
        \includegraphics[width=\textwidth]{../fix_co2/graphics/spatial/mar_es_land-use_solar_140export.pdf}
        \caption{MAR-ES + export}
        \label{fig:graphic3}
    \end{subfigure}
    \caption{(a) Land use in low export scenario, (b) Land use in high export scenario, (c) Land use in MAR-ES + export scenario. Certain regions in
    (b) are fully exploited, same as in (c). This indicates that the high export scenario and the MAR-ES scenario have competing land resource demands.}
    \label{fig:land_use_solar}
\end{figure}
    
Figure \ref{fig:morocco_water} displays the water supply costs in Morocco.

\begin{figure}[h!]
    \centering
    \includegraphics[width=\linewidth]{../graphics_general/water_supply_costs.png}
    \caption{Morocco water supply costs}
    \label{fig:morocco_water}
\end{figure}


\begin{figure}[h!]
    \centering
    \includegraphics[width=\linewidth]{../graphics_general/graphical_abstract.pdf}
    \caption{Graphical abstract}
    \label{fig:graphical_abstract}
\end{figure}

\subsection{Separated and integrated optimizations}
\label{subsec: sep_int_opti}

Note: \cite{Zeyen2022} has information on PyPSA electricity prices

\subsubsection{Total system cost}
Figure \ref{fig:int_sep_total_system_cost} displays the total system cost
of the integrated vs. islanded optimizations. The total system cost of the
islanded optimizations are higher.

\begin{figure}[h!]
    \centering
    \includegraphics[width=\linewidth]{../fix_co2/graphics/system_comp/total_system_cost_140.pdf}
    \caption{Total cost of the three different systems}
    \label{fig:int_sep_total_system_cost}
\end{figure}

\subsection{Misc}

\subsubsection{Hydrogen export cost (in 2030 and 2050)}
The cost of hydrogen is lower in the islanded scenario due to best potentials and no transport included?

\subsubsection{Spatial analysis of (competing) RE resources}
Both the islanded systems and the Moroccan energy system have the same land resources available. 
The optimal solutions for both optimisations rely on the same land resources, in combination exceeding the 
available resources in some regions (s. Fig. \ref{fig:land_conflicts}). 
Similar to Europe/Germany, public acceptance of Renewable Energies have to be taken in account. 
They are not unlimited, as discussed in \cite{Hanger2016} and \cite{TerraponPfaff2019}.
Paper on competing land resources can be found here \cite{Patankar2022}


\subsubsection{Capacity factors and utilization of assets}
(Maybe not interesting/necessary)


\subsubsection{RE Potentials}
Figure \ref{fig:potentials} displays the potentials of solar, wind onshore and CSP 
of PyPSA-Earth-Sec compared to other studies. The authors of "OSeMOSYS" 
\cite{Cannone2021} base their potential estimations on \cite[primary: 13,18,19]{Cannone2021}
The project MenaFuels/Ersoy2022 \cite{Ersoy2022} finds potentials of ...
TBD: remove ChatGPT which was added for fun.

\begin{figure}[h!]
    \centering
    \includegraphics[width=\linewidth]{../graphics_general/potentials_comparison/barplot_potentials.pdf}
    \caption{Determined potential of solar, wind onshore and CSP in PyPSA-Earth-Sec compared to other studies.}
    \label{fig:potentials}
\end{figure}