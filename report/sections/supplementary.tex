\subsection{Integration challenges}


Figure \ref{fig:diurnal_exportdim} displays the deployment of battery and hydrogen storage as well as curtailment.

\begin{figure*}[t] % Use 'figure*' to span both columns
    \centering
    \begin{subfigure}[b]{0.3\linewidth}
        \centering
        \includegraphics[width=\linewidth]{graphics/rldc/rlc_diurnal_m0.90.pdf}
        \caption{10\% emission reduction}
        \label{fig:diurnal_10emred}
    \end{subfigure}
    \hfill
    \begin{subfigure}[b]{0.3\linewidth}
        \centering
        \includegraphics[width=\linewidth]{graphics/rldc/rlc_diurnal_m0.50.pdf}
        \caption{50\% emission reduction}
        \label{fig:diurnal_50emred}
    \end{subfigure}
    \hfill
    \begin{subfigure}[b]{0.3\linewidth}
        \centering
        \includegraphics[width=\linewidth]{graphics/rldc/rlc_diurnal_m0.10.pdf}
        \caption{90\% emission reduction}
        \label{fig:diurnal_90emred}
    \end{subfigure}
    \hfill
    \caption{Export dependent (0 - 200 TWh) residual load curves with various integration options. Each row represents different integration strategies (RLC without integration, RLC with curtailment, RLC with curtailment and electrolyser load).
    Each column represents different climate ambitions (10 - 90\% emission reduction)}
    \label{fig:diurnal_exportdim}
\end{figure*}

\subsection{Diffusion of Battery Electric Vehicles}

\begin{figure*}[t]
    \centering
    \includegraphics[width=0.7\linewidth]{../graphics_general/policy/bev_share.pdf}
    \caption{Market diffusion of Battery Electricity Vehicles in Morocco, synthesized based on a s-curve with a growth rate $k=0.2$ and inflection point $x_0=2040$.}
    \label{fig:bev_diffusion}
\end{figure*}

\subsection{Operation of electrolysers and Fischer-Tropsch}


%heatmaps/elec_s_10_ec_lc1.0_Co2L0.10_3H_2030_0.15_DF_40export/cf-ts-AC.pdf
\clearpage
\onecolumn
\begin{figure}
    \centering
        \begin{subfigure}[t]{0.33\textwidth}
            \centering
        \includegraphics[width=\textwidth]{graphics/heatmaps/elec_s_10_ec_lc1.0_Co2L0.80_3H_2030_0.15_DF_40export/cf-ts-AC.pdf}
    \end{subfigure}
    \begin{subfigure}[t]{0.33\textwidth}
        \centering
        \includegraphics[width=\textwidth]{graphics/heatmaps/elec_s_10_ec_lc1.0_Co2L0.30_3H_2030_0.15_DF_40export/cf-ts-AC.pdf}
    \end{subfigure}
    \begin{subfigure}[t]{0.33\textwidth}
        \centering
        \includegraphics[width=\textwidth]{graphics/heatmaps/elec_s_10_ec_lc1.0_Co2L0.10_3H_2030_0.15_DF_40export/cf-ts-AC.pdf}
    \end{subfigure}


    \begin{subfigure}[t]{0.33\textwidth}
        \centering
        \includegraphics[width=\textwidth]{graphics/heatmaps/elec_s_10_ec_lc1.0_Co2L0.80_3H_2030_0.15_DF_40export/cf-ts-H2 Electrolysis.pdf}
    \end{subfigure}
    \begin{subfigure}[t]{0.33\textwidth}
        \centering
        \includegraphics[width=\textwidth]{graphics/heatmaps/elec_s_10_ec_lc1.0_Co2L0.30_3H_2030_0.15_DF_40export/cf-ts-H2 Electrolysis.pdf}
    \end{subfigure}
    \begin{subfigure}[t]{0.33\textwidth}
        \centering
        \includegraphics[width=\textwidth]{graphics/heatmaps/elec_s_10_ec_lc1.0_Co2L0.10_3H_2030_0.15_DF_40export/cf-ts-H2 Electrolysis.pdf}
    \end{subfigure}

    \begin{subfigure}[t]{0.33\textwidth}
        \centering
        \includegraphics[width=\textwidth]{graphics/heatmaps/elec_s_10_ec_lc1.0_Co2L0.80_3H_2030_0.15_DF_40export/cf-ts-H2.pdf}
    \end{subfigure}
    \begin{subfigure}[t]{0.33\textwidth}
        \centering
        \includegraphics[width=\textwidth]{graphics/heatmaps/elec_s_10_ec_lc1.0_Co2L0.30_3H_2030_0.15_DF_40export/cf-ts-H2.pdf}
    \end{subfigure}
    \begin{subfigure}[t]{0.33\textwidth}
        \centering
        \includegraphics[width=\textwidth]{graphics/heatmaps/elec_s_10_ec_lc1.0_Co2L0.10_3H_2030_0.15_DF_40export/cf-ts-H2.pdf}
        
    \end{subfigure}


    \begin{subfigure}[t]{0.33\textwidth}
        \centering
        \includegraphics[width=\textwidth]{graphics/heatmaps/elec_s_10_ec_lc1.0_Co2L0.80_3H_2030_0.15_DF_40export/cf-ts-Fischer-Tropsch.pdf}
        \caption{20\% emission reduction}
        \label{fig:operation20}
    \end{subfigure}
    \begin{subfigure}[t]{0.33\textwidth}
        \centering
        \includegraphics[width=\textwidth]{graphics/heatmaps/elec_s_10_ec_lc1.0_Co2L0.30_3H_2030_0.15_DF_40export/cf-ts-Fischer-Tropsch.pdf}
        \caption{70\% emission reduction}
        \label{fig:operation70}
    \end{subfigure}
    \begin{subfigure}[t]{0.33\textwidth}
        \centering
        \includegraphics[width=\textwidth]{graphics/heatmaps/elec_s_10_ec_lc1.0_Co2L0.10_3H_2030_0.15_DF_40export/cf-ts-Fischer-Tropsch.pdf}
        \caption{90\% emission reduction}
        \label{fig:operation90}
    \end{subfigure}

    \caption{Operation of electrolysers/Fischer-Tropsch and prices of electricity and hydrogen at 40 TWh export. The electrolyser operation depends on the electricity prices, strong diurnal electricity price patterns as in \ref{fig:operation70} result in a corresponding electrolyser operation. Once batteries kick in at \ref{fig:operation90} ($\rightarrow$ proof? See also \ref{fig:battery_operation} for battery operation), resulting smoother electricity prices also increase the annual capacity factor of the electrolyser. Note: electricity and hydrogen prices are spatially averaged in non-weighted manner, style adapted from cf. \cite{Neumann2022}}
    \label{fig:ren-cfs}
\end{figure}

\clearpage
\twocolumn


\subsection{Operation of batteries}


%heatmaps/elec_s_10_ec_lc1.0_Co2L0.10_3H_2030_0.15_DF_40export/cf-ts-AC.pdf
\clearpage
\onecolumn
\begin{figure}
    \centering
        \begin{subfigure}[t]{0.33\textwidth}
            \centering
        \includegraphics[width=\textwidth]{graphics/heatmaps/elec_s_10_ec_lc1.0_Co2L0.80_3H_2030_0.15_DF_40export/cf-ts-battery charger.pdf}
    \end{subfigure}
    \begin{subfigure}[t]{0.33\textwidth}
        \centering
        \includegraphics[width=\textwidth]{graphics/heatmaps/elec_s_10_ec_lc1.0_Co2L0.30_3H_2030_0.15_DF_40export/cf-ts-battery charger.pdf}
    \end{subfigure}
    \begin{subfigure}[t]{0.33\textwidth}
        \centering
        \includegraphics[width=\textwidth]{graphics/heatmaps/elec_s_10_ec_lc1.0_Co2L0.10_3H_2030_0.15_DF_40export/cf-ts-battery charger.pdf}
    \end{subfigure}


    \begin{subfigure}[t]{0.33\textwidth}
        \centering
        \includegraphics[width=\textwidth]{graphics/heatmaps/elec_s_10_ec_lc1.0_Co2L0.80_3H_2030_0.15_DF_40export/cf-ts-battery discharger.pdf}
    \end{subfigure}
    \begin{subfigure}[t]{0.33\textwidth}
        \centering
        \includegraphics[width=\textwidth]{graphics/heatmaps/elec_s_10_ec_lc1.0_Co2L0.30_3H_2030_0.15_DF_40export/cf-ts-battery discharger.pdf}
    \end{subfigure}
    \begin{subfigure}[t]{0.33\textwidth}
        \centering
        \includegraphics[width=\textwidth]{graphics/heatmaps/elec_s_10_ec_lc1.0_Co2L0.10_3H_2030_0.15_DF_40export/cf-ts-battery discharger.pdf}
        
    \end{subfigure}


    \begin{subfigure}[t]{0.33\textwidth}
        \centering
        \includegraphics[width=\textwidth]{graphics/heatmaps/elec_s_10_ec_lc1.0_Co2L0.80_3H_2030_0.15_DF_40export/cf-ts-battery.pdf}
        \caption{20\% emission reduction}
        \label{fig:operation20}
    \end{subfigure}
    \begin{subfigure}[t]{0.33\textwidth}
        \centering
        \includegraphics[width=\textwidth]{graphics/heatmaps/elec_s_10_ec_lc1.0_Co2L0.30_3H_2030_0.15_DF_40export/cf-ts-battery.pdf}
        \caption{70\% emission reduction}
        \label{fig:operation70}
    \end{subfigure}
    \begin{subfigure}[t]{0.33\textwidth}
        \centering
        \includegraphics[width=\textwidth]{graphics/heatmaps/elec_s_10_ec_lc1.0_Co2L0.10_3H_2030_0.15_DF_40export/cf-ts-battery.pdf}
        \caption{90\% emission reduction}
        \label{fig:operation90}
    \end{subfigure}

    \caption{Operation of batteries at 40 TWh export. EV batteries are not included here, but do not play a significant role ($\rightarrow$ TODO doublecheck), style adapted from cf. \cite{Neumann2022}}
    \label{fig:battery_operation}
\end{figure}

\clearpage
\twocolumn



\subsection{Flexibility options}
Figure \ref{fig:integration_options} displays the deployment of battery and hydrogen storage as well as curtailment.

\begin{figure*}[t] % Use 'figure*' to span both columns
    \centering
    \begin{subfigure}[b]{0.45\linewidth}
        \centering
        \includegraphics[width=\linewidth]{graphics/integrated_comp/contour_curtailmentrate_solar_20_filterFalse.pdf}
        \caption{Solar PV curtailment}
        \label{fig:solar_curt}
    \end{subfigure}
    \hfill
    \begin{subfigure}[b]{0.45\linewidth}
        \centering
        \includegraphics[width=\linewidth]{graphics/integrated_comp/contour_curtailmentrate_wind_20_filterFalse.pdf}
        \caption{Wind curtailment}
        \label{fig:wind_curt}
    \end{subfigure}
    \hfill
    \begin{subfigure}[b]{0.45\linewidth}
        \centering
        \includegraphics[width=\linewidth]{graphics/integrated_comp/contour_Battery_GWh_20_filterFalse.pdf}
        \caption{Battery storage capacity}
        \label{fig:battery_cap}
    \end{subfigure}
    \hfill
    \begin{subfigure}[b]{0.45\linewidth}
        \centering
        \includegraphics[width=\linewidth]{graphics/integrated_comp/contour_H2_GWh_20_filterFalse.pdf}
        \caption{Hydrogen storage capacity}
        \label{fig:hystorage_cap}
    \end{subfigure}
    \hfill

    \caption{Curtailment rates and storage capacities}
    \label{fig:integration_options}
\end{figure*}

Figure \ref{fig:barplots} displays the energy balances of modelled scenarios.


\begin{figure}[h!]
    \centering
    \begin{subfigure}[b]{\linewidth}
        \centering
        \includegraphics[trim={0 6.5cm 0 0}, clip, width=\linewidth]{../../workflow/subworkflows/pypsa-earth-sec/results/nresults_full_3H_ws/graphs/balances-AC.pdf}
        \caption{Balances AC}
        \label{fig:balances_AC}
    \end{subfigure}
    
    \vspace{0.5cm} % Add vertical space between the subfigures
    
    \begin{subfigure}[b]{\linewidth}
        \centering
        \includegraphics[trim={0 6.5cm 0 0}, clip, width=\linewidth]{../../workflow/subworkflows/pypsa-earth-sec/results/nresults_full_3H_ws/graphs/balances-energy.pdf}
        \caption{Balances energy}
        \label{fig:balances_energy}
    \end{subfigure}

    \vspace{0.5cm} % Add vertical space between the subfigures
    
    \begin{subfigure}[b]{\linewidth}
        \centering
        \includegraphics[trim={0 6.5cm 0 0}, clip, width=\linewidth]{../../workflow/subworkflows/pypsa-earth-sec/results/nresults_full_3H_ws/graphs/costs.pdf}
        \caption{costs}
        \label{fig:balances_costs}
    \end{subfigure}
    
    \caption{Electricity balance, energy balance and costs of the modelled scenarios.}
    \label{fig:barplots}
\end{figure}



% Figure \ref{fig:land_conflicts} displays the land conflicts arising from separated optimisations.

% \begin{figure}[h!]
%     \centering
%     \includegraphics[width=\linewidth]{../graphics_general/land_conflicts}
%     \caption{Land conflicts arising from separated optimisations}
%     \label{fig:land_conflicts}
% \end{figure}

% Add figure first_results/Export-only-cf-onwind.pdf here

% Figure \ref{fig:cf-onwind} displays the capacity factor of onshore wind.

% \begin{figure}[h!]
%     \centering
%     \includegraphics[width=\linewidth]{../graphics_general/first_results/Export-only-cf-onwind.pdf}
%     \caption{Capacity factor of onshore wind}
%     \label{fig:cf-onwind}
% \end{figure}

% Add figure first_results/Export-only-cf-solar.pdf here

% Figure \ref{fig:cf-solar} displays the capacity factor of solar.

% \begin{figure}[h!]
%     \centering
%     \includegraphics[width=\linewidth]{../graphics_general/first_results/Export-only-cf-solar.pdf}
%     \caption{capacity factor of solar PV}
%     \label{fig:cf-solar}
% \end{figure}

% Add figure first_results/Export-only-high-p-nom-max-onwind.pdf

% Figure \ref{fig:p-nom-max-onwind} displays the potential of onshore wind.

% \begin{figure}[h!]
%     \centering
%     \includegraphics[width=\linewidth]{../graphics_general/first_results/Export-only-p-nom-max-onwind.pdf}
%     \caption{Potential of onshore wind}
%     \label{fig:p-nom-max-onwind}
% \end{figure}

% % Add figure first_results/Export-only-high-p-nom-max-solar.pdf here

% Figure \ref{fig:p-nom-max-solar} displays the potential of solar PV.

% \begin{figure}[h!]
%     \centering
%     \includegraphics[width=\linewidth]{../graphics_general/first_results/Export-only-p-nom-max-solar.pdf}
%     \caption{Potential of solar PV}
%     \label{fig:p-nom-max-solar}
% \end{figure}


% Figure \ref{fig:land_use_solar} displays the land use of solar PV in three different scenarios.

% \begin{figure}[h]
%     \centering
%     \begin{subfigure}[b]{0.3\textwidth}
%         \includegraphics[width=\textwidth]{../fix_co2/graphics/spatial/export_only_land-use_solar_20export.pdf}
%         \caption{Low export}
%         \label{fig:graphic1}
%     \end{subfigure}
%     \begin{subfigure}[b]{0.3\textwidth}
%         \includegraphics[width=\textwidth]{../fix_co2/graphics/spatial/export_only_land-use_solar_140export.pdf}
%         \caption{High export}
%         \label{fig:graphic2}
%     \end{subfigure}
%     \begin{subfigure}[b]{0.3\textwidth}
%         \includegraphics[width=\textwidth]{../fix_co2/graphics/spatial/mar_es_land-use_solar_140export.pdf}
%         \caption{MAR-ES + export}
%         \label{fig:graphic3}
%     \end{subfigure}
%     \caption{(a) Land use in low export scenario, (b) Land use in high export scenario, (c) Land use in MAR-ES + export scenario. Certain regions in
%     (b) are fully exploited, same as in (c). This indicates that the high export scenario and the MAR-ES scenario have competing land resource demands.}
%     \label{fig:land_use_solar}
% \end{figure}
    
% Figure \ref{fig:morocco_water} displays the water supply costs in Morocco.

% \begin{figure}[h!]
%     \centering
%     \includegraphics[width=\linewidth]{../graphics_general/water_supply_costs.png}
%     \caption{Morocco water supply costs}
%     \label{fig:morocco_water}
% \end{figure}


% \begin{figure}[h!]
%     \centering
%     \includegraphics[width=\linewidth]{../graphics_general/graphical_abstract.pdf}
%     \caption{Graphical abstract}
%     \label{fig:graphical_abstract}
% \end{figure}

% \subsection{Separated and integrated optimizations}
% \label{subsec: sep_int_opti}


% \subsubsection{Total system cost}
% Figure \ref{fig:int_sep_total_system_cost} displays the total system cost
% of the integrated vs. islanded optimizations. The total system cost of the
% islanded optimizations are higher.

% \begin{figure}[h!]
%     \centering
%     \includegraphics[width=\linewidth]{../fix_co2/graphics/system_comp/total_system_cost_140.pdf}
%     \caption{Total cost of the three different systems}
%     \label{fig:int_sep_total_system_cost}
% \end{figure}

% \subsection{Misc}

% \subsubsection{Hydrogen export cost (in 2030 and 2050)}
% The cost of hydrogen is lower in the islanded scenario due to best potentials and no transport included?

% \subsubsection{Spatial analysis of (competing) RE resources}
% Both the islanded systems and the Moroccan energy system have the same land resources available. 
% The optimal solutions for both optimisations rely on the same land resources, in combination exceeding the 
% available resources in some regions (s. Fig. \ref{fig:land_conflicts}). 
% Similar to Europe/Germany, public acceptance of Renewable Energies have to be taken in account. 
% They are not unlimited, as discussed in \cite{Hanger2016} and \cite{Terrapon-Pfaff2019}.
% Paper on competing land resources can be found here \cite{Patankar2022}


% \subsubsection{Capacity factors and utilization of assets}
% (Maybe not interesting/necessary)


% \subsubsection{RE Potentials}
% Figure \ref{fig:potentials} displays the potentials of solar, wind onshore and CSP 
% of PyPSA-Earth-Sec compared to other studies. The authors of "OSeMOSYS" 
% \cite{Cannone2021} base their potential estimations on \cite[primary: 13,18,19]{Cannone2021}
% The project MenaFuels/Ersoy2022 \cite{Ersoy2022} finds potentials of ...
% TBD: remove ChatGPT which was added for fun.

% \begin{figure}[h!]
%     \centering
%     \includegraphics[width=\linewidth]{../graphics_general/potentials_comparison/barplot_potentials.pdf}
%     \caption{Determined potential of solar, wind onshore and CSP in PyPSA-Earth-Sec compared to other studies.}
%     \label{fig:potentials}
% \end{figure}


\subsection{Moroccos conventional power plants}
Moroccos conventional power plants are displayed in Table \ref{tab:powerplants}.

% Force one column page
\clearpage
\onecolumn
\begin{footnotesize}
    \begin{table}
\centering
\caption{Conventional power plants in Morocco}
\label{tab:powerplants}
\begin{tabular}{p{1cm}p{6.5cm}rrp{2cm}p{3.4cm}}
\toprule
{} &                                         Name &  Capacity in MW &  Fuel type &  Commission &  Decommission \\
\midrule
6  &                     Jorf Lasfar Jlec Morocco &          1250.2 &  Hard Coal &        1994 &          2039 \\
21 &                      Mohammedia Ocgt Morocco &           925.4 &        Oil &        1981 &          2021 \\
26 &                                     Al Wahda &           800.0 &       CCGT &        2010 &          2050 \\
3  &                         Step Afourer Morocco &           514.5 &      Hydro &        1951 &          2051 \\
25 &  Ain Beni Mathar Centrale Thermosolaire Ccgt &           472.0 &       CCGT &        2009 &          2049 \\
29 &                                      Step Ur &           464.0 &      Hydro &        2005 &          2105 \\
2  &                       Tahaddart Ccgt Morocco &           345.0 &       OCGT &        2005 &          2045 \\
28 &                                   Mohammedia &           300.0 &  Hard Coal &        1986 &          2031 \\
5  &                              Kenitra Morocco &           277.9 &        Oil &        1978 &          2018 \\
12 &                       Allal El Fassi Morocco &           236.9 &      Hydro &        1994 &          2094 \\
13 &                         Al Wahda Dam Morocco &           236.9 &      Hydro &        1997 &          2097 \\
16 &                      Tit Mellil Ocgt Morocco &           189.0 &        Oil &        1993 &          2033 \\
7  &                          Jerada Coal Morocco &           152.1 &  Hard Coal &        1971 &          2016 \\
22 &             Noor Ouarzazate Csp Osps Morocco &           145.8 &      Other &        2016 &          2021 \\
1  &                         Tetouan Ocgr Morocco &           134.3 &        Oil &        1975 &          2015 \\
11 &          Bin El Ouidane Hyroelectric Morocco &           133.3 &      Hydro &        1953 &          2053 \\
14 &                           Al Massira Morocco &           126.4 &      Hydro &        1980 &          2080 \\
27 &                    Centrale Diesel De Tantan &           116.9 &        Oil &        2009 &          2049 \\
24 &                             Tan Ocgt Morocco &            94.5 &        Oil &        1992 &          2032 \\
0  &                     Ahmed El Hansali Morocco &            90.8 &      Hydro &        2003 &          2103 \\
9  &                                       Hassan &            66.3 &      Hydro &        1991 &          2091 \\
8  &                               Idriss Morocco &            39.5 &      Hydro &        1978 &          2078 \\
23 &                    Oued El Makhazine Morocco &            35.5 &      Hydro &        1979 &          2079 \\
17 &                               Imfout Morocco &            31.6 &      Hydro &        1947 &          2047 \\
20 &                              Mohamed Morocco &            22.7 &      Hydro &        1967 &          2067 \\
10 &                              Daourat Morocco &            16.8 &      Hydro &        1950 &          2050 \\
4  &              Sidi Said Maachou Hydro Morocco &            15.4 &      Hydro &        1929 &          2029 \\
19 &                     Mansour Ed Dahbi Morocco &             9.9 &      Hydro &        1973 &          2073 \\
18 &                     Lalla Takerkoust Morocco &             8.7 &      Hydro &        1938 &          2038 \\
15 &                           Al Kensara Morocco &             8.2 &      Hydro &        1946 &          2046 \\
\bottomrule
\end{tabular}
\end{table}

\end{footnotesize}
\clearpage
\twocolumn