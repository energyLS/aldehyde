Morocco is pursuing an ambitious energy transition to reduce greenhouse gas emissions and increase the share of renewable energy sources 
in its energy mix. At the same time, the country is exploring opportunities to become a leading exporter of green hydrogen. 
This research paper investigates the synergies and conflicts between these two objectives, taking into account the potential
impacts of hydrogen exports on Morocco's national energy transition, as well as land use and neocolonialism. To achieve this goal, a 
high spatial and temporal resolution PyPSA-Earth model is employed to simulate and optimize the Moroccan energy system under multiple scenarios,
sweeping through different hydrogen export volumes and climate targets. The model results reveal that hydrogen exports could provide significant
economic benefits for Morocco, but also introduce potential conflicts with the country's renewable energy deployment, particularly in terms of 
competing demands for land use. Additionally, the paper discusses the potential risks of neocolonialism in the context of hydrogen exports and 
emphasizes the need for equitable and sustainable development. The research concludes with policy recommendations to maximize the synergies and 
minimize the potential conflicts between hydrogen exports and the Moroccan energy transition, while ensuring social and environmental sustainability.