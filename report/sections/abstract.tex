Several countries are emerging as future potential exporters of green hydrogen, while simultaneously pursuing ambitious energy transitions to reduce greenhouse gas emissions. 
Morocco serves as a blueprint for countries aiming to design their energy systems along these two dimensions, risking technology lock-in, unabated local energy-related greenhouse gas emissions, rising local energy prices, land use implications and neo-colonialism. 
Extending the current state of research, we conduct an integrated analysis of synergies and conflicts between these two objectives, taking into account the potential impact of hydrogen exports on economic benefits for both local populations and hydrogen exporters. We present a fully sector-coupled capacity expansion and dispatch model of Morocco across 53 regions, including integrated gas pipeline and electricity network planning based on PyPSA-Earth. The applied sector-coupled model simulates and optimises the Moroccan energy system in more than 100 scenarios at a 3-hourly temporal resolution, sweeping through different hydrogen export volumes and climate targets. 
Results show that hydrogen exports could provide significant economic benefits for both hydrogen exporters and the local population, but also create potential conflicts and rising electricity and hydrogen prices.
In addition, potential risks of neo-colonialism highlight the need for an integrated assessment of hydrogen exports and local decarbonization. Integrated policies unlock synergies and minimise potential conflicts between hydrogen exports and Morocco's energy transition, while ensuring social and environmental sustainability.