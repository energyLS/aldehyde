% Cover these questions:
% What is your paper about?
% Why is it important?
% How did you do it?
% What did you find?
% Why are your findings important?

% Structure
% 1. background or context: Briefly explain the problem or gap in knowledge that your research addresses. 
% 2. Objectives/Hypothesis: Clearly state the research question or objective.
% 3. methods: Provide a brief overview of the methods used in your study. Mention the study design, participants, materials, and procedures. Be concise but include enough information for readers to understand the study's approach.
% 4. results: Summarize the key findings of your study. Use quantitative information when possible, such as statistical outcomes or numerical data. Highlight the most important results that address your research question.
% 5. conclusion: Clearly state the main conclusions drawn from your study. Discuss the implications of your findings and their potential significance. Avoid making broad statements not supported by your results.

% Comments
% It would be good to have a research question or paradox or clear tension/puzzle emerge from the first 1-3 sentences

% Journal advice

% The summary is a single paragraph no longer than 150 words. An effective summary includes the following elements: 
% (1) a brief background of the question that avoids statements about how a process is not well understood; 
% (2) a description of the results and approaches/model systems framed in the context of their conceptual interest; and 
% (3) an indication of the broader significance of the work. We discourage novelty claims (e.g., use of the word “novel”) because they are overused, tend not to add meaning, and are difficult to verify. 
% Please do not include references in the summary. 

% New abstract (short)
Global green hydrogen demand surges with the need of sustainable energy systems, putting future hydrogen exporters in the spotlight. However, the installation of on-grid hydrogen electrolysis for export, reaching magnitudes multiple times of the domestic electricity demand, introduces the potential for profound impacts on both domestic energy prices, climate mitigation and energy-related emissions. Our investigation explores the dimensions of hydrogen exports, domestic mitigation and hydrogen regulation, employing a sector-coupled energy model for Morocco. The main results underscore the co-benefits of medium mitigation (40-60\%) and moderate to high hydrogen exports (25-150 TWh). Notably, temporal matching as policy instrument is not only cuts emissions but reduces the domestic electricity expenses by up to -65\% while elevating export costs by up to 7\% in scenarios marked by high exports and low mitigation. Beyond Morocco, these integrated policies unlock synergies, potentially increase local acceptance by reducing domestic electricity prices and counteract neo-colonialism implications associated with hydrogen exports.

% % New abstract (long)

% % Introduction/Background
% Several countries are emerging as future potential exporters of green hydrogen, while simultaneously pursuing their own ambitious energy transitions to reduce greenhouse gas emissions.
% The adoption of on-grid hydrogen electrolysis for export, reaching magnitudes multiple times of the domestic electricity demand, introduces the potential for profound impacts on both domestic electricity prices and energy balances.
% Vice versa, domestic mitigation and highly renewable electricity systems provide economic opportunities for hydrogen exporters.
% % Methods
% We explore the interactions between hydrogen exports, domestic mitigation and hydrogen regulation using a fully sector-coupled capacity expansion and dispatch model of Morocco across 14 regions, including integrated gas pipeline and electricity network planning based on PyPSA-Earth. The applied sector-coupled model simulates and optimises the Moroccan energy system in 484 scenarios at a 3-hourly temporal resolution, sweeping through different hydrogen export volumes, domestic  mitigation levels and hydrogen regulations.
% % Results
% Our results show co-benefits of local mitigation and hydrogen export at medium mitigation (40 - 60\%) and moderate to high hydrogen exports (25 -- 150 TWh).
% In addition, we reveal hydrogen regulation via temporal matching as decisive instrument to steer economic redistributions between hydrogen exporters and domestic electricity consumers. 
% Hourly matching decreases the expenses for domestic electricity (up to -65\%) and increases expenses for hydrogen exports (up to 7\%) in high export and low mitigation scenarios.
% Our analysis across all three dimensions (export, mitigation, hydrogen regulation) shows the importance of strong hydrogen regulation in high export and low mitigation scenarios beyond the specific case of Morocco. 
% The redistribution effects of hydrogen regulation are less pronounced in countries with a high share of renewable electricity.
% % Conclusion & Broader Significance
% We reveal how hydrogen regulation can effectively hedge domestic electricity consumers against rising prices across mitigation and export scenarios.  Integrated policies unlock synergies, potentially increase local acceptance by reducing domestic electricity prices and counteract neo-colonialism implications of hydrogen exports.



% Old abstract
% Several countries are emerging as future potential exporters of green hydrogen, while simultaneously pursuing their own ambitious energy transitions to reduce greenhouse gas emissions. 
% Morocco serves as a blueprint for countries aiming to design their energy systems along these two dimensions, risking technology lock-in, unabated local energy-related greenhouse gas emissions, rising local energy prices, land use implications and neo-colonialism. 
% Extending the current state of research, we conduct an integrated analysis of synergies and conflicts between these two objectives, taking into account the potential impact of hydrogen exports on economic benefits for both local populations and hydrogen exporters. We present a fully sector-coupled capacity expansion and dispatch model of Morocco across 14 regions, including integrated gas pipeline and electricity network planning based on PyPSA-Earth. The applied sector-coupled model simulates and optimises the Moroccan energy system in more than 100 scenarios at a 3-hourly temporal resolution, sweeping through different hydrogen export volumes and climate targets. 
% Results show that hydrogen exports could provide significant economic benefits for both hydrogen exporters and the local population, but also create potential conflicts and rising electricity and hydrogen prices.
% In addition, potential risks of neo-colonialism highlight the need for an integrated assessment of hydrogen exports and local decarbonization. Integrated policies unlock synergies and minimise potential conflicts between hydrogen exports and Morocco's energy transition, while ensuring social and environmental sustainability.