
As global demand for green hydrogen rises, potential hydrogen exporters move into the spotlight.
However, the large-scale installation of on-grid hydrogen electrolysis for export can have profound impacts on domestic energy prices and energy-related emissions.
Our investigation explores the interplay of hydrogen exports, domestic energy transition and temporal hydrogen regulation, employing a sector-coupled energy model in Morocco. 
We find substantial co-benefits of domestic climate change mitigation and hydrogen exports, whereby exports can reduce domestic electricity prices while mitigation reduces hydrogen export prices.
However, increasing hydrogen exports quickly in a system that is still dominated by fossil fuels can substantially raise domestic electricity prices, if green hydrogen production is not regulated.
Surprisingly, temporal matching of hydrogen production lowers domestic electricity cost by up to 31\% while the effect on exporters is minimal. 
This policy instrument can steer the welfare (re-)distribution between hydrogen exporting firms, hydrogen importers, and domestic electricity consumers and hereby increases acceptance among actors.