\subsection*{Conclusion}


Our study shows the importance of integrated scenario analysis combining hydrogen exports, domestic climate change mitigation and temporal hydrogen regulation.

Both hydrogen exports and domestic climate change mitigation benefit from each other mainly at emissions reduction of 40--60\% compared to the baseline scenario and moderate to high hydrogen exports (25--120 TWh/a).
However, we show that there are also risks with respect to rising domestic electricity prices
and find that hydrogen regulation via temporal matching is a decisive instrument
to protect domestic consumers. Hourly matching decreases the cost for domestic electricity consumers (up to -31\%) while the effect on hydrogen exports is minimal. Temporal hydrogen regulation can effectively hedge domestic electricity consumers against rising prices across mitigation and export scenarios.

This study contributes to the investigation of implications and benefits of hydrogen exports for domestic electricity consumers. Even though temporal hydrogen regulation hedges domestic electricity consumers against rising prices, further implications of hydrogen exporters on the domestic population (land use, competing RE resources, environmental concerns of desalination) are not within the scope of this study.

In summary, our study underscores that hydrogen export ambitions, as encouraged by entities like the European Union, is in favor of domestic electricity consumers at certain export quantities, if the appropriate hydrogen regulation is in place. A proactive and comprehensive transition strategy, considering the interplay between hydrogen exports and domestic climate change mitigation, is key to unlocking economic opportunities for both hydrogen exporters and the domestic population. 


Apart from reducing greenhouse gas emissions, temporal hydrogen regulation is a decisive policy instrument in steering the welfare (re-)distribution between i) hydrogen exporting firms and hydrogen importing countries and ii) hydrogen exporting firms and domestic electricity consumers. 
A fair balance can increase acceptance across actors, is crucial for a sustainable energy transition in Morocco and offers valuable insights for further countries facing similar energy challenges globally.
