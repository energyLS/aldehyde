\begin{itemize}
    \item Summary of key findings and insights from the analysis
    \item Implications for other countries facing similar tensions between hydrogen exports and energy transition
    \item Call for further research and policy implementation
    \item Final remarks on the importance of a holistic approach to sustainable energy strategies
\end{itemize}

From small to big:
\begin{itemize}
    \item What are the results, what has been shown? $\rightarrow$ key takeaways
    \item What is the answer to the research question: Is it beneficial (total system cost, lcoh, lcoe, emissions)
    to (simultaneously) push the national energy transition
    and hydrogen export (synergies and conflicts)?
    \item Why is this research important? Climate change, COP27, hydrogen, local decarbonization, combined efforts (assets: especially with scarcity of electrolyser capacity, we should use them better)
    \item Policy recommendations
    \item " The demand of hydrogen and derivatives to decarbonize the e.g. European energy system imposes the danger of neglecting local needs and prerequisites in Morocco. Various studies are lacking [TBD: Add studies] sufficient attention to Moroccos energy system and oversimplify the multi-dimensional task of scaling up hydrogen exports. 
    By integrating and fully considering Moroccos energy system, this study identifies..."
\end{itemize}
