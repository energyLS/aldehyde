% Purpose: The Conclusion section presents the outcome of the work by interpreting the findings at a higher level of abstraction than the Discussion and by relating these findings to the motivation stated in the Introduction.

% State most important outcome
% Interpret findings at higher level
% Show to what extent you have succeeded adressing the statement in the intro
% Show what findings mean to readers, not yourself
% Make it interesting and memorable
% Perspectives: What will the authors or readers do? "We will, .." "one remaining question is"
% Focus on findings and not what I have done



% Is the research Q adressed?
Our study shows the importance of integrated scenario analysis among hydrogen exports, domestic mitigation and hydrogen regulation as raised in Chapter \ref{sec:intro}.
% What are the main results?
Both hydrogen exports and domestic mitigation offer co-benefits mainly at medium mitigation (40--60\%) and moderate to high hydrogen exports (25--150 TWh). Hydrogen regulation via temporal matching is a decisive instrument
to protect local consumers. Hourly matching decreases the cost for domestic electricity consumers (up to -31\%) and increases cost for hydrogen exporters (up to 7\%) in high export and low mitigation scenarios. Hydrogen regulation can effectively hedge domestic electricity consumers against rising prices across mitigation and export scenarios. %Furthermore, these redistribution effects are less pronounced in countries with a high share of renewable electricity.

% High level
This study contributes to investigate neo-colonial implications of hydrogen exports. Even though hydrogen regulation hedges domestic electricity consumers against rising prices, further neo-colonial implications (land use, competing RE resources, environmental concerns of desalination) can not be excluded within the scope of this study.


In summary, our study underscores that hydrogen export ambitions, as encouraged by entities like the European Union, is in favor of domestic electricity consumers 
if the appropriate hydrogen regulation is in place. A proactive and comprehensive transition strategy, considering the interplay between hydrogen exports and domestic mitigation, is key to unlocking economic opportunities for both hydrogen exporters and the domestic population. 
Apart from reducing greenhouse gas emissions, hydrogen regulation is a decisive policy instrument in steering the welfare (re-)distribution between hydrogen exporters and domestic electricity consumers. 
This balance is crucial for a sustainable energy transition in Morocco and offers valuable insights for countries/regions facing similar energy challenges globally.



%  Come back to introduction




% \begin{itemize}
%     \item What are the results, what has been shown? $\rightarrow$ key takeaways
%     \item What is the answer to the research question: Is it beneficial (total system cost, lcoh, lcoe, emissions) to (simultaneously) push the national energy transition and hydrogen export (synergies and conflicts)?
%     \item Call for further research and policy implementation
%     \item Final remarks on the importance of a holistic approach to sustainable energy strategies
% \end{itemize}