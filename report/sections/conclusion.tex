% Purpose: The Conclusion section presents the outcome of the work by interpreting the findings at a higher level of abstraction than the Discussion and by relating these findings to the motivation stated in the Introduction.

% State most important outcome
% Interpret findings at higher level
% Show to what extent you have succeeded adressing the statement in the intro
% Show what findings mean to readers, not yourself
% Make it interesting and memorable
% Perspectives: What will the authors or readers do? "We will, .." "one remaining question is"
% Focus on findings and not what I have done



% Is the research Q adressed?
Our study shows the importance of integrated scenario analysis combining hydrogen exports, domestic mitigation and temporal hydrogen regulation.
% What are the main results?
Both hydrogen exports and domestic mitigation benefit from each other mainly at emissions reduction of 40--60\% compared to the baseline scenario and moderate to high hydrogen exports (25--120 TWh/a).
However, we show that there are also risks with respect to rising domestic electricity prices
and find that hydrogen regulation via temporal matching is a decisive instrument
to protect domestic consumers. Hourly matching decreases the cost for domestic electricity consumers (up to -31\%) while the effect on hydrogen exports is minimal. Temporal hydrogen regulation can effectively hedge domestic electricity consumers against rising prices across mitigation and export scenarios. %Furthermore, these redistribution effects are less pronounced in countries with a high share of renewable electricity.

% High level
This study contributes to the investigation of implications and benefits of hydrogen exports for domestic electricity consumers. Even though temporal hydrogen regulation hedges domestic electricity consumers against rising prices, further implications of hydrogen exporters on the domestic population (land use, competing RE resources, environmental concerns of desalination) are not within the scope of this study.

In summary, our study underscores that hydrogen export ambitions, as encouraged by entities like the European Union, is in favor of domestic electricity consumers at certain export quantities, if the appropriate hydrogen regulation is in place. A proactive and comprehensive transition strategy, considering the interplay between hydrogen exports and domestic mitigation, is key to unlocking economic opportunities for both hydrogen exporters and the domestic population. 


Apart from reducing greenhouse gas emissions, temporal hydrogen regulation is a decisive policy instrument in steering the welfare (re-)distribution between i) hydrogen exporting firms and hydrogen importing countries and ii) hydrogen exporting firms and domestic electricity consumers. 
A fair balance can increase acceptance across actors, is crucial for a sustainable energy transition in Morocco and offers valuable insights for further countries facing similar energy challenges globally.



%  Come back to introduction


% \begin{itemize}
%     \item What are the results, what has been shown? $\rightarrow$ key takeaways
%     \item What is the answer to the research question: Is it beneficial (total system cost, lcoh, lcoe, emissions) to (simultaneously) push the national energy transition and hydrogen export (synergies and conflicts)?
%     \item Call for further research and policy implementation
%     \item Final remarks on the importance of a holistic approach to sustainable energy strategies
% \end{itemize}