The demand of hydrogen and derivatives to decarbonize the e.g. European energy system imposes the danger of neglecting local needs and prerequisites in Morocco. Various studies \cite{vanWijk2021, AbouSeada2022, vanderZwaan2021, Schellekens2010, Touili2022, Timmerberg2019a, Sens2022} are lacking sufficient attention to Moroccos energy system and oversimplify the multi-dimensional task of scaling up hydrogen exports. By integrating and fully considering Moroccos energy system, this study identifies both opportunities and challenges. Chapter \ref{sec:results}, highlights a key finding: when hydrogen export ambitions are aligned with emission reduction targets, local electricity prices and hydrogen prices can be reduced. This benefits both local consumers and the growing hydrogen export industry, especially when climate targets exceed 80\% reductions. It's important to note, however, that certain combinations of emissions reductions and hydrogen export volumes can lead to higher electricity prices, creating temporary challenges for local consumers and exporters. In particular, emissions reductions of 60-70\% coupled with low hydrogen exports can lead to price spikes, which may affect local acceptance of hydrogen export ambitions.

The concerns raised in Chapter \ref{sec:intro} on neo-colonial implications have practical dimensions. The price of electricity, which is crucial to local businesses and households, requires careful consideration to ensure an equitable distribution of opportunities and burdens. Our study highlights the significant impact of hydrogen export volumes on electricity prices, and emphasises the need for careful planning to avoid increases in local electricity costs. Conversely, it emphasises that hydrogen exports, particularly at a 60\% emissions reduction level, can reduce electricity prices, thereby stimulating economic growth for local businesses and households.

% Furthermore, our investigation emphasises the complex task of aligning national energy transition goals with local renewable energy acceptance in Morocco. The extensive deployment of solar PV and onshore wind, while promising, relies on local acceptance. Shifting the burden of acceptance to Morocco through large-scale renewable energy deployments solely for hydrogen export raises moral considerations and requires careful handling.

Furthermore, the allocation of the best solar PV and onshore wind resources for islanded hydrogen production only stands in opposition to decarbonizing the local electricity system. Reserving these resources solely for hydrogen export can inadvertently raise local electricity prices and limit access to valuable energy sources.

In summary, our study underscores that hydrogen export ambitions, as encouraged by entities like the European Union, do not necessarily disadvantage the local population. A proactive and comprehensive transition strategy, considering the interplay between local electricity prices, renewable energy acceptance, and resource allocation, is key to unlocking economic opportunities for both hydrogen exporters and the local population. Striking this balance is crucial for a sustainable energy transition in Morocco and offers valuable insights for nations facing similar energy challenges globally.

% \begin{itemize}
%     \item What are the results, what has been shown? $\rightarrow$ key takeaways
%     \item What is the answer to the research question: Is it beneficial (total system cost, lcoh, lcoe, emissions) to (simultaneously) push the national energy transition and hydrogen export (synergies and conflicts)?
%     \item Call for further research and policy implementation
%     \item Final remarks on the importance of a holistic approach to sustainable energy strategies
% \end{itemize}