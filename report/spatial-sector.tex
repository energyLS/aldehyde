\documentclass[11pt,preprint]{elsarticle}
% \documentclass[12pt,1p]{elsarticle}
% \documentclass[11pt]{article}

\makeatletter
\def\ps@pprintTitle{%
 \let\@oddhead\@empty
 \let\@evenhead\@empty
 \def\@oddfoot{\centerline{\thepage}}%
 \let\@evenfoot\@oddfoot}
\makeatother

\journal{Joule}

% A typical research article will be
% 4000 words of text with
% 3–5 figures and
% 30 references.

\widowpenalty10000
\clubpenalty10000
\hyphenpenalty100

\abstracttitle{Summary}

\bibliographystyle{elsarticle-num}
\biboptions{numbers,sort&compress,super}

\usepackage{libertine}
\usepackage{libertinust1math}
\renewcommand{\ttdefault}{\sfdefault}

\usepackage{geometry}
\geometry{top=30mm, bottom=35mm}

\usepackage{amsmath}
\usepackage{bbold}
\usepackage{graphicx}
\usepackage{eurosym}
\usepackage{mathtools}
\usepackage{url}
\usepackage{booktabs}
\usepackage{epstopdf}
\usepackage{xfrac}
\usepackage{tabularx}
\usepackage[version=4]{mhchem}
\usepackage{bm}
\usepackage[colorlinks]{hyperref}
\usepackage[nameinlink,sort&compress,capitalise,noabbrev]{cleveref}
\usepackage[leftcaption,raggedright]{sidecap}
\usepackage{subcaption}
\usepackage{blindtext}
\usepackage[parfill]{parskip}


\usepackage[prependcaption,textsize=scriptsize]{todonotes}
% \usepackage[prependcaption,textsize=scriptsize,disable]{todonotes}

\usepackage{siunitx}
\sisetup{range-units=single, per-mode=symbol}
\DeclareSIUnit\year{a}
\DeclareSIUnit\tco{t_{\ce{CO2}}}
\DeclareSIUnit\sieuro{\mbox{\euro}}
\DeclareSIUnit\twh{TWh}
\DeclareSIUnit\mwh{MWh}
\DeclareSIUnit\kwh{kWh}

\newcommand{\co}{\ce{CO2}~}

\usepackage{longtable}
\usepackage{multirow}
\usepackage{threeparttable}
\usepackage{pdflscape}

\usepackage[export]{adjustbox}

\usepackage[resetlabels,labeled]{multibib}
\newcites{S}{Supplementary References}
\bibliographystyleS{elsarticle-num}

\graphicspath{
	{../results/graphics-20221227/},
	{../workflows/pypsa-eur-sec/results/}
}

\newcommand{\onwrun}{20221227-onw}
\newcommand{\gasrun}{20221227-gas}
\newcommand{\decrun}{20221227-decentral}
\newcommand{\lvrun}{20221227-lv}
\newcommand{\costrun}{20221227-costs}
\newcommand{\imprun}{20221227-import}
\newcommand{\shprun}{20221227-shipping}
\newcommand{\hyrun}{20221227-main}
\newcommand{\gasrunscen}{20221227-gas/elec_s_181_lv1.0__Co2L0-3H-T-H-B-I-A-solar+p3-linemaxext10_2050}
\newcommand{\runelec}{20221227-main/elec_s_181_lvopt__Co2L0-3H-T-H-B-I-A-solar+p3-linemaxext10-noH2network_2050}
\newcommand{\runhy}{20221227-main/elec_s_181_lv1.0__Co2L0-3H-T-H-B-I-A-solar+p3-linemaxext10_2050}

% \usepackage[
% 	type={CC},
% 	modifier={by},
% 	version={4.0},
% ]{doclicense}

\newcommand{\abs}[1]{\left|#1\right|}
\newcommand{\norm}[1]{\left\lVert#1\right\rVert}
\newcommand{\set}[1]{\left\{#1\right\}}
\DeclareMathOperator*{\argmin}{\arg\!\min}
\def\cT{\mathcal{T}}
\newcommand{\R}{\mathbb{R}}
\newcommand{\B}{\mathbb{B}}
\newcommand{\N}{\mathbb{N}}
%\def\co{CO${}_2$}
\def\el{${}_{\textrm{el}}$}
\def\th{${}_{\textrm{th}}$}
\def\hy{${}_{\textrm{H}_2}$}
\def\deg{${}^\circ$}

\usepackage{xcolor}
\usepackage{framed}
\definecolor{shadecolor}{rgb}{.95,.95,.95}

\usepackage{tocloft}
\cftsetindents{section}{1em}{2.75em}

\usepackage{orcidlink}

\input{variables.tex}

\begin{document}

\begin{frontmatter}

	\title{The Potential Role of a Hydrogen Network in Europe}

	\author[tubaddress]{Fabian Neumann\,\orcidlink{0000-0001-8551-1480}\corref{correspondingauthor}}
	\ead{f.neumann@tu-berlin.de}
	\author[tubaddress]{Elisabeth Zeyen\,\orcidlink{0000-0002-7262-3296}}
	\author[aarhus,aarhus2]{Marta Victoria\,\orcidlink{0000-0003-1665-1281}}
	\author[tubaddress]{Tom Brown\,\orcidlink{0000-0001-5898-1911}}
	% \cortext[correspondingauthor]{}
	\address[tubaddress]{Department of Digital Transformation in Energy Systems, Institute of Energy Technology, Technische Universität Berlin, Fakultät III, Einsteinufer 25 (TA 8), 10587 Berlin, Germany}
	\address[aarhus]{Department of Mechanical and Production Engineering, Aarhus University, Inge Lehmanns Gade 10, 8000 Aarhus, Denmark}
	\address[aarhus2]{Novo Nordisk Foundation CO$_2$ Research Center, Aarhus University, Aarhus, Denmark}

	\begin{abstract}
		% Cover these questions:
% What is your paper about?
% Why is it important?
% How did you do it?
% What did you find?
% Why are your findings important?

% Structure
% 1. background or context: Briefly explain the problem or gap in knowledge that your research addresses. 
% 2. Objectives/Hypothesis: Clearly state the research question or objective.
% 3. methods: Provide a brief overview of the methods used in your study. Mention the study design, participants, materials, and procedures. Be concise but include enough information for readers to understand the study's approach.
% 4. results: Summarize the key findings of your study. Use quantitative information when possible, such as statistical outcomes or numerical data. Highlight the most important results that address your research question.
% 5. conclusion: Clearly state the main conclusions drawn from your study. Discuss the implications of your findings and their potential significance. Avoid making broad statements not supported by your results.

% Comments
% It would be good to have a research question or paradox or clear tension/puzzle emerge from the first 1-3 sentences

% Journal advice

% The summary is a single paragraph no longer than 150 words. An effective summary includes the following elements: 
% (1) a brief background of the question that avoids statements about how a process is not well understood; 
% (2) a description of the results and approaches/model systems framed in the context of their conceptual interest; and 
% (3) an indication of the broader significance of the work. We discourage novelty claims (e.g., use of the word “novel”) because they are overused, tend not to add meaning, and are difficult to verify. 
% Please do not include references in the summary. 

% New abstract (short)
As global demand for green hydrogen rises, the role of potential hydrogen exporters moves into the spotlight.
However, the installation of on-grid hydrogen electrolysis for export, whose electricity consumption could be many multiples of domestic demand, introduces the potential for profound impacts on both domestic energy prices, climate mitigation and energy-related emissions. Our investigation explores the dimensions of hydrogen exports, domestic mitigation and hydrogen regulation, employing a sector-coupled energy model in Morocco. 
We find significant co-benefits of domestic mitigation and hydrogen exports, whereby exports can reduce domestic electricity costs and mitigation reduces hydrogen export costs.
However, if exports rise too high in a fossil-dominated system, they can significantly raise domestic prices if green hydrogen production is not regulated.
Temporal matching of hydrogen production can not only protect domestic consumers from high costs, but lower costs by up to 65\% in the scenarios while the effect on exporters is minimal.
% The main findings apply not only to Morocco, but to a broad range of potential hydrogen exporting countries by proposing integrated policies to balance interests of exporters and domestic electricity consumers.


% % New abstract (long)

% % Introduction/Background
% Several countries are emerging as future potential exporters of green hydrogen, while simultaneously pursuing their own ambitious energy transitions to reduce greenhouse gas emissions.
% The adoption of on-grid hydrogen electrolysis for export, reaching magnitudes multiple times of the domestic electricity demand, introduces the potential for profound impacts on both domestic electricity prices and energy balances.
% Vice versa, domestic mitigation and highly renewable electricity systems provide economic opportunities for hydrogen exporters.
% % Methods
% We explore the interactions between hydrogen exports, domestic mitigation and hydrogen regulation using a fully sector-coupled capacity expansion and dispatch model of Morocco across 14 regions, including integrated gas pipeline and electricity network planning based on PyPSA-Earth. The applied sector-coupled model simulates and optimises the Moroccan energy system in 484 scenarios at a 3-hourly temporal resolution, sweeping through different hydrogen export volumes, domestic  mitigation levels and hydrogen regulations.
% % Results
% Our results show co-benefits of local mitigation and hydrogen export at medium mitigation (40 - 60\%) and moderate to high hydrogen exports (25 -- 150 TWh).
% In addition, we reveal hydrogen regulation via temporal matching as decisive instrument to steer economic redistributions between hydrogen exporters and domestic electricity consumers. 
% Hourly matching decreases the expenses for domestic electricity (up to -65\%) and increases expenses for hydrogen exports (up to 7\%) in high export and low mitigation scenarios.
% Our analysis across all three dimensions (export, mitigation, hydrogen regulation) shows the importance of strong hydrogen regulation in high export and low mitigation scenarios beyond the specific case of Morocco. 
% The redistribution effects of hydrogen regulation are less pronounced in countries with a high share of renewable electricity.
% % Conclusion & Broader Significance
% We reveal how hydrogen regulation can effectively hedge domestic electricity consumers against rising prices across mitigation and export scenarios.  Integrated policies unlock synergies, potentially increase local acceptance by reducing domestic electricity prices and counteract neo-colonialism implications of hydrogen exports.



% Old abstract
% Several countries are emerging as future potential exporters of green hydrogen, while simultaneously pursuing their own ambitious energy transitions to reduce greenhouse gas emissions. 
% Morocco serves as a blueprint for countries aiming to design their energy systems along these two dimensions, risking technology lock-in, unabated local energy-related greenhouse gas emissions, rising local energy prices, land use implications and neo-colonialism. 
% Extending the current state of research, we conduct an integrated analysis of synergies and conflicts between these two objectives, taking into account the potential impact of hydrogen exports on economic benefits for both local populations and hydrogen exporters. We present a fully sector-coupled capacity expansion and dispatch model of Morocco across 14 regions, including integrated gas pipeline and electricity network planning based on PyPSA-Earth. The applied sector-coupled model simulates and optimises the Moroccan energy system in more than 100 scenarios at a 3-hourly temporal resolution, sweeping through different hydrogen export volumes and climate targets. 
% Results show that hydrogen exports could provide significant economic benefits for both hydrogen exporters and the local population, but also create potential conflicts and rising electricity and hydrogen prices.
% In addition, potential risks of neo-colonialism highlight the need for an integrated assessment of hydrogen exports and local decarbonization. Integrated policies unlock synergies and minimise potential conflicts between hydrogen exports and Morocco's energy transition, while ensuring social and environmental sustainability.
	\end{abstract}

	\begin{keyword}
		hydrogen network, sector-coupling, retrofitting, energy systems, transmission expansion, Europe, climate-neutral, renewables
	\end{keyword}

	% \begin{graphicalabstract}
	% \end{graphicalabstract}

\end{frontmatter}

\section*{Graphical Abstract}

% https://www.cell.com/pb/assets/raw/shared/figureguidelines/GA_guide.pdf

\includegraphics[width=\textwidth]{graphics/graphical-abstract-v2-vector.pdf}

\newpage
\par\noindent\rule{\textwidth}{0.4pt}

\section*{Highlights}


\begin{itemize}
	\item examines cost benefit of a European hydrogen network in net-zero emission scenarios
	\item H$_2$ network reduces system costs by up to 3.4\%, highest without power grid expansion
	\item between 64\% and 69\% of hydrogen network uses retrofitted gas network pipelines
	\item power grid expansion saves more than hydrogen network, but strongest savings together
\end{itemize}

\par\noindent\rule{\textwidth}{0.4pt}

\section*{Context \& Scale}

Green hydrogen is an energy carrier providing many possible applications supporting the sustainable energy transition. 
This versatility drives the global hydrogen demand, putting future hydrogen exporters in the spotlight. 
However, the export of relevant green hydrogen volumes requires large electricity demands in the exporting country's energy system which has major effects on their energy prices, domestic climate mitigation and energy-related emissions, often neglected by hydrogen importers focussing on solely costs and potentials.

We highlight the importance of the exporter's perspective by analyzing numerous scenarios of hydrogen export volumes, domestic climate change mitigation levels and temporal hydrogen regulations. In fact, both hydrogen exports and domestic climate change mitigation show synergies in the exporter's energy system. 

Strict temporal hydrogen regulation hedges domestic electricity consumers against rising prices and provides a powerful policy instrument to support the domestic energy transition.



\par\noindent\rule{\textwidth}{0.4pt}

\section*{eTOC Blurb}

We examine the interplay between a pan-continental hydrogen network and
electricity grid expansion in Europe to support renewable energy integration. By
repurposing gas pipelines, a hydrogen network can reduce energy system costs by
up to 3.4\% and help balance variations in wind and solar energy. Although the
cost benefit of electricity grid expansion is higher, combining both
infrastructures maximizes savings at 9.9\%. Together, they offer cost-effective
options for achieving a European energy system with net-zero \co emissions.

\newpage
\section*{Introduction}
\label{sec:intro}
\addcontentsline{toc}{section}{\nameref{sec:intro}}


\subsection{Ideas}
From big to small:
\begin{itemize}
    \item Why is this research important? Climate change, COP27, hydrogen, local decarbonization, combined efforts (assets: especially with scarcity of electrolyzer capacity, we should use them better)
    \item What is the aim?
    \item What are the results?
\end{itemize}
Narrative: 
\begin{itemize}
    \item Security of supply/economic growth/cheap energy for North Africa and Morocco
    \item Fair cooperation and partnerships by: including local energy demands, investigation of conflicts/synergies
    \item Best locations with high FLH should not be reserved for export (as done in Global PtX Atlas)
    \item EU can rely on cost-effective green hydrogen
\end{itemize}

Details on benefits for MAR:
\begin{itemize}
    \item MAR is ramping up hydrogen strategy
    \item MAR is a net importer of energy (Currently importing 90 \%), could become exporter
    \item Local production of RE plants possible, as stated in \cite{Ersoy2022}
    \item Avoid "gr{\"u}ner Energie-Imperialismus" (Habeck: when visiting Namibia ) \cite{HabeckEnergieimperialismus}
\end{itemize}

Current cooperations/business:
\begin{itemize}
    \item Indias role: https://www.moroccoworldnews.com/2023/02/353972/indias-renew-energy-global-eyes-moroccos-green-hydrogen-market
    \item TBD: Investigation of EU/Germany involvement in MAR?
    \item Cooperation between port authorities in Tanger Med and Hamburg
\end{itemize}


Research questions:
\begin{itemize}
    \item Are there wealth, cost and emission benefits of an integrated optimisation compared to separated analyses?
    \item Are there competing RE potential resources and where are they located? Should these resources used for local decarbonization or for export purposes?
    \item What is the role of green/grey hydrogen?   
    \item Can an integrated analysis avoid carbon leakage?
    \item Does ongrid integration avoid installing excess RES and El capacity? (That's a finding of e.g. \cite{Ruhnau2022})
    \item What are the conflicts and benefits of national transition and hydrogen export?
\end{itemize}

Literature research:
\begin{itemize}
    \item Local energy demand considered in: MenaFuels and \cite{Hampp2021}. According to MenaFuels the local demand is usually just some percentages of the potential
    \item TBD: Literature on similar topics 
\end{itemize}


\subsection{Policy and climate targets}
Morocco has implemented various strategies and policies to reach its climate targets and promote hydrogen exports. 
These include programs on the expansion of RE, the development of hydrogen for local demands and exports as well as climate targets.

\subsubsection{Moroccan RE Strategy}
The Moroccan RE strategy consists of a "Moroccan Solar Plan" initiated in 2009 \cite[p. 2]{Boulakhbar2020}
and a wind energy program.
The Moroccan wind energy program promotes 2000 MW by 2020 and 2600 MW by 2030 \cite[p. 4]{Boulakhbar2020}
The solar plan ... (TBD: Describe the solar plan)
(See \cite[p. 13]{Ersoy2022} for a good overview of the Moroccan RE strategy).

\subsubsection{Moroccan Hydrogen Strategy}
Figure \ref{fig:mar_hydrogen_strategy} shows the hydrogen strategy of Morocco, including
national (ammonia, hydrogen) and export demands (synfuels, hydrogen).
By 2030, the total hydrogen demand adds up to 13.5 TWh (2040: 58 TWh, 2050: 151 TWh), 
clearly listing higher demands for export than for local use. The hydrogen generation is backed by 5.2 GW of RE in 2030, 23 GW in 2040 and 55 GW in 2050 [TBD].


\begin{figure}
    \centering
    \includegraphics[width=\linewidth]{policy/mar_hydrogen_strategy_grouped.pdf}
    \caption{Morocco Hydrogen Strategy, cf. \cite[p. 14]{Ersoy2022}. TBD primary source. TBD include RE capacities according to Hydrogen Strategy?}
    \label{fig:mar_hydrogen_strategy}
\end{figure}

\subsubsection{Climate targets}
Apart from the RE and Hydrogen strategies, Morocco has obliged to climate targets.
Figure \ref{fig:morocco_em} displays the historical emissions and targets of Morocco.
By 2030, Morocco aims to limit it's GHGs to 75 MtCO2e (conditional) 
respectively 115 MtCO2e (unconditional) (NDC, excluding LULUCF) \cite{CAT2021}. 
According to \cite{CAT2021}, Morocco has not yet submitted a net-zero target.

TBD: A GHGe reduction of 32 \% by 2030 is mentioned in \cite[5]{Boulakhbar2020}, check that.
NDCs: https://www.iges.or.jp/en/pub/iges-indc-ndc-database/en 

\begin{figure}[h!]
    \centering
    \includegraphics[width=\linewidth]{morocco_em.pdf}
    \caption{Morocco historical emissions and targets.}
    \label{fig:morocco_em}
\end{figure}

In this section we focus on the synergies and conflicts of hydro gen export and national energy transition. First, the
hydrogen and electricity prices of the integrated optimisation are analyzed in Section \ref{subsec:results_prices} along the two-dimensional scenario space presented in Section \ref{subsec:scenarios}. Based on these prices, Section \ref{subsec:economic_benefits} derives the total expenses of electricity and hydrogen and outlines economics benefits. Section \ref{subsec:integration_challenges} points out arising integration challenges (graphs in appendix, explaination in Section \ref{subsec:results_prices}?). The sensitivity analysis in Section \ref{subsec:sensitivity} examines key parameters and their influence on the main results.


\subsection{Hydrogen and electricity prices}
\label{subsec:results_prices}

Both the electricity and hydrogen prices are derived from the nodal power balance constraint, representing the prices in a "perfect" market \cite{Zeyen2022}.
In a second step, these temporally and spatially resolved prices are weighted based on different consumptions in hourly resolution to obtain a uniform, weighted marginal price of hydrogen/electricity of Morocco in each scenario.

The hourly-resoluted consumptions on which the weighting is based on differ in:
\begin{enumerate}
    \item electricity consumption on each node ($\rightarrow$ electricity prices everyone sees: "Export-Local-Average"),
    \item hydrogen consumption on each node ($\rightarrow$ hydrogen prices everyone sees: "Export-Local-Average"),
    \item electricity demand of electrolysers on each node ($\rightarrow$ electricity prices which electrolysers see: "Export-orientated") and
    \item electricity demand of all consumers except electrolysers ($\rightarrow$ electricity prices which local consumers see: "Local-orientated").
\end{enumerate}

These various types of weighted marginal prices are displayed in Figure \ref{fig:marginal_prices}. All four subfigure are based on the same scenarios which varies the hydrogen export volume from 0 - 200 TWh, and the emission reduction between 0 - 100\% resulting in 110 optimisation results displayed in each subfigure (cf. Sec. \ref{subsec:scenarios}).

The Export-Local-Average electricity price (s. Fig. \ref{fig:export_local_el_price})...
\begin{itemize}
    \item 
\end{itemize}

The Export-Local-Average hydrogen price (s. Fig. \ref{fig:export_local_hy_price})...
\begin{itemize}
    \item More export $rightarrow$ higher prices
    \item 
\end{itemize}

The Local-orientated electricity price (s. Fig. \ref{fig:local_el_price})...
\begin{itemize}
    \item 
\end{itemize}

The Export-orientated electricity price (s. Fig. \ref{fig:export_el_price})...
\begin{itemize}
    \item 
\end{itemize}




\begin{figure*}[t] % Use 'figure*' to span both columns
    \centering
    \begin{subfigure}[b]{0.45\linewidth}
        \centering
        \includegraphics[width=\linewidth]{graphics/integrated_comp/contour_lcoe_w_20_filterFalse.pdf}
        \caption{Export-Local-Average electricity price}
        \label{fig:export_local_el_price}
    \end{subfigure}
    \hfill
    \begin{subfigure}[b]{0.45\linewidth}
        \centering
        \includegraphics[width=\linewidth]{graphics/integrated_comp/contour_lcoh_w_mixed_20_filterFalse.pdf}
        \caption{Export-Local-Average hydrogen price}
        \label{fig:export_local_hy_price}
    \end{subfigure}
    \hfill
    \begin{subfigure}[b]{0.45\linewidth}
        \centering
        \includegraphics[width=\linewidth]{graphics/integrated_comp/contour_loce_w_no_electrolysis_20_filterFalse.pdf}
        \caption{Local-orientated electricity price}
        \label{fig:local_el_price}
    \end{subfigure}
    \hfill
    \begin{subfigure}[b]{0.45\linewidth}
        \centering
        \includegraphics[width=\linewidth]{graphics/integrated_comp/contour_loce_w_electrolysis_20_filterFalse.pdf}
        \caption{Export-orientated electricity price}
        \label{fig:export_el_price}
    \end{subfigure}
    \hfill

    \caption{Marginal prices of electricity and hydrogen subject to export volumes and emission limits depending on various weightings. Black lines indicate the lowest price at each emission limit.}
    \label{fig:marginal_prices}
\end{figure*}


\subsection{Economic benefits for local consumers and exporting industry}
\label{subsec:economic_benefits}

\subsection{Integration challenges}
\label{subsec:integration_challenges}
Figure \ref{fig:diurnal} shows the diurnal residual load (electricity load minus solar PV and wind power generation) at 0 export and 0 - 90\% emission reduction. At high emission reductions, a high penetration of RE generation (especially solar PV) 
leads to negative residual load at noon imposing great system integration challenges.
These challenges are (partly) tackled by the expansion of battery storage and increasing curtailment rates of onshore wind and solar PV. At low export rates, the deployment of battery storage reduces the curtailment. Figure \ref{fig:integration_options}
reveals that at increasing hydrogen export and high emission reduction both battery storage and curtailment are reduced, due to the synergy with export driven energy infrastructure and it's ability to integrate RE at large scale.
This synergy is also displayed in Figure \ref{fig:export_local_el_price}, reflecting lower average electricity prices at high reduction and high export.

\begin{figure}[h!]
    \centering
    \includegraphics[width=\linewidth]{graphics/rldc/rlc_diurnal_inclCurtTrue.pdf}
    \caption{Diurnal residual load at 0 export and 0 - 90\% emission reduction}
    \label{fig:diurnal}
\end{figure}

\subsection{Sensitivity analysis}
\label{subsec:sensitivity}
Sensitivity analysis on:
\begin{itemize}
    \item WACC
    \item Technology cost
    \item Hourly resolution
\end{itemize}


\subsection{Miscellaneous}
List of possible results and aspects worth looking at
\begin{itemize}
    \item Carbon intensity
    \begin{itemize}
        \item What is the carbon intensity for export and local consumption of the islanded scenario? 
        \item What is the carbon intensity of the integrated scenario?
        \item Compare with blue hydrogen (1 kgCO2/kgH2) and EU threshold for low-carbon gas (3 kgCO2/kgH2)
        \item Differnce between 2030 and 2050?
    \end{itemize}
\end{itemize}



\section*{Discussion}
\label{sec:discussion}
\addcontentsline{toc}{section}{\nameref{sec:discussion}}

This chapter discusses synergies and conflicts (s. Sec. \ref{subsec:synergies-conflicts}) of the dual goal of increasing hydrogen export and emission reductions. Section \ref{subsec:timepath} puts the results into perspective of a synergy maximizing time frame. Neo-colonial implications deriving from hydrogen export are discussed in Section \ref{subsec:neocolonial}.
% and possible policies increasing fairness are suggested in Section \ref{subsec:fairness}. 
Limitations of this study are discussed in Section \ref{subsec:limitations}.

% General ideas
% \begin{itemize}
%     \item Identification of conflicts and synergies between hydrogen exports and energy transition
%     \item Implications for renewable energy deployment
%     \item Also: Refer to policy from MAR?
% \end{itemize}


\subsection{Synergies and conflicts}
\label{subsec:synergies-conflicts}
Results from Chapter \ref{sec:results} indicate various synergies and conflicts of hydrogen export and national energy transition. The price of electricity highly depends on the climate ambition and hydrogen export volume, e.g. increasing the costs of electricity for local demands at medium climate ambition (60\%) and high export volumes (150 - 200 TWh). On the other hand, the green hydrogen constraint requiring the installation of renewable energies according to the hydrogen exports hedges local electricity consumers against higher electricity prices at lower climate ambitions. If such a constraint is not imposed on the hydrogen exports, the export industry could benefit from cheaper electricity prices for electrolysis.

The electricity price increase flattens out at high climate ambitions, meaning local electricity consumers are able to avoid high economic burdens imposed by a hydrogen infrastructure in these scenarios.

The hydrogen prices increase with hydrogen exports across all climate ambition scenarios. Nonetheless, the increase is less significant when the local energy system relies already on substantial amounts of hydrogen and Fischer-Tropsch products. The price of hydrogen for export is lowest when the local energy transition is still in the early stages and the export volumes are low -- electrolysers can run at high capacity factors paying of high upfront capital costs. 


\subsection{Synergy maximizing time path}
\label{subsec:timepath}
All scenarios investigated in this study are based on the year 2030 as reasoned in Chapter \ref{sec:methods}. The transition towards climate neutrality of Moroccos energy system or the ramping up of exports up to 200 TWh/a are a matter of decades and undergoes a certain pathway. Based on the synergies and conflicts identified in Section \ref{subsec:synergies-conflicts}, steering the transition by minimizing conflicts and maximizing/amplifying synergies reduce both the price of electricity and price of hydrogen. To avoide high electricity prices, it is advisable to avoid high export volumes at medium climate ambitions. It is beneficial, to reach a certain degree of local decarbonization before exporting hydrogen volumes of 100 TWh and above, to use the synergies between the flexibilities provided by a net-zero energy system (electrolysis, hydrogen storage, Fischer-Tropsch) for a large-scale integration of variable renewable energies required for substantial export volumes.


\subsection{Neo-colonial Implications}
\label{subsec:neocolonial}
Neo-colonial implications mentioned in Chapter \ref{sec:intro} are reflected in various dimensions. First, the price of electricity for local businesses and households need to be considered to enable a fair discussion and balance of burdens and opportunities. This study shows that the price of electricity highly depends on hydrogen export volumes and uncareful planning inherits the danger of increasing electricity prices for local demands. Converesly, the export of hydrogen provides the opportunity of decreasing electricity prices through the green hydrogen constraint, providing opportunities for local businesses and households.
Second, the significant deployment of solar PV and onshore wind comes along with a substantial land use. Several studies \cite{Terrapon-Pfaff2019, Hanger2016} highlight the requirement of local acceptance of REs in Morocco. From a German/European perspective, hydrogen imports are not necessarily the most economic option but also provide a way out of having to deal with national acceptance of REs in Germany/Europe. Shifting such problems of acceptance to Morocco by deploying REs and scaling up their hydrogen exports imposes moral scruples and should be delt with carefully. %TBD: Desertec: why did it fail.
Third, the best solar PV and onshore wind potentials should not be reserved for hydrogen exports by setting up islanded hydrogen production in these locations. This leads to higher prices of electricity of the local electricity system which then can not access these cost-cutting potentials.

Hydrogen export ambitions (fostered by e.g. the EU) do not cause disadvantages for the local population \textit{per se}, a proactive transition taking into account the interplay of local electricity prices unleashes economic opportunities for both exporters and local population.

% Ideas:
% \begin{itemize}
%     \item Danger: Avoid Neo-colonialism (Export industry fostered by EU etc. has a significant impact)
%     \item Increasing electricity prices for local consumers? Driven by climate ambition, not by export. Rather synergy at high export and high climate ambition, decreasing electricity prices for local consumers
%     \item Land use: The best potentials should not be reserved for export
%     \item Land use competition and its consequences
%     \item Potential risks of neo-colonialism and their significance
% \end{itemize}


% The demand of hydrogen and derivatives to decarbonize the e.g. European energy system imposes the danger of neglecting local needs and prerequisites in Morocco. Various studies are lacking [TBD: Add studies] sufficient attention to Moroccos energy system and oversimplify the multi-dimensional task of scaling up hydrogen exports. 
% By integrating and fully considering Moroccos energy system, this study identifies resulting p


% \subsection{Integrated policies for synergy and fairness}
% \label{subsec:fairness}

% \begin{itemize}
%     \item Tax hydrogen electricity price? They get the cheaper electricity prices at least. Ensure equitable economic distribution
%     \item Land use policies?
%     \item Aligning hydrogen export goals with renewable energy targets/climate targets 
%     \item Mitigating risks of neo-colonialism through responsible practices
% \end{itemize}





\subsection{Limitations}
\label{subsec:limitations}

The sector-coupled energy model as well as the results underly certain limitations. In this study, the only export energy carrier investigated is hydrogen, although the export of hydrogen derivatives as ammonia, Fischer-Tropsch products, HVCs or even products as sponge iron etc. are under discussion \cite{Verpoort2023}. Further synthesis (Haber-Bosch, Fischer-Tropsch) and linked transportation modes are investigated in \cite{Hampp2021} or \cite{Galimova2023} but not the scope of this study. Further synthesis for export products will likely have effects on the system operation, since the synthesis in combination with cheaper storage (e.g. oil storage are cheaper than H2 storage \cite{DEA2019TechnologyData}) might provide additional flexibility to a certain extend (limited by the operation of Fischer-Tropsch or Haber-Bosch).

Additionally, the transition of certain sectors is not subject to the optimisation but exogenously defined. Such nodal shifts (e.g. in transport: share of electric vehicles) provide system benefits (e.g. Vehicle-to-Grid) for which an integrated optimisation favours e.g. higher shares of electric vehicles.
%[TBD: How does e.g. \cite{Neumann2022} justify it?].
Recent approaches as \cite{Zeyen2023} improve such caveats by incorporating endogenous learning, but research gaps on the endogenous demand of the transport sector remain.

National climate ambitions and especially the scale-up of hydrogen export require significant amounts of raw materials (copper, cement, ...) linked with resource limitations as well as upstream GHG emissions. As pointed out in \cite{Wang2023} in a global analysis, most raw material limitations required for electricity generation do not exceed geological reserves. However, material demands of e.g. electrolysers are not within the scope of \cite{Wang2023}.

The cost of water through desalination and transport has a spatial resolution on country level. However, the transportation costs highly depend on the terrain as pointed out by \cite{Caldera2016}, hence the water costs are higher in more remote areas which is not taken into account in this study. In contrast, the effect of a higher spatial resolution of water costs is assumed to be minor, since the water desalination and transport costs do not contribute significantly to the cost of hydrogen as emphasized in \cite{Hampp2023}. %Also: Direct integration of REs to desalination is not considered

% TBD: comparison with other studies
% \subsection{Comparison with other studies}
% E.g. compare to \cite{Hampp2021} and their cost of electricity. High price of hydrogen due to 15\% WACC!

% TBD: further research
% \subsection{Further research}
% \begin{itemize}
%     \item Transport options
%     \item Include other countries (But focus is Morocco)
%     \item Include other costs
% \end{itemize}

\section*{Conclusion}
\label{sec:conclusion}
\addcontentsline{toc}{section}{\nameref{sec:conclusion}}

\subsection*{Conclusion}

% Purpose: The Conclusion section presents the outcome of the work by interpreting the findings at a higher level of abstraction than the Discussion and by relating these findings to the motivation stated in the Introduction.

% State most important outcome
% Interpret findings at higher level
% Show to what extent you have succeeded adressing the statement in the intro
% Show what findings mean to readers, not yourself
% Make it interesting and memorable
% Perspectives: What will the authors or readers do? "We will, .." "one remaining question is"
% Focus on findings and not what I have done



% Is the research Q adressed?
Our study shows the importance of integrated scenario analysis combining hydrogen exports, domestic climate change mitigation and temporal hydrogen regulation.
% What are the main results?
Both hydrogen exports and domestic climate change mitigation benefit from each other mainly at emissions reduction of 40--60\% compared to the baseline scenario and moderate to high hydrogen exports (25--120 TWh/a).
However, we show that there are also risks with respect to rising domestic electricity prices
and find that hydrogen regulation via temporal matching is a decisive instrument
to protect domestic consumers. Hourly matching decreases the cost for domestic electricity consumers (up to -31\%) while the effect on hydrogen exports is minimal. Temporal hydrogen regulation can effectively hedge domestic electricity consumers against rising prices across mitigation and export scenarios. %Furthermore, these redistribution effects are less pronounced in countries with a high share of renewable electricity.

% High level
This study contributes to the investigation of implications and benefits of hydrogen exports for domestic electricity consumers. Even though temporal hydrogen regulation hedges domestic electricity consumers against rising prices, further implications of hydrogen exporters on the domestic population (land use, competing RE resources, environmental concerns of desalination) are not within the scope of this study.

In summary, our study underscores that hydrogen export ambitions, as encouraged by entities like the European Union, is in favor of domestic electricity consumers at certain export quantities, if the appropriate hydrogen regulation is in place. A proactive and comprehensive transition strategy, considering the interplay between hydrogen exports and domestic climate change mitigation, is key to unlocking economic opportunities for both hydrogen exporters and the domestic population. 


Apart from reducing greenhouse gas emissions, temporal hydrogen regulation is a decisive policy instrument in steering the welfare (re-)distribution between i) hydrogen exporting firms and hydrogen importing countries and ii) hydrogen exporting firms and domestic electricity consumers. 
A fair balance can increase acceptance across actors, is crucial for a sustainable energy transition in Morocco and offers valuable insights for further countries facing similar energy challenges globally.



%  Come back to introduction


% \begin{itemize}
%     \item What are the results, what has been shown? $\rightarrow$ key takeaways
%     \item What is the answer to the research question: Is it beneficial (total system cost, lcoh, lcoe, emissions) to (simultaneously) push the national energy transition and hydrogen export (synergies and conflicts)?
%     \item Call for further research and policy implementation
%     \item Final remarks on the importance of a holistic approach to sustainable energy strategies
% \end{itemize}

\section*{Experimental Procedures}
\label{sec:methods}
\addcontentsline{toc}{section}{\nameref{sec:methods}}

\subsection{Sector-coupled energy model}
\label{subsec:moroccan_model}

The sector-coupled energy model is based on PyPSA-Earth \cite{Parzen2022} and adds a sector extension similar to \cite{Brown2018a}.
This capacity extension model optimizes the generation, transmission and storage of Moroccos energy system by minimising the the total annualised system costs
constrained by e.g. climate targets or green hydrogen policies (s. Sec. \ref{subsubsec:green_hydrogen_constraint}). The geographical scope is Morocco.

\subsubsection{Energy demand}
Hazem
The energy demand is obtained from the "LEAP" model \cite{Heaps2022}.
TBD: More infos on MenaFuels: see \cite[p. 35]{Ersoy2022}, better: "Teilbericht 9"

The various sectors are (TBD clarify each bullet point: what is the energy prediction based on):
\begin{itemize}
    \item Electricity demand based on \cite{Parzen2022}. Electricity demand also stated in \cite[primary source 25]{Boulakhbar2020}
    \item Industry demand
    \item Transport demand
    \item Heating demand (cooling?) 
    \item Regionalisation of energy demands
\end{itemize}



\subsubsection{Renewable Energy Sources}
Atlite, using data pipeline from \cite{Parzen2022}.





\begin{figure}[h!]
    \centering
    \includegraphics[width=\linewidth]{../graphics_general/MAR_brownfield_el.png}
    \caption{Current capacities of electricity generation and distribution, obtained from \cite{Parzen2022} and visualization based on [TBD: add source of "documentation"]}
    \label{fig:MAR_brownfield}
\end{figure}


\subsubsection{Conventional electricity generation}
We use: ... (TBD: List our sources)
Other sources: S\&P Global Kraftwerksliste (used by Fraunhofer IEE).
See \cite[p. 26]{Ersoy2022}  on conventional generation in Morocco.



\subsubsection{Electricity (and gas) networks}
The electricity network is based on the PyPSA-Earth workflow using OSM data as main source \cite{Parzen2022}. Figure \ref{fig:MAR_brownfield} shows the current capacities of electricity generation and distribution based on the PyPSA-Earth workflow. [TBD: Describe more]
3,000 km of 400 kV, 9,000 km of 225 kV, 147 km of 150 kV lines 
11,780 km of 60 kV lines \cite[p. 6, primary 44]{Boulakhbar2020}





\subsubsection{Water supply}
\label{subsec:water_supply}
Water supply is a crucial input for the electrolysis process. 
Morocco: 88 \% of water is used in agriculture, secondary source \cite{Ersoy2022}.
Transfer of water from North to South is planned, seconary source \cite{Ersoy2022}.
Ideas:
\begin{itemize}
    \item Detailed water analysis in HyPat: Topography $\rightarrow$ water pumps, etc. See results in Figure \ref{fig:morocco_water}
    \item Oversize water desalination to provide water for local population/agriculture to increase public acceptance
    \item DAC: Water is a product of DAC process which could be reused for electrolysis
    \item Direct electrolysis of seawater: In development but probably not on big scale in the next 10-15 years
\end{itemize}


\subsection{Green hydrogen policy}
\label{subsubsec:green_hydrogen_constraint}
% https://www.ffe.de/en/publications/how-is-green-hydrogen-defined-according-to-the-eus-delegated-act/

A key model constraint is the requirement of the hydrogen exported from Morocco to be "green". A variety of studies looks into various dimensions (temporal, geographical, electricity origin) of green hydrogen and their trade-offs (\cite{Brauer2022}, \cite{Ruhnau2022}, \cite{Zeyen2022}). Envisioning hydrogen offtakers from the European Union, the green hydrogen constraint applied in this study aligns with the \emph{Delegated regulation on Union methodology for RFNBOs} of the European Commission \cite{Commission2023} defining green hydrogen  based on the following criteria:

\begin{enumerate}
    \item Additionality: The electricity demand of electrolysers must be provided by additional RE power generation (less than 3 years before the installation of electrolysers, from 1.1.2028 onwards),
    \item Temporal correlation: Hourly matching (from 1.1.2030 onwards, until 31.12.2029 monthly) $\rightarrow$ currently the models run on monthly matching and
    \item Geographical correlation
\end{enumerate}
which are required for PPAs with RE-installation. Apart from the PPAs, the delegated act also considers the possibility of:
\begin{itemize}
    \item direct connection of RE and electrolysers,
    \item high share of RE in power mix (> 90\%) or the
    \item avoidance of RE curtailment
\end{itemize}
to qualify for the "green" hydrogen label. These further options are not considered here, since a system-integrated electrolysis in a power system of a (current) share of RE below 90\% is the scope of this study (s. Sec. \ref{subsec:scenarios}). The sole focus on RE curtailment and hence low total hydrogen volumes is not applicable due to expected hydrogen exports of up to multiple times of Moroccos local electricity demand. The green hydrogen definition of the European Commission prior to 1.1.2028 is not applied in this study, since relevant volumes of hydrogen export are expected to materialize from 2028 onwards and is excluded by the scope of this study (s. Sec. \ref{subsec:scenarios}).

TBD: The locally demanded hydrogen does not meet the Additionality criteria (but probably the temporal correlation, be careful here)

\subsection{Hydrogen export}
\label{subsec:hydrogen_export}
The amount of hydrogen to be exported is derived from section \ref{subsec:policyandtargets} and implemented as an exogenous parameter in the range of 0 - 200 TWh (s. Sec. \ref{subsec:scenarios}). Similar to (MenaFuels) \cite{Ersoy2022} the system boundary is the country border of Morocco, hence not considering various transport options (as shipping or pipeline) or further conversions (Methane, Ammonia, etc.) [TBD: justify]
Nonetheless, the demand pattern of ship export (landing, loading, travel time) and the resulting energy system implications (compared to a constant export profile, closer to a pipeline behaviour) to Morocco are considered here and further investigated in Section \ref{subsec:sensitivity}.
The export of hydrogen is allowed via a range of selected ports in Morocco [TBD: provide table or map in supplementary]. 

The energy system model allows an endogenous spatial export decision, meaning that the total demand (and export profiles) are exogenous, but the model chooses the  cost-optimal export location(s). At each port, the model has the option to build a hydrogen underground or steel tank depending on local geological conditions [TBD: compare open data for underground storage].


\subsection{Cost of hydrogen production}
\label{subsec:cost_of_hydrogen_production}

The cost of hydrogen production is calculated as follows \cite{Zeyen2022}:
\begin{equation}
    \label{eq:cost_of_hydrogen_production}
    C_\mathrm{H2}= \frac{\sum \limits_{a\in A}^{} C_a + \sum \limits_{a,t} O_{a,t} + \sum \limits_{t} P_t \cdot (im_t - ex_t)   }{d_\mathrm{H2}}
\end{equation}
"where $C_a$ is the capital cost of the asset $a$, $O_{a,t}$ is the operational cost of the asset $a$ in time step $t$, $P_t$ is the production of hydrogen in time step $t$, $im_t$ is the import of hydrogen in time step $t$ and $ex_t$ is the export of hydrogen in time step $t$" \cite{Zeyen2022}.


Alternative method of calculating the LCOH in an integrated system:
\begin{equation}
    \label{eq:cost_of_hydrogen_production_simple}
    C_\mathrm{H2}=\frac{TSC_\mathrm{export}-TSC_\mathrm{noexport}}{d_\mathrm{H2export}}
\end{equation}
Equation \ref{eq:cost_of_hydrogen_production_simple} provides another approach where the total system cost of the integrated system without hydrogen export $TSC_\mathrm{noexport}$ is subtracted from the total system cost of the integrated system with hydrogen export $TSC_\mathrm{export}$ to get the total increase in cost, then divided by the hydrogen export volume $d_\mathrm{H2export}$.

Note, equation \ref{eq:cost_of_hydrogen_production} is not valid for the integrated system (?),
hence the cost of hydrogen of the integrated system is defined as the marginal price of hydrogen (taken from PyPSA).
Questions:
\begin{itemize}
    \item Marginal price of all hydrogen buses or just export?
    \item "Weighted" marginal price makes more sense
    \item Is the equation \ref{eq:cost_of_hydrogen_production} valid?
\end{itemize}


\subsection{Scenarios}
\label{subsec:scenarios}
To examine the conflicts and synergies of hydrogen export and national energy transition, we sweep along two dimensions:
\begin{enumerate}
    \item The hydrogen export volume varies from 0 - 200 TWh, going beyond Moroccos hydrogen export ambitions of 110 TWh [TBD: provide exact number] (cf. sec. \ref{subsec:policyandtargets}) and
    \item the emission reduction between 0 - 100\% based on Morocco's current emissions of 72 MtCO2e (cf. sec. \ref{subsec:policyandtargets}).
\end{enumerate}
This two-dimensional scenario space leads to 110 model runs. 
A 3-hourly resolution is chosen to capture energy system dynamics (e.g. RE generation profiles, energy storage operation) with it's diurnal and seasonal variations in energy supply and demand. [TBD: see Neumann for justification]
A high spatial resolution of 53 nodes represents the geographical heterogeneity of RE resources, demand centers and energy networks. Furthermore, a high spatial resolution provides insights on land use conflicts and competing RE resources as well as a spatial differentiation of export ports. Since the 53 nodes align with the GADM (Global Administrative Areas) regions, policy advisement can be tailored to specific regions considering local needs and constraints.

TBD: Elaborate more on scoping decisions
\begin{itemize}
    \item Update to 2035/2040, or even time dependent?
    \item Line expansion?
    \item Shipping profile?
    \item SMR, CO2 network
    \item Explain Brownfield here (s. Fig. \ref{fig:MAR_brownfield})?
\end{itemize}



\subsection{Miscellaneous}

\subsubsection{Islanded hydrogen production}
In the islanded hydrogen production scenario, the hydrogen production is optimised without considering the local energy system described in \ref{subsec:moroccan_model}.

\subsubsection{Integrated hydrogen production}
In the integrated scenario, both the Moroccan energy system (\ref{subsec:moroccan_model}) and the hydrogen export (\ref{subsec:hydrogen_export})
are optimised in an integrated approach.


\subsubsection{Literature: Potentials}
\begin{itemize}
    \item Benchmark potentials to other studies (e.g. IRENA, MenaFuels which uses "EnDat", HyPAT provided on online map?, ...)
    \item MenaFuels \cite[p. 25]{Ersoy2022} (better source: Teilbericht 10)
    \item See \cite{Ersoy2022} (better primary source) page 26 on Renewable generation in Morocco: Currently 37\% of the installed capacity from RE
    \item See \cite[p. 5]{Boulakhbar2020} for list of hydro projects (current \& future)
    \item See \cite[primary 10,13,17]{Boulakhbar2020} for MAR potentials
    \item See \cite[primary 13, 38]{Boulakhbar2020} wind power of 6000 MW
    \item In \cite[p. 6]{Boulakhbar2020} they outline why biomass/biogas is not exploited yet in MAR
    \item Check the RePP database \cite{Peters2023}
\end{itemize}

\section*{Supplemental Information}

Supplemental information is included in \nameref{sec:si}.

\section*{Acknowledgements}

We are grateful for helpful comments by Johannes Hampp. We thank the
reviewers for their valuable feedback and suggestions.

\section*{Declaration of Interests}

The authors declare no competing interests.

% \section*{License}

% \href{http://creativecommons.org/licenses/by/4.0/}{Creative Commons Attribution 4.0
% International License (CC-BY-4.0)}

\section*{Author Contributions}

% following https://casrai.org/credit/

\textbf{F.N.}:
Conceptualization --
Data curation --
Formal Analysis --
Investigation --
Methodology --
Software --
Supervision --
Validation --
Visualization --
Writing - original draft --
Writing - review \& editing
\textbf{E.Z.}:
Data curation --
Formal Analysis --
Investigation --
Software --
Validation --
Writing - review \& editing
\textbf{M.V.}:
Formal Analysis --
Investigation --
Methodology --
Software --
Writing - review \& editing
\textbf{T.B.}:
Conceptualization --
Data curation --
Formal Analysis --
Funding acquisition --
Investigation --
Methodology --
Project administration --
Resources --
Software --
Supervision --
Writing - original draft --
Writing - review \& editing

% tidy with https://flamingtempura.github.io/bibtex-tidy/
\addcontentsline{toc}{section}{References}
\renewcommand{\ttdefault}{\sfdefault}
%\bibliography{library}
\bibliography{/home/fneum/zotero}

% supplementary information

\newpage

\makeatletter
\renewcommand \thesection{S\@arabic\c@section}
\renewcommand\thetable{S\@arabic\c@table}
\renewcommand \thefigure{S\@arabic\c@figure}
\makeatother

\renewcommand{\citenumfont}[1]{S#1}

\setcounter{equation}{0}
\setcounter{figure}{0}
\setcounter{table}{0}
\setcounter{section}{0}

\subsection{Integration challenges}


Figure \ref{fig:diurnal_exportdim} displays the deployment of battery and hydrogen storage as well as curtailment.

\begin{figure*}[t] % Use 'figure*' to span both columns
    \centering
    \begin{subfigure}[b]{0.3\linewidth}
        \centering
        \includegraphics[width=\linewidth]{graphics/rldc/rlc_diurnal_m0.90.pdf}
        \caption{10\% emission reduction}
        \label{fig:diurnal_10emred}
    \end{subfigure}
    \hfill
    \begin{subfigure}[b]{0.3\linewidth}
        \centering
        \includegraphics[width=\linewidth]{graphics/rldc/rlc_diurnal_m0.50.pdf}
        \caption{50\% emission reduction}
        \label{fig:diurnal_50emred}
    \end{subfigure}
    \hfill
    \begin{subfigure}[b]{0.3\linewidth}
        \centering
        \includegraphics[width=\linewidth]{graphics/rldc/rlc_diurnal_m0.10.pdf}
        \caption{90\% emission reduction}
        \label{fig:diurnal_90emred}
    \end{subfigure}
    \hfill
    \caption{Export dependent (0 - 200 TWh) residual load curves with various integration options. Each row represents different integration strategies (RLC without integration, RLC with curtailment, RLC with curtailment and electrolyser load).
    Each column represents different climate ambitions (10 - 90\% emission reduction)}
    \label{fig:diurnal_exportdim}
\end{figure*}



\subsection{Operation of electrolysers and Fischer-Tropsch}


%heatmaps/elec_s_10_ec_lc1.0_Co2L0.10_3H_2030_0.15_DF_40export/cf-ts-AC.pdf
\clearpage
\onecolumn
\begin{figure}
    \centering
        \begin{subfigure}[t]{0.33\textwidth}
            \centering
        \includegraphics[width=\textwidth]{graphics/heatmaps/elec_s_10_ec_lc1.0_Co2L0.80_3H_2030_0.15_DF_40export/cf-ts-AC.pdf}
    \end{subfigure}
    \begin{subfigure}[t]{0.33\textwidth}
        \centering
        \includegraphics[width=\textwidth]{graphics/heatmaps/elec_s_10_ec_lc1.0_Co2L0.30_3H_2030_0.15_DF_40export/cf-ts-AC.pdf}
    \end{subfigure}
    \begin{subfigure}[t]{0.33\textwidth}
        \centering
        \includegraphics[width=\textwidth]{graphics/heatmaps/elec_s_10_ec_lc1.0_Co2L0.10_3H_2030_0.15_DF_40export/cf-ts-AC.pdf}
    \end{subfigure}


    \begin{subfigure}[t]{0.33\textwidth}
        \centering
        \includegraphics[width=\textwidth]{graphics/heatmaps/elec_s_10_ec_lc1.0_Co2L0.80_3H_2030_0.15_DF_40export/cf-ts-H2 Electrolysis.pdf}
    \end{subfigure}
    \begin{subfigure}[t]{0.33\textwidth}
        \centering
        \includegraphics[width=\textwidth]{graphics/heatmaps/elec_s_10_ec_lc1.0_Co2L0.30_3H_2030_0.15_DF_40export/cf-ts-H2 Electrolysis.pdf}
    \end{subfigure}
    \begin{subfigure}[t]{0.33\textwidth}
        \centering
        \includegraphics[width=\textwidth]{graphics/heatmaps/elec_s_10_ec_lc1.0_Co2L0.10_3H_2030_0.15_DF_40export/cf-ts-H2 Electrolysis.pdf}
    \end{subfigure}

    \begin{subfigure}[t]{0.33\textwidth}
        \centering
        \includegraphics[width=\textwidth]{graphics/heatmaps/elec_s_10_ec_lc1.0_Co2L0.80_3H_2030_0.15_DF_40export/cf-ts-H2.pdf}
    \end{subfigure}
    \begin{subfigure}[t]{0.33\textwidth}
        \centering
        \includegraphics[width=\textwidth]{graphics/heatmaps/elec_s_10_ec_lc1.0_Co2L0.30_3H_2030_0.15_DF_40export/cf-ts-H2.pdf}
    \end{subfigure}
    \begin{subfigure}[t]{0.33\textwidth}
        \centering
        \includegraphics[width=\textwidth]{graphics/heatmaps/elec_s_10_ec_lc1.0_Co2L0.10_3H_2030_0.15_DF_40export/cf-ts-H2.pdf}
        
    \end{subfigure}


    \begin{subfigure}[t]{0.33\textwidth}
        \centering
        \includegraphics[width=\textwidth]{graphics/heatmaps/elec_s_10_ec_lc1.0_Co2L0.80_3H_2030_0.15_DF_40export/cf-ts-Fischer-Tropsch.pdf}
        \caption{20\% emission reduction}
        \label{fig:operation20}
    \end{subfigure}
    \begin{subfigure}[t]{0.33\textwidth}
        \centering
        \includegraphics[width=\textwidth]{graphics/heatmaps/elec_s_10_ec_lc1.0_Co2L0.30_3H_2030_0.15_DF_40export/cf-ts-Fischer-Tropsch.pdf}
        \caption{70\% emission reduction}
        \label{fig:operation70}
    \end{subfigure}
    \begin{subfigure}[t]{0.33\textwidth}
        \centering
        \includegraphics[width=\textwidth]{graphics/heatmaps/elec_s_10_ec_lc1.0_Co2L0.10_3H_2030_0.15_DF_40export/cf-ts-Fischer-Tropsch.pdf}
        \caption{90\% emission reduction}
        \label{fig:operation90}
    \end{subfigure}

    \caption{Operation of electrolysers/Fischer-Tropsch and prices of electricity and hydrogen at 40 TWh export. The electrolyser operation depends on the electricity prices, strong diurnal electricity price patterns as in \ref{fig:operation70} result in a corresponding electrolyser operation. Once batteries kick in at \ref{fig:operation90} ($\rightarrow$ proof? See also \ref{fig:battery_operation} for battery operation), resulting smoother electricity prices also increase the annual capacity factor of the electrolyser. Note: electricity and hydrogen prices are spatially averaged in non-weighted manner, style adapted from cf. \cite{Neumann2022}}
    \label{fig:ren-cfs}
\end{figure}

\clearpage
\twocolumn


\subsection{Operation of batteries}


%heatmaps/elec_s_10_ec_lc1.0_Co2L0.10_3H_2030_0.15_DF_40export/cf-ts-AC.pdf
\clearpage
\onecolumn
\begin{figure}
    \centering
        \begin{subfigure}[t]{0.33\textwidth}
            \centering
        \includegraphics[width=\textwidth]{graphics/heatmaps/elec_s_10_ec_lc1.0_Co2L0.80_3H_2030_0.15_DF_40export/cf-ts-battery charger.pdf}
    \end{subfigure}
    \begin{subfigure}[t]{0.33\textwidth}
        \centering
        \includegraphics[width=\textwidth]{graphics/heatmaps/elec_s_10_ec_lc1.0_Co2L0.30_3H_2030_0.15_DF_40export/cf-ts-battery charger.pdf}
    \end{subfigure}
    \begin{subfigure}[t]{0.33\textwidth}
        \centering
        \includegraphics[width=\textwidth]{graphics/heatmaps/elec_s_10_ec_lc1.0_Co2L0.10_3H_2030_0.15_DF_40export/cf-ts-battery charger.pdf}
    \end{subfigure}


    \begin{subfigure}[t]{0.33\textwidth}
        \centering
        \includegraphics[width=\textwidth]{graphics/heatmaps/elec_s_10_ec_lc1.0_Co2L0.80_3H_2030_0.15_DF_40export/cf-ts-battery discharger.pdf}
    \end{subfigure}
    \begin{subfigure}[t]{0.33\textwidth}
        \centering
        \includegraphics[width=\textwidth]{graphics/heatmaps/elec_s_10_ec_lc1.0_Co2L0.30_3H_2030_0.15_DF_40export/cf-ts-battery discharger.pdf}
    \end{subfigure}
    \begin{subfigure}[t]{0.33\textwidth}
        \centering
        \includegraphics[width=\textwidth]{graphics/heatmaps/elec_s_10_ec_lc1.0_Co2L0.10_3H_2030_0.15_DF_40export/cf-ts-battery discharger.pdf}
        
    \end{subfigure}


    \begin{subfigure}[t]{0.33\textwidth}
        \centering
        \includegraphics[width=\textwidth]{graphics/heatmaps/elec_s_10_ec_lc1.0_Co2L0.80_3H_2030_0.15_DF_40export/cf-ts-battery.pdf}
        \caption{20\% emission reduction}
        \label{fig:operation20}
    \end{subfigure}
    \begin{subfigure}[t]{0.33\textwidth}
        \centering
        \includegraphics[width=\textwidth]{graphics/heatmaps/elec_s_10_ec_lc1.0_Co2L0.30_3H_2030_0.15_DF_40export/cf-ts-battery.pdf}
        \caption{70\% emission reduction}
        \label{fig:operation70}
    \end{subfigure}
    \begin{subfigure}[t]{0.33\textwidth}
        \centering
        \includegraphics[width=\textwidth]{graphics/heatmaps/elec_s_10_ec_lc1.0_Co2L0.10_3H_2030_0.15_DF_40export/cf-ts-battery.pdf}
        \caption{90\% emission reduction}
        \label{fig:operation90}
    \end{subfigure}

    \caption{Operation of batteries at 40 TWh export. EV batteries are not included here, but do not play a significant role ($\rightarrow$ TODO doublecheck), style adapted from cf. \cite{Neumann2022}}
    \label{fig:battery_operation}
\end{figure}

\clearpage
\twocolumn



\subsection{Flexibility options}
Figure \ref{fig:integration_options} displays the deployment of battery and hydrogen storage as well as curtailment.

\begin{figure*}[t] % Use 'figure*' to span both columns
    \centering
    \begin{subfigure}[b]{0.45\linewidth}
        \centering
        \includegraphics[width=\linewidth]{graphics/integrated_comp/contour_curtailmentrate_solar_20_filterFalse.pdf}
        \caption{Solar PV curtailment}
        \label{fig:solar_curt}
    \end{subfigure}
    \hfill
    \begin{subfigure}[b]{0.45\linewidth}
        \centering
        \includegraphics[width=\linewidth]{graphics/integrated_comp/contour_curtailmentrate_wind_20_filterFalse.pdf}
        \caption{Wind curtailment}
        \label{fig:wind_curt}
    \end{subfigure}
    \hfill
    \begin{subfigure}[b]{0.45\linewidth}
        \centering
        \includegraphics[width=\linewidth]{graphics/integrated_comp/contour_Battery_GWh_20_filterFalse.pdf}
        \caption{Battery storage capacity}
        \label{fig:battery_cap}
    \end{subfigure}
    \hfill
    \begin{subfigure}[b]{0.45\linewidth}
        \centering
        \includegraphics[width=\linewidth]{graphics/integrated_comp/contour_H2_GWh_20_filterFalse.pdf}
        \caption{Hydrogen storage capacity}
        \label{fig:hystorage_cap}
    \end{subfigure}
    \hfill

    \caption{Curtailment rates and storage capacities}
    \label{fig:integration_options}
\end{figure*}

Figure \ref{fig:barplots} displays the energy balances of modelled scenarios.


\begin{figure}[h!]
    \centering
    \begin{subfigure}[b]{\linewidth}
        \centering
        \includegraphics[trim={0 6.5cm 0 0}, clip, width=\linewidth]{../../workflow/subworkflows/pypsa-earth-sec/results/nresults_full_3H_ws/graphs/balances-AC.pdf}
        \caption{Balances AC}
        \label{fig:balances_AC}
    \end{subfigure}
    
    \vspace{0.5cm} % Add vertical space between the subfigures
    
    \begin{subfigure}[b]{\linewidth}
        \centering
        \includegraphics[trim={0 6.5cm 0 0}, clip, width=\linewidth]{../../workflow/subworkflows/pypsa-earth-sec/results/nresults_full_3H_ws/graphs/balances-energy.pdf}
        \caption{Balances energy}
        \label{fig:balances_energy}
    \end{subfigure}

    \vspace{0.5cm} % Add vertical space between the subfigures
    
    \begin{subfigure}[b]{\linewidth}
        \centering
        \includegraphics[trim={0 6.5cm 0 0}, clip, width=\linewidth]{../../workflow/subworkflows/pypsa-earth-sec/results/nresults_full_3H_ws/graphs/costs.pdf}
        \caption{costs}
        \label{fig:balances_costs}
    \end{subfigure}
    
    \caption{Electricity balance, energy balance and costs of the modelled scenarios.}
    \label{fig:barplots}
\end{figure}



% Figure \ref{fig:land_conflicts} displays the land conflicts arising from separated optimisations.

% \begin{figure}[h!]
%     \centering
%     \includegraphics[width=\linewidth]{../graphics_general/land_conflicts}
%     \caption{Land conflicts arising from separated optimisations}
%     \label{fig:land_conflicts}
% \end{figure}

% Add figure first_results/Export-only-cf-onwind.pdf here

% Figure \ref{fig:cf-onwind} displays the capacity factor of onshore wind.

% \begin{figure}[h!]
%     \centering
%     \includegraphics[width=\linewidth]{../graphics_general/first_results/Export-only-cf-onwind.pdf}
%     \caption{Capacity factor of onshore wind}
%     \label{fig:cf-onwind}
% \end{figure}

% Add figure first_results/Export-only-cf-solar.pdf here

% Figure \ref{fig:cf-solar} displays the capacity factor of solar.

% \begin{figure}[h!]
%     \centering
%     \includegraphics[width=\linewidth]{../graphics_general/first_results/Export-only-cf-solar.pdf}
%     \caption{capacity factor of solar PV}
%     \label{fig:cf-solar}
% \end{figure}

% Add figure first_results/Export-only-high-p-nom-max-onwind.pdf

% Figure \ref{fig:p-nom-max-onwind} displays the potential of onshore wind.

% \begin{figure}[h!]
%     \centering
%     \includegraphics[width=\linewidth]{../graphics_general/first_results/Export-only-p-nom-max-onwind.pdf}
%     \caption{Potential of onshore wind}
%     \label{fig:p-nom-max-onwind}
% \end{figure}

% % Add figure first_results/Export-only-high-p-nom-max-solar.pdf here

% Figure \ref{fig:p-nom-max-solar} displays the potential of solar PV.

% \begin{figure}[h!]
%     \centering
%     \includegraphics[width=\linewidth]{../graphics_general/first_results/Export-only-p-nom-max-solar.pdf}
%     \caption{Potential of solar PV}
%     \label{fig:p-nom-max-solar}
% \end{figure}


% Figure \ref{fig:land_use_solar} displays the land use of solar PV in three different scenarios.

% \begin{figure}[h]
%     \centering
%     \begin{subfigure}[b]{0.3\textwidth}
%         \includegraphics[width=\textwidth]{../fix_co2/graphics/spatial/export_only_land-use_solar_20export.pdf}
%         \caption{Low export}
%         \label{fig:graphic1}
%     \end{subfigure}
%     \begin{subfigure}[b]{0.3\textwidth}
%         \includegraphics[width=\textwidth]{../fix_co2/graphics/spatial/export_only_land-use_solar_140export.pdf}
%         \caption{High export}
%         \label{fig:graphic2}
%     \end{subfigure}
%     \begin{subfigure}[b]{0.3\textwidth}
%         \includegraphics[width=\textwidth]{../fix_co2/graphics/spatial/mar_es_land-use_solar_140export.pdf}
%         \caption{MAR-ES + export}
%         \label{fig:graphic3}
%     \end{subfigure}
%     \caption{(a) Land use in low export scenario, (b) Land use in high export scenario, (c) Land use in MAR-ES + export scenario. Certain regions in
%     (b) are fully exploited, same as in (c). This indicates that the high export scenario and the MAR-ES scenario have competing land resource demands.}
%     \label{fig:land_use_solar}
% \end{figure}
    
% Figure \ref{fig:morocco_water} displays the water supply costs in Morocco.

% \begin{figure}[h!]
%     \centering
%     \includegraphics[width=\linewidth]{../graphics_general/water_supply_costs.png}
%     \caption{Morocco water supply costs}
%     \label{fig:morocco_water}
% \end{figure}


% \begin{figure}[h!]
%     \centering
%     \includegraphics[width=\linewidth]{../graphics_general/graphical_abstract.pdf}
%     \caption{Graphical abstract}
%     \label{fig:graphical_abstract}
% \end{figure}

% \subsection{Separated and integrated optimizations}
% \label{subsec: sep_int_opti}


% \subsubsection{Total system cost}
% Figure \ref{fig:int_sep_total_system_cost} displays the total system cost
% of the integrated vs. islanded optimizations. The total system cost of the
% islanded optimizations are higher.

% \begin{figure}[h!]
%     \centering
%     \includegraphics[width=\linewidth]{../fix_co2/graphics/system_comp/total_system_cost_140.pdf}
%     \caption{Total cost of the three different systems}
%     \label{fig:int_sep_total_system_cost}
% \end{figure}

% \subsection{Misc}

% \subsubsection{Hydrogen export cost (in 2030 and 2050)}
% The cost of hydrogen is lower in the islanded scenario due to best potentials and no transport included?

% \subsubsection{Spatial analysis of (competing) RE resources}
% Both the islanded systems and the Moroccan energy system have the same land resources available. 
% The optimal solutions for both optimisations rely on the same land resources, in combination exceeding the 
% available resources in some regions (s. Fig. \ref{fig:land_conflicts}). 
% Similar to Europe/Germany, public acceptance of Renewable Energies have to be taken in account. 
% They are not unlimited, as discussed in \cite{Hanger2016} and \cite{Terrapon-Pfaff2019}.
% Paper on competing land resources can be found here \cite{Patankar2022}


% \subsubsection{Capacity factors and utilization of assets}
% (Maybe not interesting/necessary)


% \subsubsection{RE Potentials}
% Figure \ref{fig:potentials} displays the potentials of solar, wind onshore and CSP 
% of PyPSA-Earth-Sec compared to other studies. The authors of "OSeMOSYS" 
% \cite{Cannone2021} base their potential estimations on \cite[primary: 13,18,19]{Cannone2021}
% The project MenaFuels/Ersoy2022 \cite{Ersoy2022} finds potentials of ...
% TBD: remove ChatGPT which was added for fun.

% \begin{figure}[h!]
%     \centering
%     \includegraphics[width=\linewidth]{../graphics_general/potentials_comparison/barplot_potentials.pdf}
%     \caption{Determined potential of solar, wind onshore and CSP in PyPSA-Earth-Sec compared to other studies.}
%     \label{fig:potentials}
% \end{figure}


\subsection{Moroccos conventional power plants}
Moroccos conventional power plants are displayed in Table \ref{tab:powerplants}.

% Force one column page
\clearpage
\onecolumn
\begin{footnotesize}
    \begin{table}
\centering
\caption{Conventional power plants in Morocco}
\label{tab:powerplants}
\begin{tabular}{p{1cm}p{6.5cm}rrp{2cm}p{3.4cm}}
\toprule
{} &                                         Name &  Capacity in MW &  Fuel type &  Commission &  Decommission \\
\midrule
6  &                     Jorf Lasfar Jlec Morocco &          1250.2 &  Hard Coal &        1994 &          2039 \\
21 &                      Mohammedia Ocgt Morocco &           925.4 &        Oil &        1981 &          2021 \\
26 &                                     Al Wahda &           800.0 &       CCGT &        2010 &          2050 \\
3  &                         Step Afourer Morocco &           514.5 &      Hydro &        1951 &          2051 \\
25 &  Ain Beni Mathar Centrale Thermosolaire Ccgt &           472.0 &       CCGT &        2009 &          2049 \\
29 &                                      Step Ur &           464.0 &      Hydro &        2005 &          2105 \\
2  &                       Tahaddart Ccgt Morocco &           345.0 &       OCGT &        2005 &          2045 \\
28 &                                   Mohammedia &           300.0 &  Hard Coal &        1986 &          2031 \\
5  &                              Kenitra Morocco &           277.9 &        Oil &        1978 &          2018 \\
12 &                       Allal El Fassi Morocco &           236.9 &      Hydro &        1994 &          2094 \\
13 &                         Al Wahda Dam Morocco &           236.9 &      Hydro &        1997 &          2097 \\
16 &                      Tit Mellil Ocgt Morocco &           189.0 &        Oil &        1993 &          2033 \\
7  &                          Jerada Coal Morocco &           152.1 &  Hard Coal &        1971 &          2016 \\
22 &             Noor Ouarzazate Csp Osps Morocco &           145.8 &      Other &        2016 &          2021 \\
1  &                         Tetouan Ocgr Morocco &           134.3 &        Oil &        1975 &          2015 \\
11 &          Bin El Ouidane Hyroelectric Morocco &           133.3 &      Hydro &        1953 &          2053 \\
14 &                           Al Massira Morocco &           126.4 &      Hydro &        1980 &          2080 \\
27 &                    Centrale Diesel De Tantan &           116.9 &        Oil &        2009 &          2049 \\
24 &                             Tan Ocgt Morocco &            94.5 &        Oil &        1992 &          2032 \\
0  &                     Ahmed El Hansali Morocco &            90.8 &      Hydro &        2003 &          2103 \\
9  &                                       Hassan &            66.3 &      Hydro &        1991 &          2091 \\
8  &                               Idriss Morocco &            39.5 &      Hydro &        1978 &          2078 \\
23 &                    Oued El Makhazine Morocco &            35.5 &      Hydro &        1979 &          2079 \\
17 &                               Imfout Morocco &            31.6 &      Hydro &        1947 &          2047 \\
20 &                              Mohamed Morocco &            22.7 &      Hydro &        1967 &          2067 \\
10 &                              Daourat Morocco &            16.8 &      Hydro &        1950 &          2050 \\
4  &              Sidi Said Maachou Hydro Morocco &            15.4 &      Hydro &        1929 &          2029 \\
19 &                     Mansour Ed Dahbi Morocco &             9.9 &      Hydro &        1973 &          2073 \\
18 &                     Lalla Takerkoust Morocco &             8.7 &      Hydro &        1938 &          2038 \\
15 &                           Al Kensara Morocco &             8.2 &      Hydro &        1946 &          2046 \\
\bottomrule
\end{tabular}
\end{table}

\end{footnotesize}
\clearpage
\twocolumn

\addcontentsline{toc}{section}{Supplementary References}
\renewcommand{\ttdefault}{\sfdefault}
%\bibliographyS{library}
\bibliographyS{/home/fneum/zotero}

\newpage
\begin{small}
	\tableofcontents
\end{small}

\end{document}